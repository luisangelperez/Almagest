\chapter{Glossary}
{\small 
\begin{description}
\item[Altitude:] The angle subtended at the observer by the radius vector connecting a
celestial object to an observer on the earth's surface, and the vector's projection onto the {\bf horizontal plane}. Object's above/below the {\bf horizon}\/ have positive/negative altitudes.
\item[Altitude Circle:] A {\bf great circle}\/ on the celestial sphere which passes through the local {\bf zenith}\/
at a given observation site on the earth's surface.
\item [Anomaly:] Any deviation in an orbit from uniform
circular motion which is concentric with the central body. Anomaly
is also used as another word for angle.
\item [Apocenter:] Point on a {\bf Keplerian orbit}\/ which is furthest from the
central body. If the central body is the sun, then the apocenter is
generally termed the {\em aphelion}. Likewise, if the central body is the
earth, then the apocenter is termed the {\em apogee}.
\item [Arctic Circles:] Two latitude circles on
the earth's surface which are equidistant from the equator. Above the
arctic circles, the sun never sets for part of the year, and never rises
for part of the year.
\item[Argument of Latitude:] Angle subtended at the central body by the radius
vectors connecting the central body to the orbiting body, and the
central body to the {\bf ascending node}, in a {\bf Keplerian
orbit}. 
\item [Ascendent:] Point on {\bf ecliptic circle}\/ which is ascending at any given time on the eastern {\bf horizon}.
\item[Ascending Node:] Point on a {\bf Keplerian orbit}\/ at which the
orbital plane crosses the {\bf ecliptic plane}\/ from south to north
in the direction of motion of the orbiting body.
\item [Autumnal Equinox:] The point at which the 
{\bf ecliptic  circle}\/ crosses the {\bf celestial
equator}\/ from north to south (in the direction of the sun's apparent motion
along the ecliptic).
\item [Azimuth:] Angle subtended at the observer by the projection
of the vector connecting a celestial object to an observer on the  earth's surface
onto the {\bf horizontal plane}, and the vector connecting the
north {\bf compass point}\/ to the observer. Azimuth increases clockwise ({\em i.e.}, from the north to the east) looking at the
horizontal plane from above.
\item[Celestial Axis:] An imaginary extension of the earth's axis
of rotation which pierces the {\bf celestial sphere}\/ at the two
{\bf celestial poles}. The sphere's {\bf diurnal motion}\/ is about this
axis.
\item [Celestial Coordinates:] Angular coordinate system whose fundamental
plane is the
{\bf celestial plane}, and whose poles are the {\bf celestial poles}. 
The polar and azimuthal angles in this system are called {\bf declination}\/ and {\bf right ascension}, respectively.
\item[Celestial Equator:] The intersection of the imagined extension
of the earth's {\bf equatorial plane}\/ with the {\bf celestial sphere}.
\item[Celestial Plane:] The plane containing the earth's equator.
\item [Celestial Poles:] The two points at which the {\bf celestial axis}\/
pierces the {\bf celestial sphere}. The north celestial pole lies to the
north of the {\bf celestial plane}, whereas the south celestial pole lies to the south. The celestial poles are the only two
points on the celestial sphere whose positions are unaffected
by {\bf diurnal motion}.
\item[Celestial Sphere:]  An imaginary sphere of infinite radius which is concentric with the earth. All objects in the sky are thought of as attached to this sphere.
\item[Compass Points:] At a given observation site on the earth's surface, the north, east, south, and west compass points
lie on the local horizon due north, east, south, and west, respectively, of the
 observer.
 \item[Conjunction:] Two celestial objects are said to be in  conjunction when they
 have the same {\bf ecliptic longitude}. For an {\bf inferior planet}\/ in conjunction with the sun, 
 the conjunction is said to be {\em superior}\/ if the planet is further from the earth
 than the sun, and {\em inferior}\/ if the sun is further from the earth
 than the planet. 
\item [Culmination:] A celestial object is said to culminate on a given day when it attains its
maximum {\bf altitude}\/ in the sky. 
\item[Declination:] Angle subtended at the earth's center by the radius vector connecting a
celestial object to the earth's center, and the vector's projection onto the {\bf celestial
plane}. Object's to the north/south of the {\bf celestial
equator}\/ have positive/negative declinations.
\item[Deferent:] Large circle centered on the sun about which the {\bf guide
point} rotates in a {\bf geocentric planetary orbit}.
\item[Deferential Latitude:] {\bf Ecliptic latitude}\/ a {\bf superior planet}\/
has by virtue of the {\bf inclination}\/ of its {\bf deferent}. 
\item[Deferential Latitude Correction Factor:] Correction to the {\bf ecliptic
latitude}\/ of an {\bf inferior planet}\/ due to the finite size of its {\bf deferent}. 
\item [Diurnal Motion:] Daily rotation of the {\bf celestial sphere},
and the objects attached to it, from east to west (looking south in the earth's
northern hemisphere) about the {\bf celestial axis}.
\item [Eccentricity:] Measure of the displacement  along the {\bf major axis}\/ of the central body from the
geometric center in  a {\bf Keplerian orbit}. 
\item [Ecliptic Axis:] Normal to the {\bf ecliptic plane}\/ which
passes through the center of the earth.
\item [Ecliptic Circle:] Apparent path traced out by the sun on the {\bf celestial sphere}\/
during the course of a year.
\item [Ecliptic Coordinates:] Angular coordinate system whose fundamental
plane is the
{\bf ecliptic plane}, and whose poles are the {\bf ecliptic poles}. 
\item[Ecliptic Latitude:] Angle subtended at the earth's center by the radius vector connecting a
celestial object to the earth's center, and the vector's projection onto the {\bf ecliptic
plane}. Objects to the north/south of the {\bf ecliptic circle}\/ have positive/ecliptic latitudes.
\item [Ecliptic Longitude:] Angle subtended at the earth's center by the projection
of the vector connecting a celestial object to the earth's center
onto the {\bf ecliptic plane}, and the vector connecting the
{\bf vernal equinox}\/ to the earth's center. Ecliptic longitude increases counter-clockwise ({\em i.e.}, from the west to the east) looking at the
ecliptic plane from the north.
\item [Ecliptic Plane:] Plane containing the mean orbit of the earth
about the sun.
\item [Ecliptic Poles:] The two points at which the {\bf ecliptic axis}\/
pierces the {\bf celestial sphere}. The north ecliptic pole lies to the north
of the ecliptic plane, whereas the south ecliptic
pole lies to the south.
\item [Elongation:] Difference in {\bf ecliptic longitude}\/ between two
celestial objects.
\item [Epicycle:] Small circle, centered on the {\bf guide point}, about which a planet rotates in a {\bf geocentric planetary orbit}. 
\item [Epicyclic Anomaly:] Angle subtended between the radius vectors
connecting the earth to the {\bf guide-point}, and the guide-point to the
planet, in a {\bf geocentric planetary orbit}. 
\item[Epicyclic Latitude:] {\bf Ecliptic latitude}\/ an {\bf inferior planet}\/
has by virtue of the {\bf inclination}\/ of its {\bf epicycle}.
\item[Epicyclic Latitude Correction Factor:] Correction to the
{\bf ecliptic latitude}\/ of a {\bf superior planet}\/ due to the finite size of 
its {\bf epicycle}.
\item [Epoch:] Standard time at which the {\bf orbital elements}\/ of an orbiting body in the solar system are specified.
\item [Equant:] Point about which the orbiting body appears to
rotate uniformly in a  {\bf Keplerian
orbit}\/ of low {\bf eccentricity}. The equant is diagrammatically opposite the central
body with respect to the geometric center of the orbit.
\item [Equation of Center:] Difference between the {\bf true anomaly}\/ and the {\bf mean anomaly}\/ in a {\bf Keplerian orbit}. 
\item [Equation of Epicycle:] {\bf Elongation}\/ of a planet from its
{\bf guide-point}\/ in a {\bf geocentric planetary orbit}. 
\item [Equation of Time:] Time interval between {\bf local noon}\/
and {\bf mean local noon}.
\item [Equinoxes:] The two opposite points on the {\bf ecliptic circle}\/ which
 the sun reaches on the days of the year that day and night are equally long.
 \item[Evection:] An {\bf anomaly}\/ of the moon's orbit about the earth which is associated
 with the perturbing influence of the sun. 
\item [Geocentric Planetary Orbit:] An orbit in which a planet rotates
about a {\bf guide point}\/ in a small circle called an {\bf epicycle}, and
the guide point rotates about the earth in a large circle called a {\bf deferent}.
\item [Great Circle:] Circle on the surface of a sphere produced by the intersection of a plane
which bisects the sphere.
\item[Greatest Elongation:] Greatest {\bf elongation}\/ of an {\em inferior planet}\/ from the sun. If the planet is to the east/west of the sun then the
elongation is called the greatest eastern/western elongation. 
\item [Guide-Point:] Center of an epicycle in a {\bf geocentric
planetary orbit}. 
\item [Horizon:] Tangent plane to the earth's surface, at a given
observation site, which divides the {\bf celestial sphere}\/ into
visible and invisible hemispheres. 
\item [Horizontal Coordinates:] Angular coordinate system whose fundamental
plane is the
{\bf horizontal plane}, and whose poles are the {\bf zenith} and {\bf nadir}. 
\item[Horizontal Plane:] Plane containing the local horizon.
\item[Horoscope:] Point on the {\bf ecliptic circle}\/ which is ascending
at a given time
on the eastern {\bf horizon}.
\item[Inclination:] Maximum angle subtended between the plane of
a {\bf Keplerian orbit}\/ and the {\bf ecliptic plane}.
\item[Inclination of Ecliptic:] Inclination of the {\bf ecliptic plane}\/
to the {\bf equatorial plane}. 
\item[Inferior Planet:] A planet which is closer to the sun than the earth. 
\item[Julian Day Number:] Number ascribed to a particular day in a
scheme in which days are numbered consecutively from January
1, 4713 BCE, which is designated day zero. Julian days start at
12:00 UT. 
\item [Keplerian Orbit:] Ellipse which
is confocal with the central object. The radius vector
connecting the central and orbiting bodies sweeps out equal
areas in equal time intervals. 
\item [Local Mean Noon:] Instant in time at which the {\bf mean sun}\/
attains its upper {\bf transit}. 
\item[Local Noon:] Instant in time at which the sun attains its
upper {\bf transit}.
\item [Longitude of Ascending Node:] Angle subtended at the central
body by the radius
vectors connecting the central body to the {\bf ascending node}, and the central
body to the {\bf vernal equinox},  in a {\bf Keplerian orbit}.
\item [Longitude of Pericenter:] Angle subtended at the central
body by the
radius vectors connecting the central body to the {\bf pericenter}, and the central body to the {\bf vernal equinox}, in a {\bf Keplerian orbit}.
\item [Major Axis:] Longest diameter which passes through the geometric center of a {\bf Keplerian orbit}.
\item [Major Radius:] Half the length of the {\bf major axis}\/ of a {\bf Keplerian orbit}.
\item [Mean Anomaly:] Angle which would be subtended at the central
body by the radius
vectors connecting the central body to the orbiting body, and the central
body to the
{\bf pericenter}, in a {\bf Keplerian orbit}, if the orbiting body were to rotate about the central body
with a uniform angular velocity.
\item [Mean Argument of Latitude:] Value the
{\bf argument of latitude}\/ would have if the orbiting body in a
{\bf Keplerian orbit}\/ were to rotate about the central body at a fixed angular velocity.
\item [Mean Argument of Latitude at Epoch:]  Value of the {\bf mean argument of latitude}\/ of a {\bf Keplerian orbit}\/
at the {\bf epoch}.
\item[Mean (Ecliptic) Longitude:]  Value the {\bf ecliptic longitude}\/ would
have if the orbiting body in a {\bf Keplerian orbit}\/ were to rotate about the central body at a fixed angular velocity.
\item [Mean (Ecliptic) Longitude at Epoch:] Value of the {\bf mean longitude}\/ of a {\bf Keplerian orbit}\/
at the {\bf epoch}.
\item[Mean Solar Day:] Time interval between successive {\bf local mean noons}.
\item[Mean Solar Time:] Time calculated using the {\bf mean sun}.
\item [Mean Sun:] Fictitious body which travels around the
{\bf celestial equator}\/ (from west to east looking south in the
earth's northern hemisphere) at a uniform rate, and completes one
orbit every {\bf tropical year}. 
\item[Meridian Plane:] Plane passing through the {\bf zenith}\/ and the north and
south {\bf compass points}\/ at a given observation site on the earth's surface.
\item[Minor Axis:] The minor axis of a {\bf Keplerian orbit}\/ is the
shortest diameter which passes through the geometric center.
\item [Minor Radius:] The minor radius of a {\bf Keplerian orbit}\/ 
is half the length of the {\bf minor axis}.
\item [Nadir:] Point on the {\bf celestial sphere}\/ which is
directly underfoot at a given observation site on the earth's surface.
\item[Opposition:] Two celestial objects are said to be in opposition
when their {\bf ecliptic longitudes}\/ differ by $180^\circ$. 
\item [Orbital Elements:] Eight quantities which completely specify
a {\bf Keplerian orbit}: {\em i.e.}, {\bf major radius}, {\bf eccentricity},
{\bf rate of motion of mean longitude}, {\bf rate of motion of mean anomaly}, {\bf mean longitude at epoch}, {\bf mean anomaly at epoch},
{\bf inclination}, {\bf rate of motion in mean argument
of latitude}, {\bf mean argument of latitude at epoch}.
\item [Parallactic Angle:] Angle subtended between the {\bf ecliptic circle}\/
and an {\bf altitude circle}.
\item[Parallax:] Apparent change in position of a nearby celestial object
in the sky when it is viewed at different points on the earth's surface.
\item [Pericenter:] Point on a {\bf Keplerian orbit}\/ which is closest to the
central body. If the central body is the sun, then the pericenter is
generally termed the {\em perihelion}. Likewise, if the central body is the
earth, then the pericenter is termed the {\em perigee}.
\item [Precession of Equinoxes:] A slow movement of the {\bf vernal equinox}\/
relative to the fixed stars which causes the {\bf ecliptic longitude}\/ of a fixed star to increase steadily at the
rate of $50.3''$ per year.
\item [Prograde Motion:] Motion of a {\bf superior
planet}\/ in the sky in the same direction to that of its mean motion.
\item[Radial Anomaly:] Difference between the length of  the radius vector
connecting the central body to the orbiting body, in a {\bf Keplerian
orbit}, and the {\bf major radius}. 
\item [Rate of Motion in Mean Anomaly:] Time derivative of the
{\bf mean anomaly}\/ of a {\bf Keplerian orbit}.
\item [Rate of Motion in Mean Argument of Latitude:] Time
derivative of the {\bf mean argument of latitude}\/ of a
{\bf Keplerian orbit}.
\item [Rate of Motion in Mean Longitude:] Time derivative of
the {\bf mean longitude}\/ of a {\bf Keplerian orbit}.
\item [Retrograde Motion:] Motion of a {\bf superior
planet}\/ in the sky in the opposite direction to that of its mean motion.
\item [Right Ascension:] Angle subtended at the earth's center by the projection
of the vector connecting a celestial body to the earth's center
onto the {\bf celestial plane}, and the vector connecting the
{\bf vernal equinox}\/ to the earth's center. Right ascension increases counter-clockwise ({\em i.e.}, from the west to the east) looking at the
celestial plane from the north.
\item [Seasons:] Spring is the time interval between the {\bf vernal equinox}\/
and the {\bf summer solstice}, summer the  interval between the
summer solstice and the {\bf autumnal equinox}, autumn the interval
between the autumnal equinox and the {\bf winter solstice}, and
winter the interval between the winter solstice and the next spring equinox.
\item [Sidereal Day:] Time interval between successive upper
{\bf transits}\/ of a fixed star. 
\item [Sidereal Time:] Time calculated using the fixed stars. 
\item [Solar Day:] Time interval between successive {\bf local noons}.
\item [Solar Time:] Time calculated using the sun.
\item [Solstices:] The two opposite points on the {\bf ecliptic circle}\/ which  the sun reaches  on the longest and shortest days of the
year.
\item[Station:] Point in the orbit of a {\bf superior planet}\/ at
which it switches from {\bf prograde}\/ to {\bf retrograde}\/ motion,
or {\em vice versa}. The former station is called a {\em retrograde station},
whereas the latter is called a {\em prograde station}. 
\item [Summer Solstice:] Most northerly point on the {\bf ecliptic circle}.
\item [Superior Planet:] A planet further from the sun than the earth.
\item[Synodic Month:] Mean time interval between successive
new moons.
\item[Syzygy:] {\bf Conjunction}\/ or {\bf opposition}\/ of the sun and the moon.
\item [Transit:] On a given day, and at a given observation site on the earth's
surface, a celestial object is said to transit when it crosses the {\bf meridian plane}. The object simultaneously
attains either its highest or lowest altitude in the sky. The transit is called
an upper/lower transit when the object attains its highest/lowest altitude. 
\item [True Anomaly:] Angle  subtended at the central body by the radius
vectors connecting the central body to the orbiting body,  and the central body to the
{\bf pericenter}, in a {\bf Keplerian orbit}.
\item [Tropical Year:] Time interval between successive {\bf vernal equinoxes}.
\item [Tropics:] Two latitude circles on the earth's
surface which are equidistant from the equator. Between the tropics the
sun {\bf culminates}\/ both to the north and south of the {\bf zenith}\/
during the course of a year. Outside the tropics, the sun  culminates
either only to the north  or only to the south of the zenith. 
\item [Universal Time:] Time defined such that {\bf mean local noon}\/
coincides with 12:00 UT every day at an observation site of terrestrial
longitude $0^\circ$. 
\item [Vernal Equinox:] Point at which the {\bf ecliptic circle}\/ crosses the {\bf celestial
equator}\/ from south to north (in the direction of the sun's apparent motion
along the ecliptic).
\item [Winter Solstice:] Most southerly point on the {\bf ecliptic circle}.
\item [Zenith:] Point on the {\bf celestial sphere}\/ which is
directly overhead at a given observation site on the earth's surface.
\item [Zodiac:] The signs of the zodiac are conventional names given
to $30^\circ$ segments of the {\bf ecliptic circle}. 
\end{description}}