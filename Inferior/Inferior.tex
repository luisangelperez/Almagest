\chapter{The Inferior Planets}\label{cinf}
\section{Determination of Ecliptic Longitude}
Figure~\ref{vf8} compares and contrasts heliocentric and geocentric models of the
motion  of an inferior planet ({\em i.e.}, a planet which is closer to the sun than the earth), $P$,  as seen from the earth, $G$. The sun is
at $S$.
As before, in the heliocentric
model  the earth-planet displacement vector, ${\bf P}$,
is the sum of the earth-sun displacement vector, ${\bf S}$, and
the sun-planet displacement vector, ${\bf P}'$.  
On the other hand, in the geocentric model ${\bf S}$ gives the displacement of
the guide-point, $G'$, from the earth. Since ${\bf S}$ is also the displacement of the sun, $S$,
from the earth, $G$, it is clear that $G'$ executes a Keplerian orbit about the earth whose elements
are the same as those of the apparent orbit of the sun about the earth. This implies that the sun is {\em coincident}\/
with $G'$. The ellipse traced out by $G'$ is termed the deferent.
The vector
${\bf P}'$ gives the displacement of the planet, $P$, from the
guide-point, $G'$. 
Since ${\bf P'}$ is also the displacement of the planet, $P$, from the sun, $S$, it is clear
that $P$ executes a Keplerian orbit about the guide-point whose elements are the same as
those of the orbit of the planet about the sun. The ellipse traced out by $P$ about $G'$ is
termed the epicycle.

\begin{figure}[h]
\epsfysize=2.75in
\centerline{\epsffile{epsfiles/fig8.eps}}
\caption[\em Heliocentric and geocentric models of the motion of an inferior planet.]{\em Heliocentric and geocentric models of the motion of an inferior planet. Here, $S$ is the sun, $G$ the earth, and $P$ the planet. View is from the northern ecliptic pole.}\label{vf8}   
\end{figure}

As we have seen,  the deferent of a superior planet has the same elements as the planet's orbit about the sun,
whereas the epicycle has the same elements as the sun's apparent orbit about the earth. On the other
hand, the deferent of an inferior planet has the same elements as the sun's apparent orbit about the earth,
whereas the epicycle has the same elements as the planet's  orbit about the sun. It follows that we can
formulate a procedure for determining the ecliptic longitude of an inferior planet by simply taking the
procedure  used in the  previous section for determining the ecliptic longitude
of a superior planet and exchanging the roles of the sun and the planet.

Our procedure  is described below. As before, it is assumed that the ecliptic longitude, $\lambda_S$, and the
radial anomaly, $\zeta_S$, of the sun have already been calculated.
In the following, $a$, $e$, $n$, $\tilde{n}$, $\bar{\lambda}_0$, and $M_0$ represent  elements of the orbit of the planet in question
about the sun, whereas  $e_S$ is the eccentricity 
of the sun's apparent orbit about the earth.
Again, $a$ is the major radius of the planetary orbit in
units in which the major radius of the sun's apparent orbit about the
earth is unity. The requisite elements for all of the inferior planets at the J2000 epoch ($t_0=2\,451\,545.0$ JD)
are listed in Table~\ref{lt4}. The ecliptic longitude of an inferior planet
is specified by the following formulae:
\begin{eqnarray}
\bar{\lambda}&=&  \bar{\lambda}_0+ n\,(t-t_0) ,\\[0.5ex]
M &=& M_{0}  +\tilde{n}\,(t-t_0),\\[0.5ex]
q&=& 2\,e\,\sin \,M + (5/4)\,e^{\,2}\,\sin\,2M,\\[0.5ex]
\zeta &=& e\,\cos M - e^{\,2}\,\sin^2 M,\\[0.5ex]
\mu&=& \bar{\lambda}+q-\lambda_S,\\[0.5ex]
\bar{\theta} &=& \theta(\mu,\bar{z})\equiv \tan^{-1} \left(\frac{\sin\mu}{a^{-1}\,\bar{z}+\cos\mu}\right),\\[0.5ex]
\delta\theta_- &=& \theta(\mu,\bar{z}) - \theta(\mu,z_{\rm max}),\\[0.5ex]
\delta\theta_+&=& \theta(\mu,z_{\rm min}) - \theta(\mu,\bar{z}),\\[0.5ex]
z &=& \frac{1-\zeta_S}{1-\zeta},\label{e182x}\\[0.5ex]
\xi &=& \frac{\bar{z}-z}{\delta z},\\[0.5ex]
\theta  &=&\Theta_-(\xi)\,\delta\theta_-+ \bar{\theta}
+ \Theta_+(\xi)\,\delta\theta_+,\\[0.5ex]
\lambda &=&\lambda_S+ \theta.
\end{eqnarray}
Here, $\bar{z} = (1+e\,e_S)/(1-e^{\,2})$, $\delta z = (e+e_S)/(1-e^{\,2})$, $z_{\rm min} = \bar{z}-\delta z$,
and $z_{\rm max} = \bar{z}+\delta z$. The constants $\bar{z}$, $\delta z$, $z_{\rm min}$, and $z_{\rm max}$ 
for  each of the inferior planets are listed in Table.~\ref{tx}. Finally, the functions $\Theta_\pm$ are tabulated in Table~\ref{ty}.

For the case of Venus, the above formulae are capable of matching NASA ephemeris data during the years 1995--2006 CE
with a mean error of $2'$ and a maximum error of $10'$. For the case of Mercury, given its relatively large eccentricity of
$0.205636$, it is necessary to modify the formulae slightly by expressing $q$ and $\zeta$ to
third-order in the eccentricity:
\begin{eqnarray}
q &=& [2\,e - (1/4)\,e^3]\,\sin \,M + (5/4) \,e^2\,sin\,2M + (13/12)\,e^3\,\sin \,3M,\\[0.5ex]
\zeta &=& -(1/2)\,e^2+ [e-(3/8)\,e^3]\,\cos\,M + (1/2)\,e^2\,\cos\,2M + (3/8)\,e^3\,\cos\,3M.
\end{eqnarray}
With this modification, the mean error is $6'$ and the maximum error $28'$. 

\section{Venus}
The ecliptic longitude of Venus can be determined with the aid of Tables~\ref{vt17}--\ref{vt19}. Table~\ref{vt17} allows
the mean longitude, $\bar{\lambda}$, and the mean anomaly, $M$, of Venus to be calculated as functions of
time. Next, Table~\ref{vt18} permits the equation of center, $q$, and the radial anomaly, $\zeta$, to
be determined as functions of the mean anomaly. Finally, Table~\ref{vt19} allows the quantities
$\delta\theta_-$, $\bar{\theta}$, and $\delta\theta_+$ to be calculated as functions of the epicyclic
anomaly, $\mu$. 

The procedure for using the tables is as follows:
\begin{enumerate}
\item Determine the fractional Julian day number, $t$, corresponding to the date and time
at which the  ecliptic longitude is to be calculated  with the aid of Tables~\ref{kt1}--\ref{kt3}. Form $\Delta t = t-t_0$, where $t_0=2\,451\,545.0$ is the epoch. 
\item Calculate the ecliptic longitude, $\lambda_S$, and radial anomaly,
$\zeta_S$, of the sun using the procedure set out in Sect.~\ref{ssun}.
\item Enter Table~\ref{vt17} with the digit for each power of 10
in ${\Delta} t$ and take out the corresponding values of $\Delta\bar{\lambda}$ and $\Delta M$. If $\Delta t$ is negative then the corresponding
values are also negative.
The value of the mean longitude, $\bar{\lambda}$, is the
sum of all the $\Delta\bar{\lambda}$ values plus value of $\bar{\lambda}$ at the epoch. Likewise, the value of the mean anomaly, $M$, is
the sum of all the $\Delta M$ values plus the value of $M$ at the epoch. 
Add as many multiples of $360^\circ$ to $\bar{\lambda}$ and $M$
as is required to make them both fall in the range $0^\circ$ to $360^\circ$. Round $M$ to the nearest degree. 
\item Enter Table~\ref{vt18} with the value of $M$ and take out the
corresponding value of the equation of center, $q$, and the radial anomaly, $\zeta$. It is necessary to interpolate if $M$ is odd.
\item Form the epicyclic anomaly, $\mu = \bar{\lambda}+q-\lambda_S$. Add as many multiples of $360^\circ$ to $\mu$ as is required to make it fall in the range $0^\circ$ to $360^\circ$. Round $\mu$ to the nearest degree.
\item Enter Table~\ref{vt19} with the value of $\mu$ and take
out the corresponding values of $\delta\theta_-$, $\bar{\theta}$, and
$\delta\theta_+$. If $\mu > 180^\circ$ then it is necessary to make use
of the identities $\delta\theta_\pm(360^\circ - \mu) =-\delta\theta_\pm(\mu)$
and $\bar{\theta}(360^\circ - \mu) =-\bar{\theta}(\mu)$.
\item Form $z = (1-\zeta_S)/(1-\zeta)$.
\item Obtain the values of $\bar{z}$ and $\delta z$ from Table~\ref{tx}.
Form $\xi = (\bar{z}-z)/\delta z$.
\item Enter Table~\ref{ty} with the value of $\xi$ and take out
the corresponding values of $\Theta_-$ and $\Theta_+$. If $\xi<0$ then
it is necessary to use the identities $\Theta_+(\xi)=-\Theta_-(-\xi)$
and $\Theta_-(\xi)=-\Theta_+(-\xi)$. 
\item Form the equation of the epicycle, $\theta = \Theta_-\,\delta\theta_-+ \bar{\theta}
+ \Theta_+\,\delta\theta_+$.
\item The ecliptic longitude, $\lambda$, is the sum of the ecliptic longitude of the sun, $\lambda_S$,   and the equation
of the epicycle, $\theta$. If necessary convert $\lambda$
into  an angle in the range $0^\circ$ to $360^\circ$. The decimal fraction can
be converted into arc minutes
using Table~\ref{lt6a}. Round to the nearest arc minute. The final result
can be written in terms of the signs of the zodiac using the table in Sect.~\ref{szod}.
\end{enumerate}
Two examples of this procedure are given below.

~\\
\noindent {\em Example 1}: May 5,  2005 CE, 00:00 UT:\\
~\\
From Cha.~\ref{csup}, $t-t_0=1\,950.5$ JD, $\lambda_S= 44.602^\circ$, and
$\zeta_S= -8.56\times 10^{-3}$. Making use of
Table~\ref{vt17}, we find:\\
\begin{tabular}{rrr}
&&\\
$t$(JD) & $ \bar{\lambda}(^\circ)$ & $M(^\circ)$\\[-2ex]
&&\\
+1000 & $162.169$ & $162.130$\\
+900 & $1.952$ & $1.917$\\
+50 & $80.108$ & $80.107$\\
+.5 & $0.801$ & $0.801$\\
Epoch & $181.973$ & $49.237$\\\cline{2-3}
&$427.003$ & $294.192$\\\cline{2-3}
Modulus & $67.003$ & $294.192$\\ 
&&\\
\end{tabular}\\
Given that $M\simeq 294^\circ$, Table~\ref{vt18} yields 
$$
q(294^\circ)= -0.712^\circ, \mbox{\hspace{0.5cm}}\zeta(294^\circ)=2.72\times 10^{-3},
$$
so
$$
\mu=\bar{\lambda}+q-\bar{\lambda}_S = 67.003-0.712-44.602= 21.689\simeq
22^\circ.
$$
It follows from Table~\ref{vt19}
that 
$$
\delta\theta_-(22^\circ) = 0.126^\circ,\mbox{\hspace{0.5cm}}\bar{\theta}(22^\circ)=9.212^\circ, \mbox{\hspace{0.5cm}}\delta\theta_+(22^\circ) = 0.129^\circ.
$$
Now, 
$$
z= (1-\zeta_S)/(1-\zeta) = (1+8.56\times 10^{-3})/(1-2.72\times 10^{-3}) =
1.01131.
$$
However, from Table~\ref{tx}, $\bar{z}= 1.00016$ and $\delta z = 0.02349$, so
$$
\xi = (\bar{z}-z)/\delta z = (1.00016-1.01131)/0.02349 \simeq -0.48.
$$
According to Table~\ref{ty}, 
$$
\Theta_-(-0.48) = -0.355, \mbox{\hspace{0.5cm}}\Theta_+(-0.48) = -0.125,
$$
so
$$
\theta  = \Theta_-\,\delta\theta_- + \bar{\theta}+\Theta_+\,\delta\theta_+ = -0.355\times 0.126 + 9.212-0.125\times 0.129 = 9.151^\circ.
$$
Finally,
$$
\lambda=\bar{\lambda}_S + \theta= 44.602+9.151=53.753 \simeq 53^\circ 45'. 
$$
Thus,
the ecliptic longitude of Venus at 00:00 UT on May 5, 2005 CE was 23TA45.

~\\
\noindent {\em Example 2}: December 25,  1800 CE, 00:00 UT:\\
~\\
From Cha.~\ref{csup}, $t-t_0=-72\,690.5$ JD, $\lambda_S= 273.055^\circ$, and $\zeta_S= 1.662\times 10^{-2}$. Making use of
Table~\ref{vt17}, we find:\\
\begin{tabular}{rrr}
&&\\
$t$(JD) & $\bar{\lambda}(^\circ)$ & $ M(^\circ)$\\[-2ex]
&&\\
-70,000 & $-191.810$ & $-189.128$\\
-2,000 & $-324.337$ & $-324.261$\\
-600 & $-241.301$ & $-241.278$\\
-90 & $-144.195$ & $-144.192$\\
-.5 & $-0.801$ & $-0.801$\\
Epoch & $181.973$ & $49.237$\\\cline{2-3}
&$-720.471$ & $-850.423$\\\cline{2-3}
Modulus & $359.529$ & $229.577$\\
&&\\
\end{tabular}\\
Given that $M\simeq 230^\circ$, Table~\ref{vt18} yields 
$$
q(230^\circ)= -0.592^\circ,\mbox{\hspace{0.5cm}}\zeta(230^\circ)=-4.38\times 10^{-3},
$$
so
$$
\mu=\bar{\lambda}+q-\bar{\lambda}_S =359.529-0.592-273.055= 85.882\simeq 86^\circ.
$$
It follows from Table~\ref{vt19}
that 
$$
\delta\theta_-(86^\circ) = 0.589^\circ, \mbox{\hspace{0.5cm}}\bar{\theta}(86^\circ)=34.482^\circ, \mbox{\hspace{0.5cm}}\delta\theta_+(86^\circ) = 0.607^\circ.
$$
 Now, 
 $$
 z= (1-\zeta_S)/(1-\zeta) = (1-1.662\times 10^{-2})/(1+4.38\times 10^{-3}) =
0.97909,
$$
so
$$
\xi = (\bar{z}-z)/\delta z = (1.00016-0.97909)/0.02349 \simeq 0.90.
$$
According to Table~\ref{ty}, 
$$
\Theta_-(0.90) = 0.045,\mbox{\hspace{0.5cm}}\Theta_+(0.90) = 0.855,
$$
so
$$
\theta  = \Theta_-\,\delta\theta_- + \bar{\theta}+\Theta_+\,\delta\theta_+ = 0.045\times 0.589+34.482+0.855\times 0.607 = 35.027^\circ.
$$
Finally,
$$
\lambda=\bar{\lambda}_S  + \theta= 273.055+35.027= 308.082 \simeq 308^\circ 5'.
$$
Thus,
the ecliptic longitude of Venus at 00:00 UT on December 25, 1800 CE was 8AQ5.

\section{Determination of Conjunction and Greatest Elongation Dates}
The geocentric orbit of an inferior planet is similar to that of the
superior planet shown in Fig.~\ref{vf5x}, except for the fact that the sun
is coincident with guide-point $G'$ in the former case.
It follows that it is impossible
for an inferior planet to have an opposition with the sun ({\em i.e}, for the
earth to lie directly between the planet and the sun). However, inferior planets
do have two different kinds of conjunctions with the sun. A {\em superior
conjuction}\/ takes place when the sun lies directly between the planet and the earth.
Conversely, an {\em inferior conjunction}\/ takes place when the planet lies
directly between the sun and the earth. It is clear from Fig.~\ref{vf5x}
that a superior conjunction corresponds to $\mu=0^\circ$, and
an inferior conjunction to $\mu=180^\circ$. Now, the equation of the epicycle,
 $\theta$, measures the
angular separation between the planet and the sun (since the sun lies at
the guide-point). It is evident from Figure~\ref{vf5x} that $\theta$ attains
a maximum and a minimum value each time the planet revolves
around its epicycle.
In other words, there is a limit to how large the angular separation between
an inferior planet and the sun can become.
The maximum value is termed the {\em greatest eastern elongation}\/
of the planet, whereas the modulus of the minimum value is
termed the {\em greatest western elongation}. 

Tables~\ref{vt17}--\ref{vt19} can be used to determine the dates of the conjunctions  and greatest  elongations
of Venus. Consider the first superior conjunction after the epoch (January 1, 2000 CE). We can estimate the
time at which this event occurs by approximating the epicyclic anomaly as the 
mean epicyclic anomaly: 
$$
\mu \simeq \bar{\mu} = \bar{\lambda}-\bar{\lambda}_S = \bar{\lambda}_0-\bar{\lambda}_{0\,S} +
(n-n_S)\,(t-t_0) = 261.515 + 0.61652137\,(t-t_0).
$$
 Thus, 
$$
 t\simeq t_0 + (360-261.515)/0.61652137\simeq
t_0 + 160\, {\rm JD}.
$$
A calculation of the epicyclic anomaly at this time, using Tables~\ref{vt17}--\ref{vt19}, yields $\mu = -1.267^\circ$. Now, the actual conjunction
takes place when $\mu=0^\circ$. 
Hence, our final estimate is 
$$
t=t_0+160+1.267/0.61652137= t_0 + 162.1\,{\rm JD}, 
$$
which corresponds to June 11, 2000 CE.

Consider the first inferior conjunction of Venus after the epoch. Our first estimate of the time at which this
event takes place is 
$$
t\simeq t_0+(540-261.515)/0.61652137\simeq t_0 + 452\,{\rm JD}.
$$
A calculation of
the epicyclic anomaly at this time yields $\mu=178.900^\circ$. Now, the
actual conjunction takes place when $\mu=180^\circ$. 
 Hence, our final estimate is
$$
t = t_0 +452+1.100/0.61652137 = t_0+453.8\,{\rm JD},
$$
which corresponds to
March 30, 2001 CE. Incidentally, it is clear from the above analysis that the
{\em mean}\/ time period between successive superior, or  inferior, conjunctions of Venus is $360/0.61652137= 583.9$ JD, which is
equivalent to $1.60$ years.

Consider the greatest elongations of Venus. We can approximate the
equation of the epicycle as
\begin{equation}\label{vexyz}
\theta\simeq \bar{\theta} = \tan^{-1}\left(\frac{\sin \bar{\mu}}{\bar{a}^{-1}+\cos\bar{\mu}}\right),
\end{equation}
where $\bar{\mu}$ is the mean epicyclic anomaly, and $\bar{a}=a/\bar{z}$. 
It follows that
\begin{equation}
\frac{d\bar{\theta}}{d\bar{\mu}} = \frac{\bar{a}^{-1}\,\cos\bar{\mu}+1}
{1 + 2\,\bar{a}^{-1}\,\cos\bar{\mu} + \bar{a}^{-2}}.
\end{equation}
Now, $\bar{\theta}$ attains its maximum or minimum value when
$d\bar{\theta}/d\bar{\mu}=0$: {\em i.e.}, when
\begin{equation}
\bar{\mu} = \cos^{-1}(-\bar{a}).
\end{equation}
For the case of Venus, we obtain $\bar{\mu} = 136.3^\circ$ or $223.7^\circ$. 
The first solution corresponds to the greatest eastern elongation, and the
second to the greatest western elongation. Substituting back into Eq.~(\ref{vexyz}), we find that $\bar{\theta} = \pm 46.3^\circ$.
Hence, the mean value of the greatest eastern or western
elongation of Venus is $46.3^\circ$. The mean time period
between a greatest eastern elongation and the following
inferior conjunction, or between an inferior conjunction and the
following greatest western elongation, is $(180-136.3)/0.61652137\simeq
71$ JD. Unfortunately, the only option for accurately determining the  dates at which the greatest elongations occur is to calculate
the equation of the epicycle of Venus over a range of days centered 71 days  before and after an inferior conjunction.

Table~\ref{vtvenus} shows the conjunctions, and greatest elongations
of Venus for the years 2000--2015 CE, calculated using the
techniques described above.

\section{Mercury}
The ecliptic longitude of Mercury can be determined with the aid of Tables~\ref{vt20}--\ref{vt22}. Table~\ref{vt20} allows
the mean longitude, $\bar{\lambda}$, and the mean anomaly, $M$, of Mercury to be calculated as functions of
time. Next, Table~\ref{vt21} permits the equation of center, $q$, and the radial anomaly, $\zeta$, to
be determined as functions of the mean anomaly. Finally, Table~\ref{vt22} allows the quantities
$\delta\theta_-$, $\bar{\theta}$, and $\delta\theta_+$ to be calculated as functions of the epicyclic
anomaly, $\mu$.  The procedure for using the tables is analogous to the previously described procedure for
using the Venus tables.
One example of this procedure is given below.

~\\
\noindent {\em Example}: May 5,  2005 CE, 00:00 UT:\\
~\\
From Cha.~\ref{csup}, $t-t_0=1\,950.5$ JD, $\lambda_S= 44.602^\circ$, and
$\zeta_S= -8.56\times 10^{-3}$. Making use of
Table~\ref{vt20}, we find:\\
\begin{tabular}{rrr}
&&\\
$t$(JD) & $ \bar{\lambda}(^\circ)$ & $M(^\circ)$\\[-2ex]
&&\\
+1000 & $132.377$ & $132.334$\\
+900 & $83.139$ & $83.101$\\
+50 & $204.619$ & $204.617$\\
+.5 & $2.046$ & $2.046$\\
Epoch & $252.087$ & $174.693$\\\cline{2-3}
&$647.268$ & $596.791$\\\cline{2-3}
Modulus & $314.268$ & $236.791$\\ 
&&\\
\end{tabular}\\
Given that $M\simeq 237^\circ$, Table~\ref{vt21} yields 
$$
q(237^\circ)= -16.974^\circ, \mbox{\hspace{0.5cm}}\zeta(237^\circ)=-1.367\times 10^{-1},
$$
so
$$
\mu=\bar{\lambda}+q-\bar{\lambda}_S = 314.268-16.974-44.602= 252.692\simeq
253^\circ.
$$
It follows from Table~\ref{vt22}
that 
$$
\delta\theta_-(253^\circ) = -4.005^\circ,\mbox{\hspace{0.5cm}}\bar{\theta}(253^\circ)=-21.609^\circ, \mbox{\hspace{0.5cm}}\delta\theta_+(253^\circ) = -6.182^\circ.
$$
Now, 
$$
z= (1-\zeta_S)/(1-\zeta) = (1+8.56\times 10^{-3})/(1+1.367\times 10^{-1}) =
0.8873.
$$
However, from Table~\ref{tx}, $\bar{z}= 1.04774$ and $\delta z = 0.23216$, so
$$
\xi = (\bar{z}-z)/\delta z = (1.04774-0.8873)/0.23216 \simeq 0.69.
$$
According to Table~\ref{ty}, 
$$
\Theta_-(0.69) = 0.107, \mbox{\hspace{0.5cm}}\Theta_+(0.69) = 0.583,
$$
so
$$
\theta  = \Theta_-\,\delta\theta_- + \bar{\theta}+\Theta_+\,\delta\theta_+ = -0.107\times 4.005 -21.609-0.583\times 6.182 = -25.642^\circ.
$$
Finally,
$$
\lambda=\bar{\lambda}_S + \theta= 44.602-25.642=18.960 \simeq 18^\circ 58'. 
$$
Thus,
the ecliptic longitude of Mercury at 00:00 UT on May 5, 2005 CE was 18AR58.

The conjunctions  and elongations of Mercury can be investigated
using analogous methods to those employed earlier to examine the
conjunctions and elongations of Venus. We find that the mean
time period between successive superior, or inferior, conjunctions of
Mercury is 116 days. On average, the greatest eastern and
western elongations of Mercury occur when the epicyclic anomaly takes the
values $\mu=111.7^\circ$ and $248.3^\circ$, respectively. Furthermore,
the mean value of the greatest eastern or western elongation is $21.7^\circ$.
 Finally,
the mean time period between a greatest eastern elongation  and the following
inferior conjunction, or between the inferior conjunction and the following greatest western elongation, is 22 JD. The conjunctions and elongations of Mercury
during the years 2000--2002 CE are shown in Table~\ref{vtmercury}.

\clearpage
\newpage
\begin{table}
\centering
\begin{tabular}{rrrr|rrrr}
$\Delta t$(JD)& $\Delta\bar{\lambda}(^\circ)$ &  $\Delta M(^\circ)$ & $\Delta \bar{F}(^\circ)$& $\Delta t$(JD) & $\Delta\bar{\lambda}(^\circ)$ & $\Delta M(^\circ)$ 
&$\Delta \bar{F}(^\circ)$\\ \hline
&&&&&&&\\[-1.75ex]
10,000 & 181.687 & 181.304 & 181.381 & 1,000 & 162.169 & 162.130 & 162.138\\
20,000 &   3.374 &   2.608 &   2.761 & 2,000 & 324.337 & 324.261 & 324.276\\
30,000 & 185.062 & 183.912 & 184.142 & 3,000 & 126.506 & 126.391 & 126.414\\
40,000 &   6.749 &   5.216 &   5.523 & 4,000 & 288.675 & 288.522 & 288.552\\
50,000 & 188.436 & 186.520 & 186.904 & 5,000 &  90.844 &  90.652 &  90.690\\
60,000 &  10.123 &   7.824 &   8.284 & 6,000 & 253.012 & 252.782 & 252.828\\
70,000 & 191.810 & 189.128 & 189.665 & 7,000 &  55.181 &  54.913 &  54.966\\
80,000 &  13.498 &  10.432 &  11.046 & 8,000 & 217.350 & 217.043 & 217.105\\
90,000 & 195.185 & 191.736 & 192.426 & 9,000 &  19.518 &  19.174 &  19.243\\
&&&&&&&\\
100 & 160.217 & 160.213 & 160.214 & 10 &  16.022 &  16.021 &  16.021\\
200 & 320.434 & 320.426 & 320.428 & 20 &  32.043 &  32.043 &  32.043\\
300 & 120.651 & 120.639 & 120.641 & 30 &  48.065 &  48.064 &  48.064\\
400 & 280.867 & 280.852 & 280.855 & 40 &  64.087 &  64.085 &  64.086\\
500 &  81.084 &  81.065 &  81.069 & 50 &  80.108 &  80.107 &  80.107\\
600 & 241.301 & 241.278 & 241.283 & 60 &  96.130 &  96.128 &  96.128\\
700 &  41.518 &  41.491 &  41.497 & 70 & 112.152 & 112.149 & 112.150\\
800 & 201.735 & 201.704 & 201.710 & 80 & 128.173 & 128.170 & 128.171\\
900 &   1.952 &   1.917 &   1.924 & 90 & 144.195 & 144.192 & 144.192\\
&&&&&&&\\
1 &   1.602 &   1.602 &   1.602 & 0.1 &   0.160 &   0.160 &   0.160\\
2 &   3.204 &   3.204 &   3.204 & 0.2 &   0.320 &   0.320 &   0.320\\
3 &   4.807 &   4.806 &   4.806 & 0.3 &   0.481 &   0.481 &   0.481\\
4 &   6.409 &   6.409 &   6.409 & 0.4 &   0.641 &   0.641 &   0.641\\
5 &   8.011 &   8.011 &   8.011 & 0.5 &   0.801 &   0.801 &   0.801\\
6 &   9.613 &   9.613 &   9.613 & 0.6 &   0.961 &   0.961 &   0.961\\
7 &  11.215 &  11.215 &  11.215 & 0.7 &   1.122 &   1.121 &   1.121\\
8 &  12.817 &  12.817 &  12.817 & 0.8 &   1.282 &   1.282 &   1.282\\
9 &  14.420 &  14.419 &  14.419 & 0.9 &   1.442 &   1.442 &   1.442\\
\end{tabular}
\caption[\em Mean motion of Venus. ]{\em Mean motion of Venus.  Here, $\Delta t = t-t_0$, $\Delta\bar{\lambda} = \bar{\lambda}-\bar{\lambda}_0$,  $\Delta M = M - M_0$, and $\Delta\bar{F} = \bar{F}-\bar{F}_0$.  At epoch  ($t_0 = 2\,451\,545.0$ JD), $\bar{\lambda}_0 = 181.973^\circ$, $M_0 = 49.237^\circ$, and
$\bar{F}_0 =105.253^\circ$. }\label{vt17}
\end{table}

\newpage
\begin{table}\centering
\small{ \begin{tabular}{rrr|rrr|rrr|rrr}
$M(^\circ)$ & $q(^\circ)$  & $100\,\zeta$ & $M(^\circ)$ & $q(^\circ)$  & $100\,\zeta$ & $M(^\circ)$ & $q(^\circ)$  & $100\,\zeta$& $M(^\circ)$ & $q(^\circ)$  & $100\,\zeta$\\\hline
&&&&&&&&&&&\\[-1.75ex]
   0 &   0.000 &  0.678 &  90 &   0.777 & -0.005 & 180 &   0.000 & -0.678 & 270 &  -0.777 & -0.005\\
  2 &   0.027 &  0.677 &  92 &   0.776 & -0.028 & 182 &  -0.027 & -0.677 & 272 &  -0.776 &  0.019\\
  4 &   0.055 &  0.676 &  94 &   0.774 & -0.052 & 184 &  -0.054 & -0.676 & 274 &  -0.775 &  0.043\\
  6 &   0.082 &  0.674 &  96 &   0.772 & -0.075 & 186 &  -0.080 & -0.674 & 276 &  -0.773 &  0.066\\
  8 &   0.109 &  0.671 &  98 &   0.768 & -0.099 & 188 &  -0.107 & -0.671 & 278 &  -0.770 &  0.090\\
 10 &   0.136 &  0.667 & 100 &   0.764 & -0.122 & 190 &  -0.134 & -0.668 & 280 &  -0.766 &  0.113\\
 12 &   0.163 &  0.663 & 102 &   0.758 & -0.145 & 192 &  -0.160 & -0.663 & 282 &  -0.761 &  0.137\\
 14 &   0.189 &  0.657 & 104 &   0.752 & -0.168 & 194 &  -0.186 & -0.658 & 284 &  -0.755 &  0.160\\
 16 &   0.216 &  0.651 & 106 &   0.745 & -0.191 & 196 &  -0.212 & -0.652 & 286 &  -0.748 &  0.183\\
 18 &   0.242 &  0.644 & 108 &   0.737 & -0.214 & 198 &  -0.238 & -0.645 & 288 &  -0.741 &  0.205\\
 20 &   0.268 &  0.636 & 110 &   0.728 & -0.236 & 200 &  -0.263 & -0.637 & 290 &  -0.732 &  0.228\\
 22 &   0.293 &  0.628 & 112 &   0.718 & -0.258 & 202 &  -0.289 & -0.629 & 292 &  -0.722 &  0.250\\
 24 &   0.318 &  0.618 & 114 &   0.707 & -0.279 & 204 &  -0.313 & -0.620 & 294 &  -0.712 &  0.272\\
 26 &   0.343 &  0.608 & 116 &   0.695 & -0.301 & 206 &  -0.338 & -0.610 & 296 &  -0.701 &  0.293\\
 28 &   0.367 &  0.597 & 118 &   0.683 & -0.322 & 208 &  -0.362 & -0.599 & 298 &  -0.688 &  0.315\\
 30 &   0.391 &  0.586 & 120 &   0.670 & -0.342 & 210 &  -0.385 & -0.588 & 300 &  -0.675 &  0.335\\
 32 &   0.414 &  0.573 & 122 &   0.656 & -0.362 & 212 &  -0.409 & -0.576 & 302 &  -0.662 &  0.356\\
 34 &   0.437 &  0.560 & 124 &   0.641 & -0.382 & 214 &  -0.431 & -0.563 & 304 &  -0.647 &  0.376\\
 36 &   0.460 &  0.547 & 126 &   0.625 & -0.401 & 216 &  -0.453 & -0.550 & 306 &  -0.631 &  0.395\\
 38 &   0.481 &  0.532 & 128 &   0.609 & -0.420 & 218 &  -0.475 & -0.536 & 308 &  -0.615 &  0.414\\
 40 &   0.502 &  0.517 & 130 &   0.592 & -0.438 & 220 &  -0.496 & -0.521 & 310 &  -0.598 &  0.433\\
 42 &   0.523 &  0.502 & 132 &   0.574 & -0.456 & 222 &  -0.516 & -0.506 & 312 &  -0.580 &  0.451\\
 44 &   0.543 &  0.485 & 134 &   0.555 & -0.473 & 224 &  -0.536 & -0.490 & 314 &  -0.562 &  0.468\\
 46 &   0.562 &  0.468 & 136 &   0.536 & -0.490 & 226 &  -0.555 & -0.473 & 316 &  -0.543 &  0.485\\
 48 &   0.580 &  0.451 & 138 &   0.516 & -0.506 & 228 &  -0.574 & -0.456 & 318 &  -0.523 &  0.502\\
 50 &   0.598 &  0.433 & 140 &   0.496 & -0.521 & 230 &  -0.592 & -0.438 & 320 &  -0.502 &  0.517\\
 52 &   0.615 &  0.414 & 142 &   0.475 & -0.536 & 232 &  -0.609 & -0.420 & 322 &  -0.481 &  0.532\\
 54 &   0.631 &  0.395 & 144 &   0.453 & -0.550 & 234 &  -0.625 & -0.401 & 324 &  -0.460 &  0.547\\
 56 &   0.647 &  0.376 & 146 &   0.431 & -0.563 & 236 &  -0.641 & -0.382 & 326 &  -0.437 &  0.560\\
 58 &   0.662 &  0.356 & 148 &   0.409 & -0.576 & 238 &  -0.656 & -0.362 & 328 &  -0.414 &  0.573\\
 60 &   0.675 &  0.335 & 150 &   0.385 & -0.588 & 240 &  -0.670 & -0.342 & 330 &  -0.391 &  0.586\\
 62 &   0.688 &  0.315 & 152 &   0.362 & -0.599 & 242 &  -0.683 & -0.322 & 332 &  -0.367 &  0.597\\
 64 &   0.701 &  0.293 & 154 &   0.338 & -0.610 & 244 &  -0.695 & -0.301 & 334 &  -0.343 &  0.608\\
 66 &   0.712 &  0.272 & 156 &   0.313 & -0.620 & 246 &  -0.707 & -0.279 & 336 &  -0.318 &  0.618\\
 68 &   0.722 &  0.250 & 158 &   0.289 & -0.629 & 248 &  -0.718 & -0.258 & 338 &  -0.293 &  0.628\\
 70 &   0.732 &  0.228 & 160 &   0.263 & -0.637 & 250 &  -0.728 & -0.236 & 340 &  -0.268 &  0.636\\
 72 &   0.741 &  0.205 & 162 &   0.238 & -0.645 & 252 &  -0.737 & -0.214 & 342 &  -0.242 &  0.644\\
 74 &   0.748 &  0.183 & 164 &   0.212 & -0.652 & 254 &  -0.745 & -0.191 & 344 &  -0.216 &  0.651\\
 76 &   0.755 &  0.160 & 166 &   0.186 & -0.658 & 256 &  -0.752 & -0.168 & 346 &  -0.189 &  0.657\\
 78 &   0.761 &  0.137 & 168 &   0.160 & -0.663 & 258 &  -0.758 & -0.145 & 348 &  -0.163 &  0.663\\
 80 &   0.766 &  0.113 & 170 &   0.134 & -0.668 & 260 &  -0.764 & -0.122 & 350 &  -0.136 &  0.667\\
 82 &   0.770 &  0.090 & 172 &   0.107 & -0.671 & 262 &  -0.768 & -0.099 & 352 &  -0.109 &  0.671\\
 84 &   0.773 &  0.066 & 174 &   0.080 & -0.674 & 264 &  -0.772 & -0.075 & 354 &  -0.082 &  0.674\\
 86 &   0.775 &  0.043 & 176 &   0.054 & -0.676 & 266 &  -0.774 & -0.052 & 356 &  -0.055 &  0.676\\
 88 &   0.776 &  0.019 & 178 &   0.027 & -0.677 & 268 &  -0.776 & -0.028 & 358 &  -0.027 &  0.677\\
 90 &   0.777 & -0.005 & 180 &   0.000 & -0.678 & 270 &  -0.777 & -0.005 & 360 &  -0.000 &  0.678\\
\end{tabular}}
\caption{\em Deferential anomalies of Venus.}\label{vt18}
\end{table}

\newpage
\begin{table}\centering
\small{ \begin{tabular}{rrrr|rrrr|rrrr|rrrr}
$\mu$ & $\delta\theta_-$  & $\bar{\theta}~~~~$ & $\delta\theta_+$ &
$\mu$ & $\delta\theta_-$  & $\bar{\theta}~~~~$ & $\delta\theta_+$ &
$\mu$ & $\delta\theta_-$  & $\bar{\theta}~~~~$ & $\delta\theta_+$ &
$\mu$ & $\delta\theta_-$  & $\bar{\theta}~~~~$ & $\delta\theta_+$ \\\hline
&&&&&&&&&&&&&&&\\[-1.75ex]
  0 & \tiny{  0.000} &   0.000 & \tiny{  0.000} &  45 & \tiny{  0.267} &  18.694 & \tiny{  0.274} &  90 & \tiny{  0.629} &  35.875 & \tiny{  0.649} & 135 & \tiny{  1.344} &  46.305 & \tiny{  1.408}\\
  1 & \tiny{  0.006} &   0.420 & \tiny{  0.006} &  46 & \tiny{  0.273} &  19.100 & \tiny{  0.281} &  91 & \tiny{  0.640} &  36.217 & \tiny{  0.660} & 136 & \tiny{  1.369} &  46.320 & \tiny{  1.434}\\
  2 & \tiny{  0.011} &   0.839 & \tiny{  0.012} &  47 & \tiny{  0.280} &  19.505 & \tiny{  0.288} &  92 & \tiny{  0.650} &  36.557 & \tiny{  0.671} & 137 & \tiny{  1.393} &  46.317 & \tiny{  1.461}\\
  3 & \tiny{  0.017} &   1.259 & \tiny{  0.017} &  48 & \tiny{  0.286} &  19.910 & \tiny{  0.294} &  93 & \tiny{  0.661} &  36.893 & \tiny{  0.682} & 138 & \tiny{  1.418} &  46.294 & \tiny{  1.489}\\
  4 & \tiny{  0.023} &   1.679 & \tiny{  0.023} &  49 & \tiny{  0.293} &  20.314 & \tiny{  0.301} &  94 & \tiny{  0.672} &  37.227 & \tiny{  0.693} & 139 & \tiny{  1.444} &  46.252 & \tiny{  1.517}\\
  5 & \tiny{  0.028} &   2.098 & \tiny{  0.029} &  50 & \tiny{  0.300} &  20.717 & \tiny{  0.308} &  95 & \tiny{  0.683} &  37.558 & \tiny{  0.705} & 140 & \tiny{  1.470} &  46.188 & \tiny{  1.546}\\
  6 & \tiny{  0.034} &   2.518 & \tiny{  0.035} &  51 & \tiny{  0.307} &  21.119 & \tiny{  0.315} &  96 & \tiny{  0.694} &  37.886 & \tiny{  0.717} & 141 & \tiny{  1.496} &  46.102 & \tiny{  1.575}\\
  7 & \tiny{  0.040} &   2.937 & \tiny{  0.041} &  52 & \tiny{  0.313} &  21.521 & \tiny{  0.322} &  97 & \tiny{  0.706} &  38.210 & \tiny{  0.729} & 142 & \tiny{  1.523} &  45.992 & \tiny{  1.605}\\
  8 & \tiny{  0.045} &   3.357 & \tiny{  0.046} &  53 & \tiny{  0.320} &  21.921 & \tiny{  0.329} &  98 & \tiny{  0.718} &  38.531 & \tiny{  0.741} & 143 & \tiny{  1.550} &  45.857 & \tiny{  1.636}\\
  9 & \tiny{  0.051} &   3.776 & \tiny{  0.052} &  54 & \tiny{  0.327} &  22.321 & \tiny{  0.336} &  99 & \tiny{  0.729} &  38.849 & \tiny{  0.753} & 144 & \tiny{  1.577} &  45.695 & \tiny{  1.667}\\
 10 & \tiny{  0.057} &   4.195 & \tiny{  0.058} &  55 & \tiny{  0.334} &  22.720 & \tiny{  0.344} & 100 & \tiny{  0.742} &  39.164 & \tiny{  0.766} & 145 & \tiny{  1.605} &  45.505 & \tiny{  1.698}\\
 11 & \tiny{  0.062} &   4.614 & \tiny{  0.064} &  56 & \tiny{  0.341} &  23.119 & \tiny{  0.351} & 101 & \tiny{  0.754} &  39.474 & \tiny{  0.779} & 146 & \tiny{  1.632} &  45.284 & \tiny{  1.730}\\
 12 & \tiny{  0.068} &   5.033 & \tiny{  0.070} &  57 & \tiny{  0.348} &  23.516 & \tiny{  0.358} & 102 & \tiny{  0.766} &  39.781 & \tiny{  0.792} & 147 & \tiny{  1.660} &  45.032 & \tiny{  1.762}\\
 13 & \tiny{  0.074} &   5.452 & \tiny{  0.076} &  58 & \tiny{  0.355} &  23.912 & \tiny{  0.366} & 103 & \tiny{  0.779} &  40.084 & \tiny{  0.805} & 148 & \tiny{  1.688} &  44.745 & \tiny{  1.794}\\
 14 & \tiny{  0.079} &   5.870 & \tiny{  0.082} &  59 & \tiny{  0.363} &  24.308 & \tiny{  0.373} & 104 & \tiny{  0.792} &  40.383 & \tiny{  0.818} & 149 & \tiny{  1.715} &  44.422 & \tiny{  1.827}\\
 15 & \tiny{  0.085} &   6.289 & \tiny{  0.087} &  60 & \tiny{  0.370} &  24.702 & \tiny{  0.381} & 105 & \tiny{  0.805} &  40.677 & \tiny{  0.832} & 150 & \tiny{  1.742} &  44.060 & \tiny{  1.859}\\
 16 & \tiny{  0.091} &   6.707 & \tiny{  0.093} &  61 & \tiny{  0.377} &  25.095 & \tiny{  0.388} & 106 & \tiny{  0.818} &  40.968 & \tiny{  0.846} & 151 & \tiny{  1.769} &  43.657 & \tiny{  1.892}\\
 17 & \tiny{  0.097} &   7.125 & \tiny{  0.099} &  62 & \tiny{  0.385} &  25.487 & \tiny{  0.396} & 107 & \tiny{  0.832} &  41.253 & \tiny{  0.860} & 152 & \tiny{  1.795} &  43.210 & \tiny{  1.924}\\
 18 & \tiny{  0.102} &   7.543 & \tiny{  0.105} &  63 & \tiny{  0.392} &  25.879 & \tiny{  0.404} & 108 & \tiny{  0.846} &  41.534 & \tiny{  0.875} & 153 & \tiny{  1.820} &  42.716 & \tiny{  1.955}\\
 19 & \tiny{  0.108} &   7.960 & \tiny{  0.111} &  64 & \tiny{  0.400} &  26.269 & \tiny{  0.411} & 109 & \tiny{  0.860} &  41.810 & \tiny{  0.890} & 154 & \tiny{  1.844} &  42.173 & \tiny{  1.986}\\
 20 & \tiny{  0.114} &   8.378 & \tiny{  0.117} &  65 & \tiny{  0.407} &  26.658 & \tiny{  0.419} & 110 & \tiny{  0.874} &  42.081 & \tiny{  0.905} & 155 & \tiny{  1.867} &  41.577 & \tiny{  2.015}\\
 21 & \tiny{  0.120} &   8.795 & \tiny{  0.123} &  66 & \tiny{  0.415} &  27.045 & \tiny{  0.427} & 111 & \tiny{  0.889} &  42.346 & \tiny{  0.920} & 156 & \tiny{  1.888} &  40.923 & \tiny{  2.043}\\
 22 & \tiny{  0.126} &   9.212 & \tiny{  0.129} &  67 & \tiny{  0.423} &  27.432 & \tiny{  0.435} & 112 & \tiny{  0.904} &  42.606 & \tiny{  0.936} & 157 & \tiny{  1.906} &  40.210 & \tiny{  2.069}\\
 23 & \tiny{  0.131} &   9.628 & \tiny{  0.135} &  68 & \tiny{  0.431} &  27.817 & \tiny{  0.443} & 113 & \tiny{  0.919} &  42.860 & \tiny{  0.952} & 158 & \tiny{  1.921} &  39.433 & \tiny{  2.092}\\
 24 & \tiny{  0.137} &  10.045 & \tiny{  0.141} &  69 & \tiny{  0.439} &  28.201 & \tiny{  0.452} & 114 & \tiny{  0.934} &  43.108 & \tiny{  0.968} & 159 & \tiny{  1.933} &  38.587 & \tiny{  2.112}\\
 25 & \tiny{  0.143} &  10.461 & \tiny{  0.147} &  70 & \tiny{  0.447} &  28.583 & \tiny{  0.460} & 115 & \tiny{  0.950} &  43.349 & \tiny{  0.985} & 160 & \tiny{  1.942} &  37.669 & \tiny{  2.128}\\
 26 & \tiny{  0.149} &  10.876 & \tiny{  0.153} &  71 & \tiny{  0.455} &  28.964 & \tiny{  0.468} & 116 & \tiny{  0.966} &  43.585 & \tiny{  1.002} & 161 & \tiny{  1.945} &  36.675 & \tiny{  2.140}\\
 27 & \tiny{  0.155} &  11.292 & \tiny{  0.159} &  72 & \tiny{  0.463} &  29.344 & \tiny{  0.477} & 117 & \tiny{  0.983} &  43.813 & \tiny{  1.019} & 162 & \tiny{  1.943} &  35.599 & \tiny{  2.146}\\
 28 & \tiny{  0.161} &  11.707 & \tiny{  0.165} &  73 & \tiny{  0.471} &  29.722 & \tiny{  0.485} & 118 & \tiny{  1.000} &  44.034 & \tiny{  1.037} & 163 & \tiny{  1.934} &  34.437 & \tiny{  2.145}\\
 29 & \tiny{  0.167} &  12.121 & \tiny{  0.172} &  74 & \tiny{  0.480} &  30.099 & \tiny{  0.494} & 119 & \tiny{  1.017} &  44.248 & \tiny{  1.055} & 164 & \tiny{  1.918} &  33.186 & \tiny{  2.137}\\
 30 & \tiny{  0.173} &  12.536 & \tiny{  0.178} &  75 & \tiny{  0.488} &  30.474 & \tiny{  0.503} & 120 & \tiny{  1.034} &  44.453 & \tiny{  1.074} & 165 & \tiny{  1.893} &  31.840 & \tiny{  2.119}\\
 31 & \tiny{  0.179} &  12.950 & \tiny{  0.184} &  76 & \tiny{  0.497} &  30.847 & \tiny{  0.512} & 121 & \tiny{  1.052} &  44.651 & \tiny{  1.093} & 166 & \tiny{  1.859} &  30.396 & \tiny{  2.091}\\
 32 & \tiny{  0.185} &  13.363 & \tiny{  0.190} &  77 & \tiny{  0.506} &  31.219 & \tiny{  0.521} & 122 & \tiny{  1.070} &  44.840 & \tiny{  1.112} & 167 & \tiny{  1.814} &  28.850 & \tiny{  2.050}\\
 33 & \tiny{  0.191} &  13.776 & \tiny{  0.196} &  78 & \tiny{  0.514} &  31.589 & \tiny{  0.530} & 123 & \tiny{  1.089} &  45.021 & \tiny{  1.132} & 168 & \tiny{  1.758} &  27.200 & \tiny{  1.996}\\
 34 & \tiny{  0.197} &  14.189 & \tiny{  0.203} &  79 & \tiny{  0.523} &  31.958 & \tiny{  0.539} & 124 & \tiny{  1.108} &  45.191 & \tiny{  1.152} & 169 & \tiny{  1.688} &  25.442 & \tiny{  1.926}\\
 35 & \tiny{  0.203} &  14.601 & \tiny{  0.209} &  80 & \tiny{  0.532} &  32.324 & \tiny{  0.548} & 125 & \tiny{  1.127} &  45.353 & \tiny{  1.173} & 170 & \tiny{  1.604} &  23.577 & \tiny{  1.840}\\
 36 & \tiny{  0.209} &  15.013 & \tiny{  0.215} &  81 & \tiny{  0.541} &  32.689 & \tiny{  0.558} & 126 & \tiny{  1.147} &  45.503 & \tiny{  1.194} & 171 & \tiny{  1.505} &  21.604 & \tiny{  1.735}\\
 37 & \tiny{  0.216} &  15.424 & \tiny{  0.222} &  82 & \tiny{  0.551} &  33.052 & \tiny{  0.567} & 127 & \tiny{  1.167} &  45.644 & \tiny{  1.216} & 172 & \tiny{  1.391} &  19.526 & \tiny{  1.612}\\
 38 & \tiny{  0.222} &  15.834 & \tiny{  0.228} &  83 & \tiny{  0.560} &  33.412 & \tiny{  0.577} & 128 & \tiny{  1.188} &  45.772 & \tiny{  1.238} & 173 & \tiny{  1.261} &  17.347 & \tiny{  1.468}\\
 39 & \tiny{  0.228} &  16.245 & \tiny{  0.235} &  84 & \tiny{  0.569} &  33.771 & \tiny{  0.587} & 129 & \tiny{  1.209} &  45.889 & \tiny{  1.260} & 174 & \tiny{  1.116} &  15.071 & \tiny{  1.305}\\
 40 & \tiny{  0.234} &  16.654 & \tiny{  0.241} &  85 & \tiny{  0.579} &  34.127 & \tiny{  0.597} & 130 & \tiny{  1.230} &  45.994 & \tiny{  1.284} & 175 & \tiny{  0.956} &  12.707 & \tiny{  1.123}\\
 41 & \tiny{  0.241} &  17.063 & \tiny{  0.248} &  86 & \tiny{  0.589} &  34.482 & \tiny{  0.607} & 131 & \tiny{  1.252} &  46.085 & \tiny{  1.307} & 176 & \tiny{  0.783} &  10.266 & \tiny{  0.922}\\
 42 & \tiny{  0.247} &  17.472 & \tiny{  0.254} &  87 & \tiny{  0.599} &  34.834 & \tiny{  0.617} & 132 & \tiny{  1.275} &  46.163 & \tiny{  1.331} & 177 & \tiny{  0.598} &   7.760 & \tiny{  0.707}\\
 43 & \tiny{  0.254} &  17.880 & \tiny{  0.261} &  88 & \tiny{  0.609} &  35.183 & \tiny{  0.628} & 133 & \tiny{  1.297} &  46.226 & \tiny{  1.356} & 178 & \tiny{  0.404} &   5.202 & \tiny{  0.478}\\
 44 & \tiny{  0.260} &  18.287 & \tiny{  0.267} &  89 & \tiny{  0.619} &  35.530 & \tiny{  0.638} & 134 & \tiny{  1.321} &  46.274 & \tiny{  1.382} & 179 & \tiny{  0.204} &   2.610 & \tiny{  0.242}\\
 45 & \tiny{  0.267} &  18.694 & \tiny{  0.274} &  90 & \tiny{  0.629} &  35.875 & \tiny{  0.649} & 135 & \tiny{  1.344} &  46.305 & \tiny{  1.408} & 180 & \tiny{  0.000} &   0.000 & \tiny{  0.000}\\
\end{tabular}}
\caption[\em Epicyclic anomalies of Venus. ]{\em Epicyclic anomalies of Venus. All quantities are in degrees. Note that $\bar{\theta}(360^\circ-\mu) = -\bar{\theta}(\mu)$, and $\delta\theta_{\pm}(360^\circ-\mu) = -\delta\theta_{\pm}(\mu)$. }\label{vt19}
\end{table}

\newpage
\begin{table}\centering
{\small\begin{tabular}{lcll}
Event & Date & $\lambda$& Elongation \\\hline
&&&\\[-1.75ex]
Superior Conjunction & 11/06/2000 & 20GE46&\\
Greatest Elongation & 17/01/2001 & 14PI23 & $47.1^\circ$ E\\
Inferior Conjunction & 30/03/2001 & 09AR36&\\
Greatest Elongation & 08/06/2001 & 01TA39 & $45.8^\circ$ W\\
Superior Conjunction & 14/01/2002 & 23CP59&\\
Greatest Elongation & 22/08/2002 & 15LI08 & $46.0^\circ$ E\\
Inferior Conjunction & 31/10/2002 & 07SC58&\\
Greatest Elongation & 11/01/2003 & 03SG31 & $47.0^\circ$ W\\
Superior Conjunction & 18/08/2003 & 25LE20&\\
Greatest Elongation & 29/03/2004 & 25TA04 & $46.0^\circ$ E\\
Inferior Conjunction & 08/06/2004 & 17GE52&\\
Greatest Elongation & 17/08/2004 & 09CN29 & $45.8^\circ$ W\\
Superior Conjunction & 31/03/2005 & 10AR33&\\
Greatest Elongation & 03/11/2005 & 28SG27 & $47.1^\circ$ E\\
Inferior Conjunction & 13/01/2006 & 23CP36&\\
Greatest Elongation & 25/03/2006 & 18AQ07 & $46.5^\circ$ W\\
Superior Conjunction & 27/10/2006 & 04SC13&\\
Greatest Elongation & 09/06/2007 & 03LE20 & $45.4^\circ$ E\\
Inferior Conjunction & 18/08/2007 & 24LE45&\\
Greatest Elongation & 28/10/2007 & 18VI19 & $46.5^\circ$ W\\
Superior Conjunction & 09/06/2008 & 18GE41&\\
Greatest Elongation & 14/01/2009 & 12PI03 & $47.1^\circ$ E\\
Inferior Conjunction & 27/03/2009 & 07AR19&\\
Greatest Elongation & 05/06/2009 & 29AR25 & $45.8^\circ$ W\\
Superior Conjunction & 11/01/2010 & 21CP24&\\
Greatest Elongation & 19/08/2010 & 12LI49 & $46.0^\circ$ E\\
Inferior Conjunction & 29/10/2010 & 05SC34&\\
Greatest Elongation & 08/01/2011 & 01SG06 & $47.0^\circ$ W\\
Superior Conjunction & 16/08/2011 & 23LE13&\\
Greatest Elongation & 27/03/2012 & 22TA50 & $46.0^\circ$ E\\
Inferior Conjunction & 06/06/2012 & 15GE43&\\
Greatest Elongation & 15/08/2012 & 07CN18 & $45.8^\circ$ W\\
Superior Conjunction & 28/03/2013 & 08AR12&\\
Greatest Elongation & 01/11/2013 & 26SG02 & $47.1^\circ$ E\\
Inferior Conjunction & 11/01/2014 & 21CP08&\\
Greatest Elongation & 23/03/2014 & 15AQ44 & $46.5^\circ$ W\\
Superior Conjunction & 25/10/2014 & 01SC51&\\
Greatest Elongation & 06/06/2015 & 01LE09 & $45.4^\circ$ E\\
Inferior Conjunction & 15/08/2015 & 22LE33&\\
Greatest Elongation & 26/10/2015 & 16VI02 & $46.4^\circ$ W\\
\end{tabular}}
\caption{\em The conjunctions and greatest elongations of Venus
during the years 2000--2015 CE.}\label{vtvenus}
\end{table}

\clearpage
\newpage
\begin{table}
\centering
\begin{tabular}{rrrr|rrrr}
$\Delta t$(JD)& $\Delta\bar{\lambda}(^\circ)$ &  $\Delta M(^\circ)$ & $\Delta \bar{F}(^\circ)$& $\Delta t$(JD) & $\Delta\bar{\lambda}(^\circ)$ & $\Delta M(^\circ)$ 
&$\Delta \bar{F}(^\circ)$\\ \hline
&&&&&&&\\[-1.75ex]
10,000 & 243.770 & 243.344 & 243.422 & 1,000 & 132.377 & 132.334 & 132.342\\
20,000 & 127.541 & 126.688 & 126.844 & 2,000 & 264.754 & 264.669 & 264.684\\
30,000 &  11.311 &  10.032 &  10.266 & 3,000 &  37.131 &  37.003 &  37.027\\
40,000 & 255.081 & 253.376 & 253.688 & 4,000 & 169.508 & 169.338 & 169.369\\
50,000 & 138.852 & 136.720 & 137.110 & 5,000 & 301.885 & 301.672 & 301.711\\
60,000 &  22.622 &  20.063 &  20.533 & 6,000 &  74.262 &  74.006 &  74.053\\
70,000 & 266.392 & 263.407 & 263.955 & 7,000 & 206.639 & 206.341 & 206.395\\
80,000 & 150.162 & 146.751 & 147.377 & 8,000 & 339.016 & 338.675 & 338.738\\
90,000 &  33.933 &  30.095 &  30.799 & 9,000 & 111.393 & 111.010 & 111.080\\
&&&&&&&\\
100 &  49.238 &  49.233 &  49.234 & 10 &  40.924 &  40.923 &  40.923\\
200 &  98.475 &  98.467 &  98.468 & 20 &  81.848 &  81.847 &  81.847\\
300 & 147.713 & 147.700 & 147.703 & 30 & 122.771 & 122.770 & 122.770\\
400 & 196.951 & 196.934 & 196.937 & 40 & 163.695 & 163.693 & 163.694\\
500 & 246.189 & 246.167 & 246.171 & 50 & 204.619 & 204.617 & 204.617\\
600 & 295.426 & 295.401 & 295.405 & 60 & 245.543 & 245.540 & 245.541\\
700 & 344.664 & 344.634 & 344.640 & 70 & 286.466 & 286.463 & 286.464\\
800 &  33.902 &  33.868 &  33.874 & 80 & 327.390 & 327.387 & 327.387\\
900 &  83.139 &  83.101 &  83.108 & 90 &   8.314 &   8.310 &   8.311\\
&&&&&&&\\
1 &   4.092 &   4.092 &   4.092 & 0.1 &   0.409 &   0.409 &   0.409\\
2 &   8.185 &   8.185 &   8.185 & 0.2 &   0.818 &   0.818 &   0.818\\
3 &  12.277 &  12.277 &  12.277 & 0.3 &   1.228 &   1.228 &   1.228\\
4 &  16.370 &  16.369 &  16.369 & 0.4 &   1.637 &   1.637 &   1.637\\
5 &  20.462 &  20.462 &  20.462 & 0.5 &   2.046 &   2.046 &   2.046\\
6 &  24.554 &  24.554 &  24.554 & 0.6 &   2.455 &   2.455 &   2.455\\
7 &  28.647 &  28.646 &  28.646 & 0.7 &   2.865 &   2.865 &   2.865\\
8 &  32.739 &  32.739 &  32.739 & 0.8 &   3.274 &   3.274 &   3.274\\
9 &  36.831 &  36.831 &  36.831 & 0.9 &   3.683 &   3.683 &   3.683\\
\end{tabular}
\caption[\em Mean motion of Mercury.]{\em Mean motion of Mercury.  Here, $\Delta t = t-t_0$, $\Delta\bar{\lambda} = \bar{\lambda}-\bar{\lambda}_0$, $\Delta M = M - M_0$, and $\Delta \bar{F} = \bar{F} - \bar{F}_0$. At epoch  ($t_0 = 2\,451\,545.0$ JD), $\bar{\lambda}_0 = 252.087^\circ$,  $M_0 = 174.693^\circ$,
and $\bar{F}_0 = 204.436^\circ$. }\label{vt20}
\end{table}

\newpage
\begin{table}\centering
\small{ \begin{tabular}{rrr|rrr|rrr|rrr}
$M(^\circ)$ & $q(^\circ)$  & $100\,\zeta$ & $M(^\circ)$ & $q(^\circ)$  & $100\,\zeta$ & $M(^\circ)$ & $q(^\circ)$  & $100\,\zeta$& $M(^\circ)$ & $q(^\circ)$  & $100\,\zeta$\\\hline
&&&&&&&&&&&\\[-1.75ex]
  0 &   0.000 & 20.564 &  90 &  22.900 & -4.229 & 180 &   0.000 & -20.564 & 270 & -22.900 & -4.229\\
  2 &   1.086 & 20.544 &  92 &  22.677 & -4.896 & 182 &  -0.663 & -20.555 & 272 & -23.100 & -3.551\\
  4 &   2.169 & 20.487 &  94 &  22.433 & -5.552 & 184 &  -1.326 & -20.528 & 274 & -23.276 & -2.864\\
  6 &   3.247 & 20.391 &  96 &  22.168 & -6.197 & 186 &  -1.987 & -20.483 & 276 & -23.428 & -2.168\\
  8 &   4.316 & 20.257 &  98 &  21.884 & -6.831 & 188 &  -2.647 & -20.420 & 278 & -23.553 & -1.463\\
 10 &   5.376 & 20.085 & 100 &  21.580 & -7.452 & 190 &  -3.304 & -20.340 & 280 & -23.652 & -0.750\\
 12 &   6.422 & 19.876 & 102 &  21.259 & -8.062 & 192 &  -3.959 & -20.242 & 282 & -23.723 & -0.030\\
 14 &   7.454 & 19.631 & 104 &  20.920 & -8.659 & 194 &  -4.610 & -20.126 & 284 & -23.764 &  0.697\\
 16 &   8.467 & 19.350 & 106 &  20.566 & -9.243 & 196 &  -5.257 & -19.993 & 286 & -23.775 &  1.429\\
 18 &   9.460 & 19.035 & 108 &  20.195 & -9.815 & 198 &  -5.900 & -19.842 & 288 & -23.755 &  2.165\\
 20 &  10.431 & 18.685 & 110 &  19.809 & -10.373 & 200 &  -6.538 & -19.675 & 290 & -23.703 &  2.905\\
 22 &  11.377 & 18.303 & 112 &  19.409 & -10.918 & 202 &  -7.170 & -19.490 & 292 & -23.617 &  3.648\\
 24 &  12.298 & 17.889 & 114 &  18.996 & -11.450 & 204 &  -7.796 & -19.288 & 294 & -23.497 &  4.392\\
 26 &  13.190 & 17.445 & 116 &  18.569 & -11.969 & 206 &  -8.417 & -19.070 & 296 & -23.342 &  5.137\\
 28 &  14.052 & 16.971 & 118 &  18.129 & -12.473 & 208 &  -9.030 & -18.835 & 298 & -23.150 &  5.880\\
 30 &  14.882 & 16.469 & 120 &  17.677 & -12.964 & 210 &  -9.637 & -18.583 & 300 & -22.922 &  6.621\\
 32 &  15.680 & 15.941 & 122 &  17.212 & -13.441 & 212 & -10.236 & -18.316 & 302 & -22.656 &  7.359\\
 34 &  16.443 & 15.388 & 124 &  16.737 & -13.904 & 214 & -10.827 & -18.032 & 304 & -22.353 &  8.091\\
 36 &  17.171 & 14.811 & 126 &  16.250 & -14.353 & 216 & -11.410 & -17.733 & 306 & -22.010 &  8.818\\
 38 &  17.862 & 14.212 & 128 &  15.752 & -14.787 & 218 & -11.985 & -17.418 & 308 & -21.629 &  9.536\\
 40 &  18.517 & 13.593 & 130 &  15.243 & -15.207 & 220 & -12.552 & -17.087 & 310 & -21.208 & 10.245\\
 42 &  19.133 & 12.954 & 132 &  14.724 & -15.613 & 222 & -13.109 & -16.741 & 312 & -20.748 & 10.942\\
 44 &  19.710 & 12.299 & 134 &  14.196 & -16.004 & 224 & -13.657 & -16.380 & 314 & -20.249 & 11.628\\
 46 &  20.249 & 11.628 & 136 &  13.657 & -16.380 & 226 & -14.196 & -16.004 & 316 & -19.710 & 12.299\\
 48 &  20.748 & 10.942 & 138 &  13.109 & -16.741 & 228 & -14.724 & -15.613 & 318 & -19.133 & 12.954\\
 50 &  21.208 & 10.245 & 140 &  12.552 & -17.087 & 230 & -15.243 & -15.207 & 320 & -18.517 & 13.593\\
 52 &  21.629 &  9.536 & 142 &  11.985 & -17.418 & 232 & -15.752 & -14.787 & 322 & -17.862 & 14.212\\
 54 &  22.010 &  8.818 & 144 &  11.410 & -17.733 & 234 & -16.250 & -14.353 & 324 & -17.171 & 14.811\\
 56 &  22.353 &  8.091 & 146 &  10.827 & -18.032 & 236 & -16.737 & -13.904 & 326 & -16.443 & 15.388\\
 58 &  22.656 &  7.359 & 148 &  10.236 & -18.316 & 238 & -17.212 & -13.441 & 328 & -15.680 & 15.941\\
 60 &  22.922 &  6.621 & 150 &   9.637 & -18.583 & 240 & -17.677 & -12.964 & 330 & -14.882 & 16.469\\
 62 &  23.150 &  5.880 & 152 &   9.030 & -18.835 & 242 & -18.129 & -12.473 & 332 & -14.052 & 16.971\\
 64 &  23.342 &  5.137 & 154 &   8.417 & -19.070 & 244 & -18.569 & -11.969 & 334 & -13.190 & 17.445\\
 66 &  23.497 &  4.392 & 156 &   7.796 & -19.288 & 246 & -18.996 & -11.450 & 336 & -12.298 & 17.889\\
 68 &  23.617 &  3.648 & 158 &   7.170 & -19.490 & 248 & -19.409 & -10.918 & 338 & -11.377 & 18.303\\
 70 &  23.703 &  2.905 & 160 &   6.538 & -19.675 & 250 & -19.809 & -10.373 & 340 & -10.431 & 18.685\\
 72 &  23.755 &  2.165 & 162 &   5.900 & -19.842 & 252 & -20.195 & -9.815 & 342 &  -9.460 & 19.035\\
 74 &  23.775 &  1.429 & 164 &   5.257 & -19.993 & 254 & -20.566 & -9.243 & 344 &  -8.467 & 19.350\\
 76 &  23.764 &  0.697 & 166 &   4.610 & -20.126 & 256 & -20.920 & -8.659 & 346 &  -7.454 & 19.631\\
 78 &  23.723 & -0.030 & 168 &   3.959 & -20.242 & 258 & -21.259 & -8.062 & 348 &  -6.422 & 19.876\\
 80 &  23.652 & -0.750 & 170 &   3.304 & -20.340 & 260 & -21.580 & -7.452 & 350 &  -5.376 & 20.085\\
 82 &  23.553 & -1.463 & 172 &   2.647 & -20.420 & 262 & -21.884 & -6.831 & 352 &  -4.316 & 20.257\\
 84 &  23.428 & -2.168 & 174 &   1.987 & -20.483 & 264 & -22.168 & -6.197 & 354 &  -3.247 & 20.391\\
 86 &  23.276 & -2.864 & 176 &   1.326 & -20.528 & 266 & -22.433 & -5.552 & 356 &  -2.169 & 20.487\\
 88 &  23.100 & -3.551 & 178 &   0.663 & -20.555 & 268 & -22.677 & -4.896 & 358 &  -1.086 & 20.544\\
 90 &  22.900 & -4.229 & 180 &   0.000 & -20.564 & 270 & -22.900 & -4.229 & 360 &  -0.000 & 20.564\\
\end{tabular}}
\caption{\em Deferential anomalies of Mercury.}\label{vt21}
\end{table}

\newpage
\begin{table}\centering
\small{ \begin{tabular}{rrrr|rrrr|rrrr|rrrr}
$\mu$ & $\delta\theta_-$  & $\bar{\theta}~~~~$ & $\delta\theta_+$ &
$\mu$ & $\delta\theta_-$  & $\bar{\theta}~~~~$ & $\delta\theta_+$ &
$\mu$ & $\delta\theta_-$  & $\bar{\theta}~~~~$ & $\delta\theta_+$ &
$\mu$ & $\delta\theta_-$  & $\bar{\theta}~~~~$ & $\delta\theta_+$ \\\hline
&&&&&&&&&&&&&&&\\[-1.75ex]
  0 & \tiny{  0.000} &   0.000 & \tiny{  0.000} &  45 & \tiny{  1.711} &  11.702 & \tiny{  2.403} &  90 & \tiny{  3.450} &  20.277 & \tiny{  5.113} & 135 & \tiny{  4.257} &  19.475 & \tiny{  7.325}\\
  1 & \tiny{  0.038} &   0.270 & \tiny{  0.052} &  46 & \tiny{  1.749} &  11.941 & \tiny{  2.459} &  91 & \tiny{  3.486} &  20.395 & \tiny{  5.177} & 136 & \tiny{  4.237} &  19.267 & \tiny{  7.327}\\
  2 & \tiny{  0.075} &   0.540 & \tiny{  0.104} &  47 & \tiny{  1.788} &  12.179 & \tiny{  2.515} &  92 & \tiny{  3.522} &  20.509 & \tiny{  5.241} & 137 & \tiny{  4.214} &  19.048 & \tiny{  7.324}\\
  3 & \tiny{  0.113} &   0.809 & \tiny{  0.156} &  48 & \tiny{  1.827} &  12.415 & \tiny{  2.572} &  93 & \tiny{  3.557} &  20.618 & \tiny{  5.305} & 138 & \tiny{  4.188} &  18.818 & \tiny{  7.317}\\
  4 & \tiny{  0.150} &   1.079 & \tiny{  0.208} &  49 & \tiny{  1.866} &  12.650 & \tiny{  2.628} &  94 & \tiny{  3.592} &  20.722 & \tiny{  5.368} & 139 & \tiny{  4.159} &  18.578 & \tiny{  7.304}\\
  5 & \tiny{  0.188} &   1.348 & \tiny{  0.260} &  50 & \tiny{  1.905} &  12.882 & \tiny{  2.685} &  95 & \tiny{  3.627} &  20.822 & \tiny{  5.432} & 140 & \tiny{  4.127} &  18.326 & \tiny{  7.286}\\
  6 & \tiny{  0.225} &   1.618 & \tiny{  0.313} &  51 & \tiny{  1.944} &  13.114 & \tiny{  2.742} &  96 & \tiny{  3.662} &  20.917 & \tiny{  5.496} & 141 & \tiny{  4.092} &  18.064 & \tiny{  7.262}\\
  7 & \tiny{  0.263} &   1.887 & \tiny{  0.365} &  52 & \tiny{  1.983} &  13.343 & \tiny{  2.799} &  97 & \tiny{  3.696} &  21.007 & \tiny{  5.559} & 142 & \tiny{  4.053} &  17.791 & \tiny{  7.233}\\
  8 & \tiny{  0.301} &   2.156 & \tiny{  0.417} &  53 & \tiny{  2.022} &  13.571 & \tiny{  2.857} &  98 & \tiny{  3.729} &  21.091 & \tiny{  5.623} & 143 & \tiny{  4.011} &  17.506 & \tiny{  7.197}\\
  9 & \tiny{  0.338} &   2.425 & \tiny{  0.469} &  54 & \tiny{  2.061} &  13.797 & \tiny{  2.914} &  99 & \tiny{  3.762} &  21.171 & \tiny{  5.686} & 144 & \tiny{  3.966} &  17.210 & \tiny{  7.155}\\
 10 & \tiny{  0.376} &   2.693 & \tiny{  0.521} &  55 & \tiny{  2.100} &  14.021 & \tiny{  2.972} & 100 & \tiny{  3.795} &  21.246 & \tiny{  5.749} & 145 & \tiny{  3.917} &  16.903 & \tiny{  7.106}\\
 11 & \tiny{  0.414} &   2.961 & \tiny{  0.574} &  56 & \tiny{  2.139} &  14.244 & \tiny{  3.030} & 101 & \tiny{  3.827} &  21.315 & \tiny{  5.812} & 146 & \tiny{  3.865} &  16.585 & \tiny{  7.050}\\
 12 & \tiny{  0.451} &   3.229 & \tiny{  0.626} &  57 & \tiny{  2.178} &  14.464 & \tiny{  3.088} & 102 & \tiny{  3.858} &  21.378 & \tiny{  5.874} & 147 & \tiny{  3.809} &  16.255 & \tiny{  6.986}\\
 13 & \tiny{  0.489} &   3.497 & \tiny{  0.678} &  58 & \tiny{  2.217} &  14.683 & \tiny{  3.146} & 103 & \tiny{  3.889} &  21.436 & \tiny{  5.937} & 148 & \tiny{  3.749} &  15.914 & \tiny{  6.915}\\
 14 & \tiny{  0.527} &   3.764 & \tiny{  0.731} &  59 & \tiny{  2.257} &  14.899 & \tiny{  3.205} & 104 & \tiny{  3.919} &  21.488 & \tiny{  5.999} & 149 & \tiny{  3.686} &  15.561 & \tiny{  6.836}\\
 15 & \tiny{  0.564} &   4.031 & \tiny{  0.783} &  60 & \tiny{  2.296} &  15.113 & \tiny{  3.263} & 105 & \tiny{  3.949} &  21.534 & \tiny{  6.060} & 150 & \tiny{  3.619} &  15.197 & \tiny{  6.749}\\
 16 & \tiny{  0.602} &   4.298 & \tiny{  0.836} &  61 & \tiny{  2.335} &  15.326 & \tiny{  3.322} & 106 & \tiny{  3.977} &  21.575 & \tiny{  6.121} & 151 & \tiny{  3.548} &  14.822 & \tiny{  6.654}\\
 17 & \tiny{  0.640} &   4.564 & \tiny{  0.889} &  62 & \tiny{  2.374} &  15.536 & \tiny{  3.381} & 107 & \tiny{  4.005} &  21.609 & \tiny{  6.182} & 152 & \tiny{  3.473} &  14.436 & \tiny{  6.549}\\
 18 & \tiny{  0.678} &   4.829 & \tiny{  0.941} &  63 & \tiny{  2.413} &  15.743 & \tiny{  3.441} & 108 & \tiny{  4.032} &  21.636 & \tiny{  6.242} & 153 & \tiny{  3.394} &  14.039 & \tiny{  6.436}\\
 19 & \tiny{  0.716} &   5.094 & \tiny{  0.994} &  64 & \tiny{  2.453} &  15.949 & \tiny{  3.500} & 109 & \tiny{  4.059} &  21.658 & \tiny{  6.301} & 154 & \tiny{  3.311} &  13.630 & \tiny{  6.314}\\
 20 & \tiny{  0.753} &   5.359 & \tiny{  1.047} &  65 & \tiny{  2.492} &  16.152 & \tiny{  3.560} & 110 & \tiny{  4.084} &  21.673 & \tiny{  6.360} & 155 & \tiny{  3.224} &  13.211 & \tiny{  6.182}\\
 21 & \tiny{  0.791} &   5.622 & \tiny{  1.100} &  66 & \tiny{  2.531} &  16.353 & \tiny{  3.620} & 111 & \tiny{  4.109} &  21.681 & \tiny{  6.418} & 156 & \tiny{  3.133} &  12.780 & \tiny{  6.041}\\
 22 & \tiny{  0.829} &   5.886 & \tiny{  1.153} &  67 & \tiny{  2.570} &  16.551 & \tiny{  3.680} & 112 & \tiny{  4.132} &  21.682 & \tiny{  6.475} & 157 & \tiny{  3.039} &  12.339 & \tiny{  5.890}\\
 23 & \tiny{  0.867} &   6.148 & \tiny{  1.206} &  68 & \tiny{  2.609} &  16.747 & \tiny{  3.740} & 113 & \tiny{  4.155} &  21.676 & \tiny{  6.532} & 158 & \tiny{  2.940} &  11.888 & \tiny{  5.729}\\
 24 & \tiny{  0.905} &   6.410 & \tiny{  1.259} &  69 & \tiny{  2.649} &  16.940 & \tiny{  3.801} & 114 & \tiny{  4.176} &  21.663 & \tiny{  6.587} & 159 & \tiny{  2.838} &  11.426 & \tiny{  5.558}\\
 25 & \tiny{  0.943} &   6.672 & \tiny{  1.312} &  70 & \tiny{  2.688} &  17.131 & \tiny{  3.862} & 115 & \tiny{  4.197} &  21.643 & \tiny{  6.641} & 160 & \tiny{  2.732} &  10.955 & \tiny{  5.377}\\
 26 & \tiny{  0.981} &   6.932 & \tiny{  1.366} &  71 & \tiny{  2.727} &  17.319 & \tiny{  3.923} & 116 & \tiny{  4.216} &  21.616 & \tiny{  6.695} & 161 & \tiny{  2.622} &  10.474 & \tiny{  5.186}\\
 27 & \tiny{  1.019} &   7.192 & \tiny{  1.419} &  72 & \tiny{  2.766} &  17.504 & \tiny{  3.984} & 117 & \tiny{  4.233} &  21.580 & \tiny{  6.747} & 162 & \tiny{  2.508} &   9.983 & \tiny{  4.985}\\
 28 & \tiny{  1.057} &   7.451 & \tiny{  1.473} &  73 & \tiny{  2.805} &  17.686 & \tiny{  4.045} & 118 & \tiny{  4.250} &  21.538 & \tiny{  6.797} & 163 & \tiny{  2.391} &   9.483 & \tiny{  4.774}\\
 29 & \tiny{  1.095} &   7.709 & \tiny{  1.526} &  74 & \tiny{  2.844} &  17.865 & \tiny{  4.107} & 119 & \tiny{  4.265} &  21.487 & \tiny{  6.846} & 164 & \tiny{  2.271} &   8.974 & \tiny{  4.554}\\
 30 & \tiny{  1.133} &   7.967 & \tiny{  1.580} &  75 & \tiny{  2.882} &  18.042 & \tiny{  4.169} & 120 & \tiny{  4.278} &  21.428 & \tiny{  6.894} & 165 & \tiny{  2.147} &   8.457 & \tiny{  4.324}\\
 31 & \tiny{  1.172} &   8.223 & \tiny{  1.634} &  76 & \tiny{  2.921} &  18.215 & \tiny{  4.231} & 121 & \tiny{  4.291} &  21.361 & \tiny{  6.940} & 166 & \tiny{  2.019} &   7.932 & \tiny{  4.084}\\
 32 & \tiny{  1.210} &   8.479 & \tiny{  1.688} &  77 & \tiny{  2.960} &  18.385 & \tiny{  4.293} & 122 & \tiny{  4.301} &  21.286 & \tiny{  6.984} & 167 & \tiny{  1.889} &   7.399 & \tiny{  3.835}\\
 33 & \tiny{  1.248} &   8.734 & \tiny{  1.742} &  78 & \tiny{  2.999} &  18.552 & \tiny{  4.355} & 123 & \tiny{  4.310} &  21.202 & \tiny{  7.026} & 168 & \tiny{  1.756} &   6.859 & \tiny{  3.578}\\
 34 & \tiny{  1.286} &   8.987 & \tiny{  1.796} &  79 & \tiny{  3.037} &  18.716 & \tiny{  4.417} & 124 & \tiny{  4.317} &  21.109 & \tiny{  7.067} & 169 & \tiny{  1.620} &   6.312 & \tiny{  3.312}\\
 35 & \tiny{  1.325} &   9.240 & \tiny{  1.851} &  80 & \tiny{  3.075} &  18.876 & \tiny{  4.480} & 125 & \tiny{  4.322} &  21.008 & \tiny{  7.105} & 170 & \tiny{  1.481} &   5.759 & \tiny{  3.038}\\
 36 & \tiny{  1.363} &   9.491 & \tiny{  1.905} &  81 & \tiny{  3.114} &  19.033 & \tiny{  4.543} & 126 & \tiny{  4.326} &  20.898 & \tiny{  7.140} & 171 & \tiny{  1.340} &   5.200 & \tiny{  2.757}\\
 37 & \tiny{  1.402} &   9.742 & \tiny{  1.960} &  82 & \tiny{  3.152} &  19.186 & \tiny{  4.606} & 127 & \tiny{  4.327} &  20.778 & \tiny{  7.173} & 172 & \tiny{  1.197} &   4.636 & \tiny{  2.469}\\
 38 & \tiny{  1.440} &   9.991 & \tiny{  2.015} &  83 & \tiny{  3.190} &  19.336 & \tiny{  4.669} & 128 & \tiny{  4.326} &  20.649 & \tiny{  7.204} & 173 & \tiny{  1.052} &   4.067 & \tiny{  2.174}\\
 39 & \tiny{  1.479} &  10.240 & \tiny{  2.070} &  84 & \tiny{  3.227} &  19.482 & \tiny{  4.732} & 129 & \tiny{  4.323} &  20.511 & \tiny{  7.231} & 174 & \tiny{  0.905} &   3.494 & \tiny{  1.875}\\
 40 & \tiny{  1.517} &  10.487 & \tiny{  2.125} &  85 & \tiny{  3.265} &  19.625 & \tiny{  4.795} & 130 & \tiny{  4.318} &  20.363 & \tiny{  7.256} & 175 & \tiny{  0.756} &   2.917 & \tiny{  1.570}\\
 41 & \tiny{  1.556} &  10.732 & \tiny{  2.180} &  86 & \tiny{  3.302} &  19.763 & \tiny{  4.859} & 131 & \tiny{  4.311} &  20.206 & \tiny{  7.277} & 176 & \tiny{  0.607} &   2.337 & \tiny{  1.261}\\
 42 & \tiny{  1.594} &  10.977 & \tiny{  2.236} &  87 & \tiny{  3.340} &  19.898 & \tiny{  4.922} & 132 & \tiny{  4.301} &  20.038 & \tiny{  7.295} & 177 & \tiny{  0.456} &   1.755 & \tiny{  0.949}\\
 43 & \tiny{  1.633} &  11.220 & \tiny{  2.291} &  88 & \tiny{  3.377} &  20.029 & \tiny{  4.986} & 133 & \tiny{  4.289} &  19.861 & \tiny{  7.309} & 178 & \tiny{  0.304} &   1.171 & \tiny{  0.634}\\
 44 & \tiny{  1.672} &  11.462 & \tiny{  2.347} &  89 & \tiny{  3.413} &  20.155 & \tiny{  5.049} & 134 & \tiny{  4.274} &  19.673 & \tiny{  7.319} & 179 & \tiny{  0.152} &   0.586 & \tiny{  0.317}\\
 45 & \tiny{  1.711} &  11.702 & \tiny{  2.403} &  90 & \tiny{  3.450} &  20.277 & \tiny{  5.113} & 135 & \tiny{  4.257} &  19.475 & \tiny{  7.325} & 180 & \tiny{  0.000} &   0.000 & \tiny{  0.000}\\
\end{tabular}}
\caption[\em Epicyclic anomalies of Mercury.]{\em Epicyclic anomalies of Mercury. All quantities are in degrees. Note that $\bar{\theta}(360^\circ-\mu) = -\bar{\theta}(\mu)$, and $\delta\theta_{\pm}(360^\circ-\mu) = -\delta\theta_{\pm}(\mu)$. }\label{vt22}
\end{table}


\newpage
\begin{table}\centering
{\small\begin{tabular}{lcll}
Event & Date & $\lambda$& Elongation \\\hline
&&&\\[-1.75ex]
Superior Conjunction & 15/01/2000 & 25CP08&\\
Greatest Elongation & 15/02/2000 & 13PI44 & $18.1^\circ$ E\\
Inferior Conjunction & 01/03/2000 & 11PI23&\\
Greatest Elongation & 28/03/2000 & 10PI35 & $27.9^\circ$ W\\
Superior Conjunction & 09/05/2000 & 18TA59&\\
Greatest Elongation & 09/06/2000 & 13CN27 & $24.2^\circ$ E\\
Inferior Conjunction & 06/07/2000 & 14CN39&\\
Greatest Elongation & 27/07/2000 & 15CN03 & $19.7^\circ$ W\\
Superior Conjunction & 22/08/2000 & 29LE16&\\
Greatest Elongation & 06/10/2000 & 09SC17 & $25.6^\circ$ E\\
Inferior Conjunction & 30/10/2000 & 06SC58&\\
Greatest Elongation & 15/11/2000 & 04SC02 & $19.3^\circ$ W\\
Superior Conjunction & 25/12/2000 & 04CP18&\\
Greatest Elongation & 28/01/2001 & 27AQ07 & $18.4^\circ$ E\\
Inferior Conjunction & 13/02/2001 & 24AQ23&\\
Greatest Elongation & 11/03/2001 & 23AQ05 & $27.5^\circ$ W\\
Superior Conjunction & 23/04/2001 & 03TA22&\\
Greatest Elongation & 22/05/2001 & 24GE13 & $22.6^\circ$ E\\
Inferior Conjunction & 16/06/2001 & 25GE26&\\
Greatest Elongation & 09/07/2001 & 26GE18 & $21.1^\circ$ W\\
Superior Conjunction & 05/08/2001 & 13LE32&\\
Greatest Elongation & 19/09/2001 & 22LI50 & $26.6^\circ$ E\\
Inferior Conjunction & 14/10/2001 & 20LI51&\\
Greatest Elongation & 29/10/2001 & 17LI50 & $18.5^\circ$ W\\
Superior Conjunction & 04/12/2001 & 12SG45&\\
Greatest Elongation & 12/01/2002 & 10AQ33 & $18.9^\circ$ E\\
Inferior Conjunction & 27/01/2002 & 07AQ42&\\
Greatest Elongation & 21/02/2002 & 05AQ55 & $26.6^\circ$ W\\
Superior Conjunction & 07/04/2002 & 17AR27&\\
Greatest Elongation & 04/05/2002 & 04GE54 & $21.0^\circ$ E\\
Inferior Conjunction & 27/05/2002 & 05GE48&\\
Greatest Elongation & 21/06/2002 & 07GE04 & $22.8^\circ$ W\\
Superior Conjunction & 21/07/2002 & 28CN06&\\
Greatest Elongation & 01/09/2002 & 06LI10 & $27.3^\circ$ E\\
Inferior Conjunction & 27/09/2002 & 04LI35&\\
Greatest Elongation & 13/10/2002 & 01LI44 & $18.0^\circ$ W\\
Superior Conjunction & 14/11/2002 & 21SC39&\\
Greatest Elongation & 26/12/2002 & 24CP01 & $19.8^\circ$ E\\
\end{tabular}}
\caption{\em The conjunctions and greatest elongations of Mercury
during the years 2000--2002 CE.}\label{vtmercury}
\end{table}
