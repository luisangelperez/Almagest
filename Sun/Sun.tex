\chapter{The Sun}\label{csun}
\section{Determination of Ecliptic Longitude}\label{ssun}
Our solar longitude model is sketched in Figure~\ref{lf3}. From a geocentric point of view, the sun, $S$,  appears to execute
a (counterclockwise) Keplerian orbit of major radius $a$, and eccentricity $e$, about the
earth, $G$. As has already been mentioned, the circle traced out by the sun on the celestial sphere is
known as the {\em ecliptic}\/ circle. This circle is inclined at $23^\circ 26'$ to
the {\em celestial equator}, which is the projection of the earth's equator onto
the celestial sphere.
 Suppose that the angle subtended at the earth between the vernal equinox ({\em i.e.}, the point at which the ecliptic crosses the celestial equator from
south to north) and the
sun's perigee ({\em i.e.}, the point of closest approach to the earth)   is
 $\varpi$. This
angle is termed the {\em longitude of the perigee}, and
 is assumed
to vary {\em linearly}\/ with time: {\em i.e.}, 
\begin{equation}\label{ae79}
\varpi  = \varpi_0 + \varpi_1\,(t-t_0).
\end{equation}
\begin{figure}[h]
\epsfysize=3in
\centerline{\epsffile{epsfiles/fig3.eps}}
\caption[\em The apparent orbit of the sun about the earth.]{\em The apparent orbit of the sun about the earth.  Here, $S$, $G$, $\Pi$, $A$, $\varpi$, $T$,  $\lambda$, and $\Upsilon$
represent the sun, earth, perigee, apogee, longitude of the perigee, true anomaly, ecliptic longitude, and
vernal equinox, respectively. View is from northern ecliptic pole. The sun orbits counterclockwise.}\label{lf3}
\end{figure}

 The sun's {\em ecliptic
longitude}\/ is defined as the angle subtended at the earth between the vernal equinox and the sun.
Hence, from Fig.~\ref{lf3},
\begin{equation}
\lambda = \varpi + T,
\end{equation}
where $T$ is the true anomaly (see Cha.~\ref{ckep}). By analogy, the  {\em mean longitude}\/ is written
\begin{equation}
\bar{\lambda} = \varpi + M,
\end{equation}
where $M$ is the  mean anomaly (see Cha.~\ref{ckep}). It follows from Eq.~(\ref{je23}) that
\begin{equation}\label{ae82}
\lambda = \bar{\lambda}  + q,
\end{equation}
where
\begin{equation}
q = 2\,e\,\sin M + (5/4)\,e^2\,\sin\,2M,
\end{equation}
 is called  the {\em equation of  center}. Note that $\lambda$, $\bar{\lambda}$, $T$, and $M$ are usually written as angles in the range
 $0^\circ$ to $360^\circ$, whereas $q$ is generally written as an
 angle in the range $-180^\circ$ to $+180^\circ$. 

The mean longitude increases
uniformly with time (since both $\varpi$ and $M$ increase uniformly with time) as
\begin{equation}\label{ae84}
\bar{\lambda} =  \bar{\lambda}_0+ n\,(t -t_0),
\end{equation}
where $\bar{\lambda}_0$ is termed the
{\em mean longitude at epoch}, 
$n$ the {\em rate of motion in mean longitude}, and $t_0$ the {\em epoch}. 
We can also write
\begin{equation}\label{ae85}
M = M_0 + \tilde{n}\,(t-t_0),
\end{equation}
where
\begin{equation}
M_0 = \bar{\lambda}_0 - \varpi_0
\end{equation}
is called the {\em mean anomaly at epoch}, and 
\begin{equation}\label{ae87}
\tilde{n} = n - \varpi_1
\end{equation}
 the {\em rate of motion in mean anomaly}. 

Our procedure for determining the ecliptic longitude of the sun is described
below. 
The requisite orbital elements ({\em i.e.}, $e$, $n$, $\tilde{n}$, $\bar{\lambda}_0$, and $M_0$) for the J2000 epoch ({\em i.e.}, 12:00 UT on January 1, 2000 CE, which corresponds to $t_0= 2\,451\,545.0$ JD) are listed
in Table~\ref{lt4}. These elements are calculated
on the assumption that the vernal equinox  {\em precesses}\/ at the uniform
rate of $-3.8246\times 10^{-5}\,\,^\circ/{\rm day}$. 
The ecliptic longitude of the sun is specified by the
following formulae:
\begin{eqnarray}
\bar{\lambda} &=&  \bar{\lambda}_0+ n\,(t-t_0),\\[0.5ex]
M &=& M_0 + \tilde{n}\,(t-t_0),\\[0.5ex]
q &=& 2\,e\,\sin M + (5/4)\,e^2\,\sin\,2M,\label{le2.8}\\[0.5ex]
\lambda &=& \bar{\lambda}  + q.
\end{eqnarray}
 These formulae are capable of matching NASA ephemeris data
during the years 1995--2006 CE (see {\tt http://}\-{\tt ssd.jpl.nasa.gov/}) with a mean error of $0.2'$ and a maximum error of $0.7'$. 

The ecliptic longitude of the sun can be calculated with the aid of Tables~\ref{lt5} and \ref{lt6}.
 Table~\ref{lt5} allows the mean longitude, $\bar{\lambda}$, and mean anomaly,
 $M$, of the
sun to be determined as  functions of time. Table~\ref{lt6} specifies the equation of center, $q$, as a
function of the mean anomaly. 

The procedure for using the tables is as follows:
\begin{enumerate}
\item Determine the fractional Julian day number, $t$, corresponding to the date and time
at which the sun's ecliptic longitude is to be calculated with the aid of Tables~\ref{kt1}--\ref{kt3}. Form $\Delta t = t-t_0$, where $t_0=2\,451\,545.0$ is the epoch. 
\item Enter Table~\ref{lt5} with the digit for each power of 10
in ${\Delta} t$ and take out the corresponding values of $\Delta\bar{\lambda}$ and $\Delta M$. If $\Delta t$ is negative then the corresponding
values are also negative.
The value of the mean longitude, $\bar{\lambda}$, is the
sum of all the $\Delta\bar{\lambda}$ values plus the value of $\bar{\lambda}$ at the epoch. Likewise, the value of the mean anomaly, $M$, is
the sum of all the $\Delta M$ values plus the value of $M$ at the epoch. 
Add as many multiples of $360^\circ$ to $\bar{\lambda}$ and $M$
as is required to make them both fall in the range $0^\circ$ to $360^\circ$. Round $M$ to the nearest degree. 
\item Enter Table~\ref{lt6} with the value of $M$ and take out the
corresponding value of the equation of center, $q$, and the radial anomaly, $\zeta$. (The latter step is only necessary if the ecliptic longitude of the sun is
to be used to determine that of a planet.) It is necessary to interpolate if $M$ is odd.
\item The ecliptic longitude, $\lambda$, is the sum of the mean longitude, $\bar{\lambda}$, and the equation of center, $q$. If necessary, convert $\lambda$ into an angle in the range $0^\circ$ to $360^\circ$. 
The decimal fraction can be converted into arc minutes
using Table~\ref{lt6a}. Round to the nearest arc minute. 
\end{enumerate}
Two examples of the use of this procedure are given below.

\section{Example Longitude Calculations}
\noindent {\em Example 1}: May 5, 2005 CE, 00:00 UT:\\
~\\
According to Tables~\ref{kt1}--\ref{kt3}, $t = 2\,453\,495.5$ JD. Hence,
$t-t_0 = 2\,453\,495.5-2\,451\,545.0=1\,950.5$ JD. Making use of
Table~\ref{lt5}, we find:\\
\begin{tabular}{rrr}
&&\\
$t$(JD) & $ \bar{\lambda}(^\circ)$ & $M(^\circ)$\\[-2ex]
&&\\
+1000 & $265.647$ & $265.600$\\
+900 & $167.083$ & $167.040$\\
+50 & $49.280$ & $49.280$\\
+.5 & $0.493$ & $0.493$\\
Epoch & $280.458$ & $357.588$\\\cline{2-3}
&$762.961$ & $840.001$\\\cline{2-3}
Modulus & $42.961$ & $120.001$\\ 
&&\\
\end{tabular}\\
Rounding the mean anomaly to the nearest degree, we obtain $M\simeq 120^\circ$.
It follows from  Table~\ref{lt6} that
$$
q(120^\circ)= 1.641^\circ,
$$ 
so
$$
\lambda =\bar{\lambda} + q =42.961+ 1.641=44.602\simeq 44^\circ36'.
$$
Here, we have converted the decimal fraction into arc
minutes using Table~\ref{lt6a}, and 
then rounded the final result to the nearest arc minute.

 Following
the practice of the Ancient Greeks (and modern-day astrologers), we shall express ecliptic longitudes
in terms of the {\em signs of the zodiac}, which  are listed in Sect.~\ref{szod}. The ecliptic longitude $44^\circ36'$ is conventionally written 14TA36: {\em i.e.}, 
$14^\circ36'$ into the sign of Taurus. Thus, we conclude that the position of
the sun 
at 00:00 UT on May 5, 2005 CE was 14TA36.

~\\
\noindent {\em Example 2}: December 25, 1800 CE, 00:00 UT:\\
~\\
According to Tables~\ref{kt1}--\ref{kt3}, $t = 2\,378\,854.5$ JD. Hence,
$ t-t_0 = 2\,378\,854.5-2\,451\,545.0=-72\,690.5$ JD. Making use of
Table~\ref{lt5}, we find:\\
\begin{tabular}{rrr}
&&\\
$t$(JD) & $\bar{\lambda}(^\circ)$ & $ M(^\circ)$\\[-2ex]
&&\\
-70,000 & $-235.315$ & $-232.017$\\
-2,000 & $-171.295$ & $-171.200$\\
-600 & $-231.388$ & $-231.360$\\
-90 & $-88.708$ & $-88.704$\\
-.5 & $-0.493$ & $-0.493$\\
Epoch & $280.458$ & $357.588$\\\cline{2-3}
&$-446.741$ & $-366.186$\\\cline{2-3}
Modulus & $273.259$ & $353.814$\\
&&\\
\end{tabular}\\
We conclude that  $M\simeq 354^\circ$. 
 From Table~\ref{lt6}, 
$$
q(354^\circ)= -0.204^\circ,
$$  
so
$$
\lambda =\bar{\lambda} + q = 273.259 - 0.204=273.055\simeq 273^\circ03'.
$$
 Thus, the position of the sun at 00:00 UT on December 25, 1800 CE was 3CP03. 

\section{Determination of Equinox and Solstice Dates}
We can also use Tables~\ref{lt5} and \ref{lt6} to calculate the dates of the equinoxes and solstices,
and, hence, the lengths of the seasons, in a given year. The {\em vernal equinox}\/ ({\em i.e.}, the point on the sun's apparent orbit at which it passes through the celestial equator
from south to north)
corresponds to $\lambda=0^\circ$, the {\em summer solstice}\/ ({\em i.e.}, the
point at which the sun is furthest north of the celestial equator) to $\lambda=90^\circ$, the {\em autumnal equinox}\/ ({\em i.e.}, the point at which the
sun passes through the celestial equator from north to south) to $\lambda = 180^\circ$, and the
{\em winter solstice}\/ ({\em i.e.}, the point at which the sun is furthest south of the celestial equator) to $\lambda = 270^\circ$---see Fig.~\ref{lf6a}. Furthermore, {\em spring}\/ is defined as the period between the spring
equinox and the summer solstice, {\em summer}\/ as the period between the summer solstice and
the autumnal equinox, {\em autumn}\/ as the period between the autumnal equinox and the
winter solstice, and {\em winter}\/ as the period between the winter solstice and the following
vernal equinox. Consider the year 2000 CE.
For the case of the vernal equinox, we can
 first estimate the
 time at which this event takes place by approximating the solar
 longitude as the {\em mean
solar longitude}: {\em i.e.}, 
$$
\lambda\simeq \bar{\lambda} = \bar{\lambda}_0 + n\,(t-t_0)
= 280.458 +  0.98564735\,(t-t_0),
$$
 We obtain 
$$
t \simeq t_0+(360-280.458)/0.98564735 \simeq t_0+81\,{\rm JD}.
$$ 
Calculating the true solar longitude at this time, using Tables~\ref{lt5} and \ref{lt6}, we get  
$\lambda = 2.177^\circ.$ Now, the actual vernal equinox occurs
when $\lambda=0^\circ$.
Thus, a much better estimate for the date of the vernal equinox
is
$$
t = t_0 + 81 -2.177/0.98564735 \simeq t_0 + 78.8\,{\rm JD},
$$
 which
corresponds to 7:00 UT on March 20. Similar calculations show that the summer solstice takes place at 
$$
t = t_0+ 171.6\, {\rm JD},
$$
corresponding to 2:00 UT on June 21, that the autumnal equinox
takes place at 
$$
t = t_0+265.2\,{\rm JD},
$$
corresponding to 17:00 UT
on September 22, and that the winter solstice takes place at
$$
t =t_0+355.1\, {\rm JD},
$$
corresponding to 14:00 UT on December 21. 
Thus, the length of spring is $92.8$ days,
the length of summer $93.6$ days,
and the length of autumn  $89.9$ days.
Finally, the length of winter is the length
of the tropical year ({\em i.e.}, the time period between successive vernal equinoxes), which is $360/0.98564735 = 325.24$ days, minus the sum of the lengths of the
other three seasons. This gives $88.9$ days. 

\begin{figure}
\epsfysize=3.5in
\centerline{\epsffile{epsfiles/solstice.eps}}
\caption[\em The equinoxes and solstices.]{\em The sun's apparent orbit around the earth, $G$, showing the vernal equinox (VE), summer
solstice (SS), autumnal equinox (AE), and winter solstice (WS). Here, $\lambda$, $\Pi$, $A$, and $C$
are the ecliptic longitude, perigee, apogee, and geometric center of the orbit, respectively. The lengths
of the seasons (in days) are  indicated. }\label{lf6a}   
\end{figure}


Figure~\ref{lf6a} illustrates the relationship between the equinox and solstice points, and the
lengths of the seasons. The earth is displaced from the geometric center of the sun's apparent orbit in the direction of
the solar perigee, which presently lies between the winter solstice and the vernal equinox. This displacement (which is
greatly exaggerated in the figure) has
two effects. Firstly, it causes the arc of the sun's apparent orbit between the summer solstice and autumnal equinox
to be longer than that between the winter solstice and the vernal equinox. Secondly, it causes the
sun to appear to move faster in winter than in summer, in accordance with Kepler's second law, since the sun is closer to the earth in the
former season. Both of these effects tend to lengthen summer, and
shorten winter. Hence, summer is presently the longest season, and winter the shortest.

\section{Equation of Time}
At any particular observation site on the earth's surface, {\em local noon}\/
is defined as the instant in time when the sun culminates at the
meridian. However, as a consequence of the inclination of the
ecliptic to the celestial equatior, as well as the uneven motion of the
sun around the ecliptic, the time interval  between successive local noons, which
is known as a {\em solar day}, 
is not constant, but varies throughout the year. Hence, if we were to
define a second as $1/86,400$ of a solar day then the length of a second
would also vary throughout the year, which is clearly undesirable. In
order to avoid this problem, astronomers have invented a fictitious
body called the {\em mean sun}. The mean sun travels around the
celestial equator (from west to east) at a constant
rate which is such that it completes one orbit every tropical year. Moreover,
the mean sun and the true sun coincide at the spring equinox. {\em Local mean noon}\/ at a particular observation
site is defined as the instance in time when the mean sun culminates at
the meridian. Since the orbit of the mean sun is not inclined to the
celestial equator, and the mean sun travels  around the celestial equator at a uniform rate, the time
interval between successive mean noons, which is known as a
{\em mean solar day}, takes the constant value of 24 hours, or
86,400 seconds, throughout the year. {\em Universal time}\/ (UT)
is defined such that 12:00  UT coincides with mean noon every day at
an observation site of terrestrial longitude $0^\circ$. If we define {\em local  time}\/
(LT)
as $LT = UT- \phi(^\circ)/15^\circ$ hrs., where $\phi$ is the terrestrial longitude
of the observation site, then 12:00  LT coincides with mean noon
every day at a general observation site on the earth's surface.

According to the above definition, the right ascension, $\bar{\alpha}$, of the mean
sun satisfies
\begin{equation}
\bar{\alpha} = \bar{\lambda},
\end{equation}
where $\bar{\lambda}$ is the sun's mean ecliptic longitude. Moreover,
it follows from Eqs.~(\ref{e15r})  and (\ref{ae82}) that the right ascension
of the true sun is given by
\begin{equation}
\tan\alpha = \cos\epsilon\,\tan(\bar{\lambda}+q),
\end{equation}
where $\epsilon$ is the inclination of the ecliptic to the celestial
equator, $q(M)$  the sun's equation of center, and $M$ its mean anomaly. Now, neglecting the small time variation of the longitude of the
sun's perigee [{\em i.e.}, setting $\varpi_1=0$ in Eq.~(\ref{ae79})], we 
can write [see Eqs.~(\ref{ae84}), (\ref{ae85}), and (\ref{ae87}), as
well as Table~\ref{lt4}]
\begin{equation}
M = \bar{\lambda} + M_0-\bar{\lambda}_0 = \bar{\lambda} +77.213^\circ.
\end{equation}
It follows that, to first order in the solar eccentricity, $e$, we have
\begin{equation}
\Delta\alpha = \bar{\alpha}-\alpha = \lambda - \tan^{-1}(\cos \epsilon\,\tan\lambda) - 2\,e\,\sin\,M,
\end{equation}
where
\begin{equation}
M = \lambda + 77.213^\circ.
\end{equation}
Now,
\begin{equation}
\Delta t = \Delta\alpha(^\circ)/15^\circ
\end{equation}
represents the time difference (in hours) between local noon and mean local noon (since
right ascension crosses the meridian at the uniform rate of $15^\circ$ an
hour), and  is known as the {\em equation of time}. If $\Delta t$ is
positive then local noon occurs {\em before}\/ mean local noon, and {\em vice
versa}. 

The equation of time specifies the difference between  time calculated using a sundial or sextant---which is known as
{\em solar time}---and
time obtained from  an accurate clock---which is known as {\em mean solar time}. Table~\ref{ttime} shows the equation of time as a function of the sun's
ecliptic longitude. It can be seen that the difference between solar time and mean solar time can be as much as
16 minutes, and attains its maximum value between the autumnal equinox and the winter solstice, and its
minimum value between the winter solstice and vernal equinox.


\clearpage
\begin{table}\centering
\begin{tabular}{l|rcrrrr}
Object &  $a\,(AU)$ & $e$ & $n\,(^\circ/{\rm day})$ & $\tilde{n}\,(^\circ/{\rm day})$ & $\bar{\lambda}_0\,(^\circ)$ & $M_0\,(^\circ)$\\\hline
&&&&&&\\[-2.2ex]
Mercury & $0.387098$  & $0.205636$ & $4.09237703$     & $4.09233439$    & $252.087$ & $174.693$  \\
Venus     & $0.723334$  & $0.006777$  & $1.60216872$    & $1.60213040$  & $181.973$ & $49.237$ \\
Sun       & $1.000000$  & $0.016711$  & $0.98564735$      & $0.98560025$  & $280.458$ & $357.588$   \\
Mars      & $1.523706$  & $0.093394$  & $0.52407118$     & $0.52402076$  & $355.460$ & $19.388$ \\ 
Jupiter   & $5.202873$  & $0.048386$  & $0.08312507$     & $0.08308100$    & $34.365$ & $19.348$\\
Saturn   & $9.536651$  & $0.053862$    & $0.03350830$     & $0.03348152$   & $50.059$ & $317.857$ \\
\end{tabular}
\caption[\em  Keplerian orbital elements for the sun and the five visible planets.]{\em Keplerian orbital elements for the sun and the five visible planets at the J2000 epoch ({\em i.e.}, 12:00 UT, January 1, 2000 CE,
which corresponds to $t_0 = 2\,451\,545.0$ JD). The elements are optimized for use in the
time period 1800 CE to 2050 CE. Source: Jet Propulsion Laboratory (NASA), {\tt http://ssd.jpl.nasa.gov/}.  The motion rates have been converted into tropical motion rates assuming a uniform precession of the equinoxes
 of $3.8246\times 10^{-5}\,\,^\circ/{\rm day}$.}\label{lt4}
\end{table}

\clearpage
\begin{table}\centering
{\small\begin{tabular}{ll|ll|ll|ll|ll|ll}
$00.0'$ & .000 & $10.0'$ & .167 & $20.0'$ & .333 & $30.0'$ & .500 & $40.0'$ & .667 & $50.0'$ & .833\\
$00.2'$ & .003 & $10.2'$ & .170 & $20.2'$ & .337 & $30.2'$ & .503 & $40.2'$ & .670 & $50.2'$ & .837\\
$00.4'$ & .007 & $10.4'$ & .173 & $20.4'$ & .340 & $30.4'$ & .507 & $40.4'$ & .673 & $50.4'$ & .840\\
$00.6'$ & .010 & $10.6'$ & .177 & $20.6'$ & .343 & $30.6'$ & .510 & $40.6'$ & .677 & $50.6'$ & .843\\
$00.8'$ & .013 & $10.8'$ & .180 & $20.8'$ & .347 & $30.8'$ & .513 & $40.8'$ & .680 & $50.8'$ & .847\\
$01.0'$ & .017 & $11.0'$ & .183 & $21.0'$ & .350 & $31.0'$ & .517 & $41.0'$ & .683 & $51.0'$ & .850\\
$01.2'$ & .020 & $11.2'$ & .187 & $21.2'$ & .353 & $31.2'$ & .520 & $41.2'$ & .687 & $51.2'$ & .853\\
$01.4'$ & .023 & $11.4'$ & .190 & $21.4'$ & .357 & $31.4'$ & .523 & $41.4'$ & .690 & $51.4'$ & .857\\
$01.6'$ & .027 & $11.6'$ & .193 & $21.6'$ & .360 & $31.6'$ & .527 & $41.6'$ & .693 & $51.6'$ & .860\\
$01.8'$ & .030 & $11.8'$ & .197 & $21.8'$ & .363 & $31.8'$ & .530 & $41.8'$ & .697 & $51.8'$ & .863\\
$02.0'$ & .033 & $12.0'$ & .200 & $22.0'$ & .367 & $32.0'$ & .533 & $42.0'$ & .700 & $52.0'$ & .867\\
$02.2'$ & .037 & $12.2'$ & .203 & $22.2'$ & .370 & $32.2'$ & .537 & $42.2'$ & .703 & $52.2'$ & .870\\
$02.4'$ & .040 & $12.4'$ & .207 & $22.4'$ & .373 & $32.4'$ & .540 & $42.4'$ & .707 & $52.4'$ & .873\\
$02.6'$ & .043 & $12.6'$ & .210 & $22.6'$ & .377 & $32.6'$ & .543 & $42.6'$ & .710 & $52.6'$ & .877\\
$02.8'$ & .047 & $12.8'$ & .213 & $22.8'$ & .380 & $32.8'$ & .547 & $42.8'$ & .713 & $52.8'$ & .880\\
$03.0'$ & .050 & $13.0'$ & .217 & $23.0'$ & .383 & $33.0'$ & .550 & $43.0'$ & .717 & $53.0'$ & .883\\
$03.2'$ & .053 & $13.2'$ & .220 & $23.2'$ & .387 & $33.2'$ & .553 & $43.2'$ & .720 & $53.2'$ & .887\\
$03.4'$ & .057 & $13.4'$ & .223 & $23.4'$ & .390 & $33.4'$ & .557 & $43.4'$ & .723 & $53.4'$ & .890\\
$03.6'$ & .060 & $13.6'$ & .227 & $23.6'$ & .393 & $33.6'$ & .560 & $43.6'$ & .727 & $53.6'$ & .893\\
$03.8'$ & .063 & $13.8'$ & .230 & $23.8'$ & .397 & $33.8'$ & .563 & $43.8'$ & .730 & $53.8'$ & .897\\
$04.0'$ & .067 & $14.0'$ & .233 & $24.0'$ & .400 & $34.0'$ & .567 & $44.0'$ & .733 & $54.0'$ & .900\\
$04.2'$ & .070 & $14.2'$ & .237 & $24.2'$ & .403 & $34.2'$ & .570 & $44.2'$ & .737 & $54.2'$ & .903\\
$04.4'$ & .073 & $14.4'$ & .240 & $24.4'$ & .407 & $34.4'$ & .573 & $44.4'$ & .740 & $54.4'$ & .907\\
$04.6'$ & .077 & $14.6'$ & .243 & $24.6'$ & .410 & $34.6'$ & .577 & $44.6'$ & .743 & $54.6'$ & .910\\
$04.8'$ & .080 & $14.8'$ & .247 & $24.8'$ & .413 & $34.8'$ & .580 & $44.8'$ & .747 & $54.8'$ & .913\\
$05.0'$ & .083 & $15.0'$ & .250 & $25.0'$ & .417 & $35.0'$ & .583 & $45.0'$ & .750 & $55.0'$ & .917\\
$05.2'$ & .087 & $15.2'$ & .253 & $25.2'$ & .420 & $35.2'$ & .587 & $45.2'$ & .753 & $55.2'$ & .920\\
$05.4'$ & .090 & $15.4'$ & .257 & $25.4'$ & .423 & $35.4'$ & .590 & $45.4'$ & .757 & $55.4'$ & .923\\
$05.6'$ & .093 & $15.6'$ & .260 & $25.6'$ & .427 & $35.6'$ & .593 & $45.6'$ & .760 & $55.6'$ & .927\\
$05.8'$ & .097 & $15.8'$ & .263 & $25.8'$ & .430 & $35.8'$ & .597 & $45.8'$ & .763 & $55.8'$ & .930\\
$06.0'$ & .100 & $16.0'$ & .267 & $26.0'$ & .433 & $36.0'$ & .600 & $46.0'$ & .767 & $56.0'$ & .933\\
$06.2'$ & .103 & $16.2'$ & .270 & $26.2'$ & .437 & $36.2'$ & .603 & $46.2'$ & .770 & $56.2'$ & .937\\
$06.4'$ & .107 & $16.4'$ & .273 & $26.4'$ & .440 & $36.4'$ & .607 & $46.4'$ & .773 & $56.4'$ & .940\\
$06.6'$ & .110 & $16.6'$ & .277 & $26.6'$ & .443 & $36.6'$ & .610 & $46.6'$ & .777 & $56.6'$ & .943\\
$06.8'$ & .113 & $16.8'$ & .280 & $26.8'$ & .447 & $36.8'$ & .613 & $46.8'$ & .780 & $56.8'$ & .947\\
$07.0'$ & .117 & $17.0'$ & .283 & $27.0'$ & .450 & $37.0'$ & .617 & $47.0'$ & .783 & $57.0'$ & .950\\
$07.2'$ & .120 & $17.2'$ & .287 & $27.2'$ & .453 & $37.2'$ & .620 & $47.2'$ & .787 & $57.2'$ & .953\\
$07.4'$ & .123 & $17.4'$ & .290 & $27.4'$ & .457 & $37.4'$ & .623 & $47.4'$ & .790 & $57.4'$ & .957\\
$07.6'$ & .127 & $17.6'$ & .293 & $27.6'$ & .460 & $37.6'$ & .627 & $47.6'$ & .793 & $57.6'$ & .960\\
$07.8'$ & .130 & $17.8'$ & .297 & $27.8'$ & .463 & $37.8'$ & .630 & $47.8'$ & .797 & $57.8'$ & .963\\
$08.0'$ & .133 & $18.0'$ & .300 & $28.0'$ & .467 & $38.0'$ & .633 & $48.0'$ & .800 & $58.0'$ & .967\\
$08.2'$ & .137 & $18.2'$ & .303 & $28.2'$ & .470 & $38.2'$ & .637 & $48.2'$ & .803 & $58.2'$ & .970\\
$08.4'$ & .140 & $18.4'$ & .307 & $28.4'$ & .473 & $38.4'$ & .640 & $48.4'$ & .807 & $58.4'$ & .973\\
$08.6'$ & .143 & $18.6'$ & .310 & $28.6'$ & .477 & $38.6'$ & .643 & $48.6'$ & .810 & $58.6'$ & .977\\
$08.8'$ & .147 & $18.8'$ & .313 & $28.8'$ & .480 & $38.8'$ & .647 & $48.8'$ & .813 & $58.8'$ & .980\\
$09.0'$ & .150 & $19.0'$ & .317 & $29.0'$ & .483 & $39.0'$ & .650 & $49.0'$ & .817 & $59.0'$ & .983\\
$09.2'$ & .153 & $19.2'$ & .320 & $29.2'$ & .487 & $39.2'$ & .653 & $49.2'$ & .820 & $59.2'$ & .987\\
$09.4'$ & .157 & $19.4'$ & .323 & $29.4'$ & .490 & $39.4'$ & .657 & $49.4'$ & .823 & $59.4'$ & .990\\
$09.6'$ & .160 & $19.6'$ & .327 & $29.6'$ & .493 & $39.6'$ & .660 & $49.6'$ & .827 & $59.6'$ & .993\\
$09.8'$ & .163 & $19.8'$ & .330 & $29.8'$ & .497 & $39.8'$ & .663 & $49.8'$ & .830 & $59.8'$ & .997\\
\end{tabular}}
\caption{\em Arc minute to decimal fraction conversion table.}\label{lt6a}
\end{table}

\newpage
\begin{table}[h]
\centering
\begin{tabular}{rrr|rrr|rrr}
$\Delta t$(JD)& $\Delta\bar{\lambda}(^\circ)$ &  $\Delta M(^\circ)$ & $\Delta t$(JD)& $\Delta \bar{\lambda}(^\circ)$ & $\Delta M(^\circ)$ &$\Delta t$(JD)& $\Delta \bar{\lambda}(^\circ)$ & $\Delta M(^\circ)$\\ \hline
&&&&&&&&\\[-1.75ex]
10,000 & 136.474 & 136.002 & 1,000 & 265.647 & 265.600 & 100 &  98.565 &  98.560\\
20,000 & 272.947 & 272.005 & 2,000 & 171.295 & 171.200 & 200 & 197.129 & 197.120\\
30,000 &  49.421 &  48.007 & 3,000 &  76.942 &  76.801 & 300 & 295.694 & 295.680\\
40,000 & 185.894 & 184.010 & 4,000 & 342.589 & 342.401 & 400 &  34.259 &  34.240\\
50,000 & 322.367 & 320.012 & 5,000 & 248.237 & 248.001 & 500 & 132.824 & 132.800\\
60,000 &  98.841 &  96.015 & 6,000 & 153.884 & 153.601 & 600 & 231.388 & 231.360\\
70,000 & 235.315 & 232.017 & 7,000 &  59.531 &  59.202 & 700 & 329.953 & 329.920\\
80,000 &  11.788 &   8.020 & 8,000 & 325.179 & 324.802 & 800 &  68.518 &  68.480\\
90,000 & 148.262 & 144.022 & 9,000 & 230.826 & 230.402 & 900 & 167.083 & 167.040\\
&&&&&&&&\\
10 &   9.856 &   9.856 & 1 &   0.986 &   0.986 & 0.1 &   0.099 &   0.099\\
20 &  19.713 &  19.712 & 2 &   1.971 &   1.971 & 0.2 &   0.197 &   0.197\\
30 &  29.569 &  29.568 & 3 &   2.957 &   2.957 & 0.3 &   0.296 &   0.296\\
40 &  39.426 &  39.424 & 4 &   3.943 &   3.942 & 0.4 &   0.394 &   0.394\\
50 &  49.282 &  49.280 & 5 &   4.928 &   4.928 & 0.5 &   0.493 &   0.493\\
60 &  59.139 &  59.136 & 6 &   5.914 &   5.914 & 0.6 &   0.591 &   0.591\\
70 &  68.995 &  68.992 & 7 &   6.900 &   6.899 & 0.7 &   0.690 &   0.690\\
80 &  78.852 &  78.848 & 8 &   7.885 &   7.885 & 0.8 &   0.789 &   0.788\\
90 &  88.708 &  88.704 & 9 &   8.871 &   8.870 & 0.9 &   0.887 &   0.887\\
\end{tabular}
\caption[\em Mean motion of the sun.]{\em Mean motion of the sun.  Here, $\Delta t = t-t_0$, $\Delta\bar{\lambda} = \bar{\lambda}-\bar{\lambda}_0$, and $\Delta M = M - M_0$. 
At epoch  ($t_0 = 2\,451\,545.0$ JD), $\bar{\lambda}_0 = 280.458^\circ$, and $M_0 = 357.588^\circ$.}\label{lt5}
\end{table}

\newpage
\begin{table}\centering
\small{ \begin{tabular}{rrr|rrr|rrr|rrr}
$M(^\circ)$ & $q(^\circ)$  & $100\,\zeta$ & $M(^\circ)$ & $q(^\circ)$  & $100\,\zeta$ & $M(^\circ)$ & $q(^\circ)$  & $100\,\zeta$& $M(^\circ)$ & $q(^\circ)$  & $100\,\zeta$\\\hline
&&&&&&&&&&&\\[-1.75ex]
  0 &  0.000 &  1.671 &  90 &  1.915 & -0.028 & 180 &  0.000 & -1.671 & 270 & -1.915 & -0.028\\
  2 &  0.068 &  1.670 &  92 &  1.912 & -0.086 & 182 & -0.065 & -1.670 & 272 & -1.915 &  0.030\\
  4 &  0.136 &  1.667 &  94 &  1.907 & -0.144 & 184 & -0.131 & -1.667 & 274 & -1.913 &  0.089\\
  6 &  0.204 &  1.662 &  96 &  1.900 & -0.202 & 186 & -0.196 & -1.662 & 276 & -1.909 &  0.147\\
  8 &  0.272 &  1.654 &  98 &  1.891 & -0.260 & 188 & -0.261 & -1.655 & 278 & -1.902 &  0.205\\
 10 &  0.339 &  1.645 & 100 &  1.879 & -0.317 & 190 & -0.326 & -1.647 & 280 & -1.893 &  0.263\\
 12 &  0.406 &  1.633 & 102 &  1.865 & -0.374 & 192 & -0.390 & -1.636 & 282 & -1.881 &  0.321\\
 14 &  0.473 &  1.620 & 104 &  1.849 & -0.431 & 194 & -0.454 & -1.623 & 284 & -1.867 &  0.378\\
 16 &  0.538 &  1.604 & 106 &  1.830 & -0.486 & 196 & -0.517 & -1.608 & 286 & -1.851 &  0.435\\
 18 &  0.604 &  1.587 & 108 &  1.809 & -0.542 & 198 & -0.580 & -1.592 & 288 & -1.833 &  0.491\\
 20 &  0.668 &  1.567 & 110 &  1.787 & -0.596 & 200 & -0.642 & -1.574 & 290 & -1.812 &  0.547\\
 22 &  0.731 &  1.545 & 112 &  1.762 & -0.650 & 202 & -0.703 & -1.553 & 292 & -1.789 &  0.602\\
 24 &  0.794 &  1.522 & 114 &  1.735 & -0.703 & 204 & -0.764 & -1.531 & 294 & -1.764 &  0.656\\
 26 &  0.855 &  1.497 & 116 &  1.705 & -0.755 & 206 & -0.824 & -1.507 & 296 & -1.737 &  0.710\\
 28 &  0.916 &  1.469 & 118 &  1.674 & -0.806 & 208 & -0.882 & -1.482 & 298 & -1.707 &  0.763\\
 30 &  0.975 &  1.440 & 120 &  1.641 & -0.856 & 210 & -0.940 & -1.454 & 300 & -1.676 &  0.815\\
 32 &  1.033 &  1.409 & 122 &  1.606 & -0.906 & 212 & -0.997 & -1.425 & 302 & -1.642 &  0.865\\
 34 &  1.089 &  1.377 & 124 &  1.569 & -0.954 & 214 & -1.052 & -1.394 & 304 & -1.606 &  0.915\\
 36 &  1.145 &  1.342 & 126 &  1.530 & -1.001 & 216 & -1.107 & -1.362 & 306 & -1.568 &  0.964\\
 38 &  1.198 &  1.306 & 128 &  1.490 & -1.046 & 218 & -1.160 & -1.327 & 308 & -1.528 &  1.011\\
 40 &  1.251 &  1.269 & 130 &  1.447 & -1.091 & 220 & -1.211 & -1.292 & 310 & -1.487 &  1.058\\
 42 &  1.301 &  1.229 & 132 &  1.403 & -1.134 & 222 & -1.261 & -1.254 & 312 & -1.443 &  1.103\\
 44 &  1.350 &  1.189 & 134 &  1.358 & -1.175 & 224 & -1.310 & -1.216 & 314 & -1.397 &  1.146\\
 46 &  1.397 &  1.146 & 136 &  1.310 & -1.216 & 226 & -1.358 & -1.175 & 316 & -1.350 &  1.189\\
 48 &  1.443 &  1.103 & 138 &  1.261 & -1.254 & 228 & -1.403 & -1.134 & 318 & -1.301 &  1.229\\
 50 &  1.487 &  1.058 & 140 &  1.211 & -1.292 & 230 & -1.447 & -1.091 & 320 & -1.251 &  1.269\\
 52 &  1.528 &  1.011 & 142 &  1.160 & -1.327 & 232 & -1.490 & -1.046 & 322 & -1.198 &  1.306\\
 54 &  1.568 &  0.964 & 144 &  1.107 & -1.362 & 234 & -1.530 & -1.001 & 324 & -1.145 &  1.342\\
 56 &  1.606 &  0.915 & 146 &  1.052 & -1.394 & 236 & -1.569 & -0.954 & 326 & -1.089 &  1.377\\
 58 &  1.642 &  0.865 & 148 &  0.997 & -1.425 & 238 & -1.606 & -0.906 & 328 & -1.033 &  1.409\\
 60 &  1.676 &  0.815 & 150 &  0.940 & -1.454 & 240 & -1.641 & -0.856 & 330 & -0.975 &  1.440\\
 62 &  1.707 &  0.763 & 152 &  0.882 & -1.482 & 242 & -1.674 & -0.806 & 332 & -0.916 &  1.469\\
 64 &  1.737 &  0.710 & 154 &  0.824 & -1.507 & 244 & -1.705 & -0.755 & 334 & -0.855 &  1.497\\
 66 &  1.764 &  0.656 & 156 &  0.764 & -1.531 & 246 & -1.735 & -0.703 & 336 & -0.794 &  1.522\\
 68 &  1.789 &  0.602 & 158 &  0.703 & -1.553 & 248 & -1.762 & -0.650 & 338 & -0.731 &  1.545\\
 70 &  1.812 &  0.547 & 160 &  0.642 & -1.574 & 250 & -1.787 & -0.596 & 340 & -0.668 &  1.567\\
 72 &  1.833 &  0.491 & 162 &  0.580 & -1.592 & 252 & -1.809 & -0.542 & 342 & -0.604 &  1.587\\
 74 &  1.851 &  0.435 & 164 &  0.517 & -1.608 & 254 & -1.830 & -0.486 & 344 & -0.538 &  1.604\\
 76 &  1.867 &  0.378 & 166 &  0.454 & -1.623 & 256 & -1.849 & -0.431 & 346 & -0.473 &  1.620\\
 78 &  1.881 &  0.321 & 168 &  0.390 & -1.636 & 258 & -1.865 & -0.374 & 348 & -0.406 &  1.633\\
 80 &  1.893 &  0.263 & 170 &  0.326 & -1.647 & 260 & -1.879 & -0.317 & 350 & -0.339 &  1.645\\
 82 &  1.902 &  0.205 & 172 &  0.261 & -1.655 & 262 & -1.891 & -0.260 & 352 & -0.272 &  1.654\\
 84 &  1.909 &  0.147 & 174 &  0.196 & -1.662 & 264 & -1.900 & -0.202 & 354 & -0.204 &  1.662\\
 86 &  1.913 &  0.089 & 176 &  0.131 & -1.667 & 266 & -1.907 & -0.144 & 356 & -0.136 &  1.667\\
 88 &  1.915 &  0.030 & 178 &  0.065 & -1.670 & 268 & -1.912 & -0.086 & 358 & -0.068 &  1.670\\
 90 &  1.915 & -0.028 & 180 &  0.000 & -1.671 & 270 & -1.915 & -0.028 & 360 & -0.000 &  1.671\\ 
\end{tabular}}
\caption{\em Anomalies of the sun.}\label{lt6}
\end{table}

\begin{table}
\centering
{\small \begin{tabular}{cr|cr|cr|cr|cr|cr}
\multicolumn{2}{c}{Aries}\vline & \multicolumn{2}{c}{Taurus} \vline& \multicolumn{2}{c}{Gemini} \vline& \multicolumn{2}{c}{Cancer}\vline &
\multicolumn{2}{c}{Leo}\vline & \multicolumn{2}{c}{Virgo}\\\hline
$\lambda$& $\Delta t~~~~~$& $\lambda$& $\Delta t~~~~~$& $\lambda$& $\Delta t~~~~~$& $\lambda$& $\Delta t~~~~~$& $\lambda$& $\Delta t~~~~~$& $\lambda$& $\Delta t~~~~~$\\\hline
&&&&&&&&&&&\\[-2ex]
00$^\circ$ & -07$^m$28$^s$ & 00$^\circ$ & +01$^m$02$^s$ & 00$^\circ$ & +03$^m$30$^s$ & 00$^\circ$ & -01$^m$42$^s$ & 00$^\circ$ & -06$^m$27$^s$ & 00$^\circ$ & -02$^m$44$^s$\\
02$^\circ$ & -06$^m$51$^s$ & 02$^\circ$ & +01$^m$27$^s$ & 02$^\circ$ & +03$^m$21$^s$ & 02$^\circ$ & -02$^m$09$^s$ & 02$^\circ$ & -06$^m$30$^s$ & 02$^\circ$ & -02$^m$11$^s$\\
04$^\circ$ & -06$^m$15$^s$ & 04$^\circ$ & +01$^m$50$^s$ & 04$^\circ$ & +03$^m$10$^s$ & 04$^\circ$ & -02$^m$36$^s$ & 04$^\circ$ & -06$^m$31$^s$ & 04$^\circ$ & -01$^m$36$^s$\\
06$^\circ$ & -05$^m$38$^s$ & 06$^\circ$ & +02$^m$12$^s$ & 06$^\circ$ & +02$^m$56$^s$ & 06$^\circ$ & -03$^m$03$^s$ & 06$^\circ$ & -06$^m$29$^s$ & 06$^\circ$ & +00$^m$59$^s$\\
08$^\circ$ & -05$^m$01$^s$ & 08$^\circ$ & +02$^m$31$^s$ & 08$^\circ$ & +02$^m$41$^s$ & 08$^\circ$ & -03$^m$29$^s$ & 08$^\circ$ & -06$^m$24$^s$ & 08$^\circ$ & +00$^m$21$^s$\\
10$^\circ$ & -04$^m$24$^s$ & 10$^\circ$ & +02$^m$48$^s$ & 10$^\circ$ & +02$^m$23$^s$ & 10$^\circ$ & -03$^m$53$^s$ & 10$^\circ$ & -06$^m$16$^s$ & 10$^\circ$ & +00$^m$18$^s$\\
12$^\circ$ & -03$^m$48$^s$ & 12$^\circ$ & +03$^m$03$^s$ & 12$^\circ$ & +02$^m$04$^s$ & 12$^\circ$ & -04$^m$17$^s$ & 12$^\circ$ & -06$^m$06$^s$ & 12$^\circ$ & +00$^m$58$^s$\\
14$^\circ$ & -03$^m$12$^s$ & 14$^\circ$ & +03$^m$16$^s$ & 14$^\circ$ & +01$^m$43$^s$ & 14$^\circ$ & -04$^m$39$^s$ & 14$^\circ$ & -05$^m$54$^s$ & 14$^\circ$ & +01$^m$40$^s$\\
16$^\circ$ & -02$^m$36$^s$ & 16$^\circ$ & +03$^m$26$^s$ & 16$^\circ$ & +01$^m$20$^s$ & 16$^\circ$ & -05$^m$00$^s$ & 16$^\circ$ & -05$^m$38$^s$ & 16$^\circ$ & +02$^m$22$^s$\\
18$^\circ$ & -02$^m$01$^s$ & 18$^\circ$ & +03$^m$34$^s$ & 18$^\circ$ & +00$^m$57$^s$ & 18$^\circ$ & -05$^m$19$^s$ & 18$^\circ$ & -05$^m$20$^s$ & 18$^\circ$ & +03$^m$05$^s$\\
20$^\circ$ & -01$^m$28$^s$ & 20$^\circ$ & +03$^m$40$^s$ & 20$^\circ$ & +00$^m$32$^s$ & 20$^\circ$ & -05$^m$35$^s$ & 20$^\circ$ & -05$^m$00$^s$ & 20$^\circ$ & +03$^m$49$^s$\\
22$^\circ$ & +00$^m$55$^s$ & 22$^\circ$ & +03$^m$43$^s$ & 22$^\circ$ & +00$^m$06$^s$ & 22$^\circ$ & -05$^m$50$^s$ & 22$^\circ$ & -04$^m$37$^s$ & 22$^\circ$ & +04$^m$32$^s$\\
24$^\circ$ & +00$^m$23$^s$ & 24$^\circ$ & +03$^m$43$^s$ & 24$^\circ$ & +00$^m$20$^s$ & 24$^\circ$ & -06$^m$03$^s$ & 24$^\circ$ & -04$^m$12$^s$ & 24$^\circ$ & +05$^m$16$^s$\\
26$^\circ$ & +00$^m$06$^s$ & 26$^\circ$ & +03$^m$41$^s$ & 26$^\circ$ & +00$^m$47$^s$ & 26$^\circ$ & -06$^m$14$^s$ & 26$^\circ$ & -03$^m$45$^s$ & 26$^\circ$ & +06$^m$00$^s$\\
28$^\circ$ & +00$^m$34$^s$ & 28$^\circ$ & +03$^m$37$^s$ & 28$^\circ$ & -01$^m$14$^s$ & 28$^\circ$ & -06$^m$22$^s$ & 28$^\circ$ & -03$^m$15$^s$ & 28$^\circ$ & +06$^m$44$^s$\\
30$^\circ$ & +01$^m$02$^s$ & 30$^\circ$ & +03$^m$30$^s$ & 30$^\circ$ & -01$^m$42$^s$ & 30$^\circ$ & -06$^m$27$^s$ & 30$^\circ$ & -02$^m$44$^s$ & 30$^\circ$ & +07$^m$28$^s$\\
\multicolumn{12}{c}{}\\
\multicolumn{2}{c}{Libra}\vline & \multicolumn{2}{c}{Scorpio} \vline& \multicolumn{2}{c}{Sagittarius} \vline& \multicolumn{2}{c}{Capricorn}\vline &
\multicolumn{2}{c}{Aquarius}\vline & \multicolumn{2}{c}{Pisces}\\\hline
$\lambda$& $\Delta t~~~~~$& $\lambda$& $\Delta t~~~~~$& $\lambda$& $\Delta t~~~~~$& $\lambda$& $\Delta t~~~~~$& $\lambda$& $\Delta t~~~~~$& $\lambda$& $\Delta t~~~~~$\\\hline
&&&&&&&&&&&\\[-2ex]
00$^\circ$ & +07$^m$28$^s$ & 00$^\circ$ & +15$^m$40$^s$ & 00$^\circ$ & +13$^m$55$^s$ & 00$^\circ$ & +01$^m$42$^s$ & 00$^\circ$ & -10$^m$58$^s$ & 00$^\circ$ & -13$^m$58$^s$\\
02$^\circ$ & +08$^m$10$^s$ & 02$^\circ$ & +15$^m$55$^s$ & 02$^\circ$ & +13$^m$22$^s$ & 02$^\circ$ & +00$^m$43$^s$ & 02$^\circ$ & -11$^m$32$^s$ & 02$^\circ$ & -13$^m$46$^s$\\
04$^\circ$ & +08$^m$53$^s$ & 04$^\circ$ & +16$^m$08$^s$ & 04$^\circ$ & +12$^m$46$^s$ & 04$^\circ$ & +00$^m$15$^s$ & 04$^\circ$ & -12$^m$02$^s$ & 04$^\circ$ & -13$^m$31$^s$\\
06$^\circ$ & +09$^m$34$^s$ & 06$^\circ$ & +16$^m$17$^s$ & 06$^\circ$ & +12$^m$08$^s$ & 06$^\circ$ & -01$^m$13$^s$ & 06$^\circ$ & -12$^m$30$^s$ & 06$^\circ$ & -13$^m$14$^s$\\
08$^\circ$ & +10$^m$14$^s$ & 08$^\circ$ & +16$^m$23$^s$ & 08$^\circ$ & +11$^m$26$^s$ & 08$^\circ$ & -02$^m$11$^s$ & 08$^\circ$ & -12$^m$54$^s$ & 08$^\circ$ & -12$^m$55$^s$\\
10$^\circ$ & +10$^m$53$^s$ & 10$^\circ$ & +16$^m$26$^s$ & 10$^\circ$ & +10$^m$42$^s$ & 10$^\circ$ & -03$^m$07$^s$ & 10$^\circ$ & -13$^m$16$^s$ & 10$^\circ$ & -12$^m$33$^s$\\
12$^\circ$ & +11$^m$30$^s$ & 12$^\circ$ & +16$^m$26$^s$ & 12$^\circ$ & +09$^m$55$^s$ & 12$^\circ$ & -04$^m$03$^s$ & 12$^\circ$ & -13$^m$34$^s$ & 12$^\circ$ & -12$^m$10$^s$\\
14$^\circ$ & +12$^m$06$^s$ & 14$^\circ$ & +16$^m$23$^s$ & 14$^\circ$ & +09$^m$07$^s$ & 14$^\circ$ & -04$^m$57$^s$ & 14$^\circ$ & -13$^m$49$^s$ & 14$^\circ$ & -11$^m$44$^s$\\
16$^\circ$ & +12$^m$41$^s$ & 16$^\circ$ & +16$^m$16$^s$ & 16$^\circ$ & +08$^m$16$^s$ & 16$^\circ$ & -05$^m$50$^s$ & 16$^\circ$ & -14$^m$01$^s$ & 16$^\circ$ & -11$^m$17$^s$\\
18$^\circ$ & +13$^m$13$^s$ & 18$^\circ$ & +16$^m$06$^s$ & 18$^\circ$ & +07$^m$23$^s$ & 18$^\circ$ & -06$^m$41$^s$ & 18$^\circ$ & -14$^m$09$^s$ & 18$^\circ$ & -10$^m$48$^s$\\
20$^\circ$ & +13$^m$43$^s$ & 20$^\circ$ & +15$^m$52$^s$ & 20$^\circ$ & +06$^m$29$^s$ & 20$^\circ$ & -07$^m$30$^s$ & 20$^\circ$ & -14$^m$15$^s$ & 20$^\circ$ & -10$^m$17$^s$\\
22$^\circ$ & +14$^m$12$^s$ & 22$^\circ$ & +15$^m$36$^s$ & 22$^\circ$ & +05$^m$33$^s$ & 22$^\circ$ & -08$^m$16$^s$ & 22$^\circ$ & -14$^m$17$^s$ & 22$^\circ$ & -09$^m$45$^s$\\
24$^\circ$ & +14$^m$38$^s$ & 24$^\circ$ & +15$^m$15$^s$ & 24$^\circ$ & +04$^m$37$^s$ & 24$^\circ$ & -09$^m$01$^s$ & 24$^\circ$ & -14$^m$17$^s$ & 24$^\circ$ & -09$^m$12$^s$\\
26$^\circ$ & +15$^m$01$^s$ & 26$^\circ$ & +14$^m$52$^s$ & 26$^\circ$ & +03$^m$39$^s$ & 26$^\circ$ & -09$^m$42$^s$ & 26$^\circ$ & -14$^m$13$^s$ & 26$^\circ$ & -08$^m$38$^s$\\
28$^\circ$ & +15$^m$22$^s$ & 28$^\circ$ & +14$^m$25$^s$ & 28$^\circ$ & +02$^m$41$^s$ & 28$^\circ$ & -10$^m$22$^s$ & 28$^\circ$ & -14$^m$07$^s$ & 28$^\circ$ & -08$^m$03$^s$\\
30$^\circ$ & +15$^m$40$^s$ & 30$^\circ$ & +13$^m$55$^s$ & 30$^\circ$ & +01$^m$42$^s$ & 30$^\circ$ & -10$^m$58$^s$ & 30$^\circ$ & -13$^m$58$^s$ & 30$^\circ$ & -07$^m$28$^s$\\
\end{tabular}}
\caption[\em The equation of time.]{\em The equation of time. The superscripts $m$ and $s$ denote minutes and seconds.}\label{ttime}
\end{table}

