\chapter{Geometric Planetary Orbit Models}\label{ckep}
\section{Introduction}
In this section, 
 Kepler's geometric model of  a geocentric planetary orbit is  examined in detail, and then compared to the 
less accurate geometric models of Hipparchus, Ptolemy, and Copernicus. In the following,
all orbits are viewed from the {\em northern}\/ ecliptic pole.

\section{Model of Kepler}
Kepler's geometric model of a heliocentric planetary orbit is summed up in his   three well-known laws of planetary motion.
According to Kepler's first law, all planetary orbits are {\em ellipses}\/  which are {\em confocal}\/ with the sun and lie in a
fixed plane.
Moreover, according to Kepler's second law, the radius vector which connects the sun to a given planet sweeps out {\em equal areas in equal time intervals}. 

\begin{figure}[h]
\epsfysize=2.5in
\centerline{\epsffile{epsfiles/fig1.eps}}
\caption{\em A Keplerian orbit.}\label{jf1}
\end{figure}

Consider Figure~\ref{jf1}. $\Pi PBA$ is half  of an elliptical planetary orbit. Furthermore, $C$ is the
geometric center of the orbit, $S$ the focus at which the sun is located,
$P$ the instantaneous position of the planet, $\Pi$ the perihelion point ({\em i.e.}, the planet's point of closest approach to the sun),
and $A$ the aphelion point ({\em i.e.}, the  point of furthest distance from the sun). The ellipse is symmetric about
$\Pi A$, which is termed the {\em major axis}, and about $CB$,
which is termed the {\em minor axis}.
The length $CA\equiv a$ is called the orbital
{\em major radius}. The length $CS$  represents the displacement of the sun from
the geometric center of the orbit, and is generally written $e\,a$, where $e$ is termed the
orbital {\em eccentricity}, where $0\leq e\leq 1$. The length $CB\equiv b = a\,(1-e^2)^{1/2}$ is called
the orbital {\em minor radius}. 
The length $SP\equiv r$ represents the radial distance of the planet from the sun.
Finally, the angle $RSP\equiv T$ is the angular bearing of the planet from the sun,
relative to the major axis of the orbit, and is termed the {\em true anomaly}. 

$\Pi QDA$ is half of a circle whose geometric center is $C$, and whose
radius is $a$. Hence, the circle passes through the perihelion and aphelion
points. $R$ is the point at which the perpendicular from $P$ 
meets the major axis $\Pi A$. The point where $RP$ produced
meets circle $\Pi QDA$ is denoted $Q$. Finally,
the angle $SCQ\equiv E$ is called the {\em elliptic anomaly}. 

Now, the equation of the ellipse $\Pi PBA$ is
\begin{equation}
\frac{x^2}{a^2}+ \frac{y^2}{b^2} = 1,
\end{equation}
where $x$ and $y$ are the perpendicular distances from the minor and major
axes, respectively. Likewise, the equation of the circle $\Pi QDA$ is
\begin{equation}
\frac{x'^{\,2}}{a^2} + \frac{y'^{\,2}}{a^2} = 1.
\end{equation}
Hence, if $x=x'$ then
\begin{equation}
\frac{y}{y'} = \frac{b}{a},
\end{equation}
and it follows that
\begin{equation}\label{je4}
\frac{RP}{RQ} = \frac{b}{a}.
\end{equation}

Now, $CS = e\,a$. Furthermore, it is easily demonstrated that
 $SR=r\,\cos T$, $RP=r\,\sin T$, $CR= a\,\cos E$, and $RQ= a\,\sin E$. Consequently, Eq.~(\ref{je4}) yields
\begin{equation}\label{je5}
r\,\sin T = b\,\sin E = a\,(1-e^2)^{1/2}\,\sin E.
\end{equation}
Also, since $SR = CR-CS$, we have
\begin{equation}\label{je6}
r\,\cos T = a\,(\cos E - e).
\end{equation}
Taking the square root of the sum of the squares of the previous two equations, we obtain
\begin{equation}\label{je7}
r = a\,(1-e\,\cos E),
\end{equation}
which can be combined with Eq.~(\ref{je6}) to give
\begin{equation}
\cos T = \frac{\cos E - e}{1-e\,\cos E}.
\end{equation}

Now, according to Kepler's second law,
\begin{equation}
\frac{{\rm Area}\,\Pi PS}{\pi\,a\,b} = \frac{t-t_\ast}{\tau},
\end{equation}
where $t$ is the time at which the planet passes point $P$,  $t_\ast$  the time at which it passes the perihelion point,
and $\tau$ the {\em orbital period}.
However,
\begin{eqnarray}
{\rm Area}  \,\Pi PS &=& {\rm Area}\, SRP + {\rm Area} \,\Pi RP
= \frac{1}{2}\,r^2\,\cos T\,\sin T + {\rm Area}\,\Pi RP.
\end{eqnarray}
But,
\begin{equation}
{\rm Area}\,\Pi RP = \frac{b}{a}\,{\rm Area}\,\Pi RQ,
\end{equation}
since $RP/RQ = b/a$ for all values of $T$. In addition, 
\begin{eqnarray}
{\rm Area}\,\Pi RQ = {\rm Area}\,\Pi QC - {\rm Area}\,RQC= \frac{1}{2}\,E\,a^2 - \frac{1}{2}\,a^2\,\cos E\,\sin E.
\end{eqnarray}
Hence, we can write
\begin{equation}
\left(\frac{t-t_\ast}{\tau}\right) \pi\,a\,b= \frac{1}{2}\,r^2\,\cos T\,\sin T + \frac{b}{a}\,\frac{a^2}{2}\,(E - \cos E\,\sin E).
\end{equation}
According to Eqs.~(\ref{je5}) and (\ref{je6}), $r\,\sin T = b\,\sin E$, and $r\,\cos T = a\,(\cos E - e)$, so
the above expression reduces to
\begin{equation}
M = E - e\,\sin E,
\end{equation}
where
\begin{equation}
M = \left(\frac{2\pi}{\tau}\right)\,(t-t_\ast)
\end{equation}
is an angle which is zero at the perihelion point, increases {\em uniformly}\/ in time, and has a repetition period which
matches the period of the planetary orbit. This angle is termed the {\em mean anomaly}. 

In summary, the radial and angular polar coordinates, $r$ and $T$, respectively,
of a planet in a Keplerian orbit about the sun are specified as {\em implicit}\/ functions
of the mean anomaly, which is a {\em linear}\/ function of time, by the
following three equations:
\begin{eqnarray}
M &=& E - e\,\sin E,\\[0.5ex]
r &=& a\,(1- e\,\cos E),\\[0.5ex]
\cos T &=& \frac{\cos E - e}{1-e\,\cos E}.
\end{eqnarray}
It turns out that the earth and the five visible planets all possess {\em low eccentricity}\/ orbits characterized by $e\ll 1$. Hence, it is a good approximation to expand the above three equations
using $e$ as a small parameter. To second-order, we
get
\begin{eqnarray}
E &=& M + e\,\sin M + (1/2)\,e^2\,\sin 2M,\\[0.5ex]
r& =& a\,(1-e\,\cos T - e^2\,\sin^2 T),\\[0.5ex]
T &=& E + e\,\sin E + (1/4)\,e^2\,\sin 2 E.
\end{eqnarray}
Finally, these equations can be combined to give $r$ and $T$ as {\em explicit}\/ functions of the
mean anomaly:
\begin{eqnarray}
\frac{r}{a} &=& 1 -e\,\cos M + e^2\,\sin^2 M,\label{je22}\\[0.5ex]
T &=& M + 2\,e\,\sin M + (5/4)\,e^2\,\sin 2M.\label{je23}
\end{eqnarray}

\section{Model of Hipparchus}
Hipparchus' geometric model of the apparent orbit of the sun around the earth can also be used to describe a heliocentric planetary orbit. The model is illustrated  in Fig.~\ref{hipp}. The orbit of the planet corresponds
to the circle $\Pi P D A$ (only half of which is shown), where $\Pi$ is the perihelion point, $P$ the planet's instantaneous position,
and $A$ the aphelion point. The diameter $\Pi   S C A$ is the effective major axis of the orbit (to be more exact, it
is the line of apsides), where $C$ is the geometric center of circle $\Pi P D A$, and $S$ the fixed position of the sun.
The radius $CP$ of circle $\Pi PDA$ is   the effective major radius, $a$, of the orbit.  The distance $SC$ is
equal to $2\,e\,a$, where $e$ is the orbit's effective eccentricity. The angle $PC\Pi$
is  identified with the  mean anomaly, $M$, and increases {\em linearly}\/  in time.  In other words, as seen from $C$, the planet
$P$ moves {\em uniformly}\/ around circle $\Pi PDA$ in a counterclockwise direction. Finally, $SP$ is the radial
distance, $r$, of the planet from the sun, and angle $P S \Pi$ is the planet's true anomaly, $T$.

\begin{figure}[h]
\epsfysize=3in
\centerline{\epsffile{epsfiles/hipp1.eps}}
\caption{\em A Hipparchian orbit.}\label{hipp}
\end{figure}

Let us draw the straight-line $KSL$ parallel to $CP$, and passing through point $S$, and then complete the rectangle $PCKL$. Simple geometry reveals that 
$CK = PL = 2\,e\,a\,\sin M$, $KS=2\,e\,a\,\cos M$, and  $SL = a-2\,e\,a\,\cos M$. Moreover, $SP^2 = SL^2+ PL^2$,
which implies that
\begin{equation}
\frac{r}{a} = (1-4\,e\,\cos M + 4\,e^2)^{1/2}.
\end{equation}
Now, $T = M + q$, where $q$ is angle $PSL$. However,
\begin{equation}
\sin q = \frac{PL}{SP} = \frac{2\,e\,\sin M}{(1-4\,e\,\cos M + 4\,e^2)^{1/2}}.
\end{equation}

Finally, expanding the previous two equations to second-order in the small parameter $e$, we obtain
\begin{eqnarray}\label{e4.26}
\frac{r}{a} &=& 1 -2\,e\,\cos M + 2\,e^2\,\sin^2 M,\\[0.5ex]
T &=& M + 2\,e\,\sin M + 2\,e^2\,\sin 2M.\label{e4.27}
\end{eqnarray}
It can be seen, by comparison with Eqs.~(\ref{je22}) and (\ref{je23}), that   the relative radial distance, $r/a$, in the Hipparchian model deviates from that in the (correct) Keplerian model to {\em first-order}\/ in $e$ (in fact, the variation of $r/a$
is greater by a factor of $2$ in the former model), whereas 
 the true anomaly, $T$,  only deviates  to {\em second-order}\/ in $e$.  We conclude that  Hipparchus' geometric model
 of a heliocentric planetary orbit does a
 reasonably good job at predicting the angular position of the planet, relative to the sun, but significantly
 exaggerates (by a factor of $2$) the variation in the radial distance between the two during the course of a complete  orbital rotation. 

\section{Model of Ptolemy}
Ptolemy's geometric model of the motion of the center of an epicycle around a deferent can also be used to describe a heliocentric planetary orbit. The model is illustrated  in Fig.~\ref{ptol}. The orbit of the planet corresponds
to the circle $\Pi P D A$ (only half of which is shown), where $\Pi$ is the perihelion point, $P$ the planet's instantaneous position,
and $A$ the aphelion point. The diameter $\Pi   S C Q A$ is the effective major axis of the orbit, where $C$ is the geometric center of circle $\Pi P D A$,  $S$ the fixed position of the sun, and $Q$  the location of the so-called {\em equant}.
The radius $CP$ of circle $\Pi PDA$ is   the effective major radius, $a$, of the orbit.  The distances $SC$ and $CQ$ are both
equal to $e\,a$, where $e$ is the orbit's effective eccentricity. The angle $PQ\Pi$
is  identified with the  mean anomaly, $M$, and increases {\em linearly}\/  in time.  In other words, as seen from $Q$, the planet
$P$ moves {\em uniformly}\/ around circle $\Pi PDA$ in a counterclockwise direction. Finally, $SP$ is the radial
distance, $r$, of the planet from the sun, and angle $P S \Pi$ is the planet's true anomaly, $T$.

\begin{figure}[h]
\epsfysize=3in
\centerline{\epsffile{epsfiles/ptol1.eps}}
\caption{\em A Ptolemaic orbit.}\label{ptol}
\end{figure}

Let us draw the straight-line $KSL$ parallel to $QP$, and passing through point $S$, and then complete the rectangle $PQKL$. Simple geometry reveals that 
$QK = PL = 2\,e\,a\,\sin M$, $KS=2\,e\,a\,\cos M$, and  $SL = \rho-2\,e\,a\,\cos M$, where
$\rho=QP$. The cosine rule applied to triangle $CQP$ yields
$CP^2 = CQ^2+QP^2-2\,CQ\,QP\,\cos M$, or $\rho^2-2\,e\,a\,\cos M\,\rho -a^2\,(1-e^2)=0$, which
can be solved to give $\rho/a = e\,\cos M +(1-e^2\,\sin^2 M)^{1/2}$. 
 Moreover, $SP^2 = SL^2+ PL^2$,
which implies that
\begin{equation}\label{e4.30x}
\frac{r}{a} = [1-2\,e\,\cos M\,(1-e^2\,\sin^2 M)^{1/2}+e^2+2\,e^2\,\sin^2 M]^{1/2}.
\end{equation}
Now, $T = M + q$, where $q$ is angle $PSL$. However,
\begin{equation}\label{e4.31x}
\sin q = \frac{PL}{SP} = \frac{2\,e\,\sin M}{[1-2\,e\,\cos M\,(1-e^2\,\sin^2 M)^{1/2}+e^2+2\,e^2\,\sin^2 M]^{1/2}}.
\end{equation}

Finally, expanding the previous two equations to second-order in the small parameter $e$, we obtain
\begin{eqnarray}
\frac{r}{a} &=& 1 -e\,\cos M + (3/2)\,e^2\,\sin^2 M,\label{e4.30}\\[0.5ex]
T &=& M + 2\,e\,\sin M + e^2\,\sin 2M.\label{e4.31}
\end{eqnarray}
It can be seen, by comparison with Eqs.~(\ref{je22})--(\ref{je23}) and (\ref{e4.26})--(\ref{e4.27}), that   Ptolemy's geometric model of a heliocentric planetary orbit is significantly
more accurate than  Hipparchus' model, since the 
relative radial distance, $r/a$,  and the true anomaly, $T$, in the former model both only deviate from those in the (correct) Keplerian model to {\em second-order}\/ in $e$.

\section{Model of Copernicus}
Copernicus' geometric model of  a heliocentric planetary orbit is illustrated  in Fig.~\ref{cop}. 
The planet $P$ rotates on a circular epicycle $YP$ whose center $X$ moves around the sun on the eccentric circle $\Pi X D A$ (only
half of which is shown). The diameter $\Pi S C A$ is the effective major axis of the orbit, where $C$ is the geometric center of circle $\Pi X D A$,  and $S$ the fixed position of the sun. When $X$ is at  $\Pi$ or $A$ the planet is at its perihelion
or aphelion points, respectively.  The radius $CX$ of circle $\Pi XDA$ is   the effective major radius, $a$, of the orbit.  The distance $SC$ is
equal to $(3/2)\,e\,a$, where $e$ is the orbit's effective eccentricity. Moreover, the radius $XP$ of the epicycle is equal to $(1/2)\,e\,a$. 
The angle $XC\Pi$
is  identified with the  mean anomaly, $M$, and increases {\em linearly}\/  in time.  In other words, as seen from $C$, the center of
the epicycle
$X$ moves {\em uniformly}\/ around circle $\Pi XDA$ in a counterclockwise direction. The angle $PXY$, where $Y$ is
point at which $CX$ produced meets the epicycle,  is equal to
the mean anomaly $M$. In other words, the planet $P$ moves {\em uniformly}\/ around the epicycle $YP$, in an counterclockwise direction, at {\em twice}\/
the speed that point $X$ moves around circle $\Pi XDA$.
 Finally, $SP$ is the radial
distance, $r$, of the planet from the sun, and angle $P S \Pi$ is the planet's true anomaly, $T$.

\begin{figure}[h]
\epsfysize=3in
\centerline{\epsffile{epsfiles/cop1.eps}}
\caption{\em A Copernican orbit.}\label{cop}
\end{figure}

Let us draw the straight-line $KSL$ parallel to $CX$, and passing through point $S$, and then complete the rectangle $XCKL$. Simple geometry reveals that 
$CK = XL = (3/2)\,e\,a\,\sin M$, $KS=(3/2)\,e\,a\,\cos M$, and  $SL = a-(3/2)\,e\,a\,\cos M$. Let $PZ$ be drawn normal
to $XY$, and let it meet $KSL$ produced at point $W$. Simple geometry reveals that $ZW=XL$,  $ZP=(1/2)\,e\,a\,\sin M$, and $XZ =LW = (1/2)\,e\,a\,\cos M$. It follows that $WP = ZW+ZP = XL+ZP = 2\,e\,a\,\sin M$, and $SW = SL+LW=SL+XZ= a-e\,a\,\cos\,M$.
Moreover, $SP^2 = SW^2+ WP^2$,
which implies that
\begin{equation}
\frac{r}{a} = (1-2\,e\,\cos M +e^2+3\,e^2\,\sin^2 M)^{1/2}.
\end{equation}
Now, $T = M + q$, where $q$ is angle $PSW$. However,
\begin{equation}
\sin q = \frac{WP}{SP} = \frac{2\,e\,\sin M}{(1-2\,e\,\cos M +e^2+3\,e^2\,\sin^2 M)^{1/2}}.
\end{equation}

Finally, expanding the previous two equations to second-order in the small parameter $e$, we obtain
\begin{eqnarray}
\frac{r}{a} &=& 1 -e\,\cos M + 2\,e^2\,\sin^2 M,\\[0.5ex]
T &=& M + 2\,e\,\sin M + e^2\,\sin 2M.
\end{eqnarray}
It can be seen, by comparison with Eqs.~(\ref{je22})--(\ref{je23}) and (\ref{e4.30})--(\ref{e4.31}), that, as is the case for Ptolemy's  model, both the 
relative radial distance, $r/a$,  and the true anomaly, $T$, in Copernicus' geometric model of a heliocentric planetary orbit only deviate from those in the (correct) Keplerian model to {\em second-order}\/ in $e$. However, the deviation  in the Ptolemaic
model is {\em slightly smaller}\/ than that in the Copernican model. To be more exact, the maximum deviation in $r/a$
is $(1/2)\,e^2$ in the former model, and $e^2$ in the latter. On the other hand, the maximum deviation in $T$ is $(1/4)\,e^2$
in both models.