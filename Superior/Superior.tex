\chapter{The Superior Planets}\label{csup}
\section{Determination of Ecliptic Longitude}
Figure~\ref{vf7} compares and contrasts heliocentric and geocentric
models of the
motion  of a superior planet ({\em i.e.}, a planet which is further from the
sun than the earth), $P$,  as seen from the earth, $G$. The sun is
at $S$. In the heliocentric
model, we can write the earth-planet displacement vector, ${\bf P}$,
as the sum of the earth-sun displacement vector, ${\bf S}$, and
the sun-planet displacement vector, ${\bf P}'$.  The geocentric model,
which is entirely equivalent to the heliocentric model as far as
the {\em relative motion}\/ of the planet with respect to the
earth is concerned, and is much more convenient,  relies on the simple vector identity
\begin{equation}\label{ve2.15}
{\bf P} = {\bf S}+{\bf P}' \equiv{\bf P}'+{\bf S}.
\end{equation}
In other words, we can get from the earth to the planet by one of two
different routes. The first route corresponds to the heliocentric model, and
the second to the geocentric model.
In the latter model, ${\bf P}'$ gives the displacement of
the so-called {\em guide-point}, $G'$, from the earth. 
Since ${\bf P}'$ is also the displacement of the planet, $P$, from the
sun, $S$, it is clear that  $G'$ executes a
Keplerian orbit about the earth whose  elements are the
same as those of the orbit of the planet about the sun.
 The ellipse traced out by $G'$ is termed the {\em deferent}. The vector ${\bf S}$ gives
the displacement of the planet from the guide-point.
However, ${\bf S}$ is also the displacement of the sun from the earth.
Hence, it is clear that the planet, $P$,
executes a Keplerian orbit about the guide-point, $G'$, whose
 elements are the same as the sun's apparent orbit about the earth.
 The ellipse traced out by $P$ about $G'$ is termed the {\em epicycle}. 
\begin{figure}[h]
\epsfysize=2.75in
\centerline{\epsffile{epsfiles/figv7.eps}}
\caption[\em Heliocentric and geocentric models of the motion of a superior planet. ]{\em Heliocentric and geocentric models of the motion of a superior planet. Here, $S$ is the sun, $G$ the earth, and $P$ the planet. View is from the northern ecliptic pole.}\label{vf7}   
\end{figure}
 
\begin{figure}
\epsfysize=3in
\centerline{\epsffile{epsfiles/figv4.eps}}
\caption[\em Planetary longitude model.]{\em Planetary longitude model. View is from northern ecliptic pole. }\label{vf4}
\end{figure}

Figure~\ref{vf4} illustrates in more detail how the deferent-epicycle model
is used to determine the ecliptic longitude of a superior planet.
The planet $P$  orbits (counterclockwise) on a small Keplerian orbit $\Pi'PA'$
about  guide-point $G'$, which, in turn, orbits the earth, $G$, (counterclockwise) on a large
Keplerian orbit $\Pi G'A$. As has already been mentioned, the small orbit is termed the epicycle, and the large orbit  the deferent. Both orbits are assumed to lie in the plane of the ecliptic. This approximation does not
introduce a large error into our calculations because the orbital inclinations  of the visible planets to
the ecliptic plane are all fairly small.
 Let $C$, $A$, $\Pi$, $a$, $e$, $\varpi$,
and $T$ denote the geometric center, apocenter ({\em i.e.}, the point of
furthest distance from the central object), pericenter ({\em i.e.}, the
point of closest approach to the central object), major radius, eccentricity, longitude
of the pericenter, and true anomaly of the deferent, respectively. Let $C'$, $A'$, $\Pi'$, $a'$, $e'$, $\varpi'$, and $T'$ denote the corresponding
quantities for the epicycle.

\begin{figure}[h]
\epsfysize=2.5in
\centerline{\epsffile{epsfiles/figv5.eps}}
\caption{\em The triangle $GBP$.}\label{vf5xx}
\end{figure}

Let the line $GG'$ be produced, and let the perpendicular $PB$ be
dropped to it from $P$, as shown in Fig.~\ref{vf5xx}. The angle $\mu\equiv PG'B$
is termed the {\em epicyclic anomaly}\/ (see Fig.~\ref{vf5x}), and takes the form
\begin{equation}
\mu =T'+\varpi' - T - \varpi = \bar{\lambda}' + q' - \bar{\lambda}-q,
\end{equation}
where $\bar{\lambda}$ and $q$ are the mean longitude and equation of center for the
deferent, whereas $\bar{\lambda}'$ and $q'$ are the corresponding quantities for  the
epicycle---see Cha.~\ref{csun}. The epicyclic anomaly is generally written in the 
range $0^\circ$ to $360^\circ$. 
 The angle $\theta\equiv PGG'$ is termed the {\em equation of the epicycle}, and is usually written in the range $-180^\circ$ to $+180^\circ$.
It is clear from the figure that
\begin{equation}
\tan\theta = \frac{\sin\mu}{r/r'+ \cos\mu},
\end{equation}
where $r\equiv GG'$ and $r'\equiv G'P$ are the radial polar coordinates for the deferent and epicycle,
respectively. 
Moreover, according to Equation~(\ref{je22}), $r/r' = (a/a')\,z$, where
\begin{equation}\label{ezdef}
z = \frac{1-\zeta}{1-\zeta'},
\end{equation}
and
\begin{eqnarray}
\zeta &=& e\,\cos\,M -e^{\,2}\,\sin^2\,M,\\[0.5ex]
\zeta' &=& e'\cos\,M' -e'^{\,2}\sin^2\,M'
\end{eqnarray}
are termed {\em radial anomalies}. 
Finally, the ecliptic longitude of the planet is given by  (see Fig.~\ref{vf5x})
\begin{equation}
\lambda = \bar{\lambda} + q + \theta.
\end{equation}

Now, 
\begin{equation}
\theta(\mu,z) \equiv \tan^{-1}\left[ \frac{\sin\mu}{(a/a')\,z+ \cos\mu}\right]
\end{equation}
is a function of two variables, $\mu$ and $z$. It is impractical to tabulate
such a function directly. Fortunately, whilst $\theta(\mu,z)$
has a strong dependence on $\mu$, it only has a fairly weak dependence on $z$.
In fact, it is easily seen that $z$ varies between $z_{\rm min} = \bar{z} - \delta z$
and $z_{\rm max} = \bar{z}+\delta z$, where
\begin{eqnarray}
\bar{z} = \frac{1+e\,e'}{1-e'^{\,2}},\\[0.5ex]
\delta z = \frac{e+e'}{1-e'^{\,2}}.
\end{eqnarray}
Let us define
\begin{equation}
\xi = \frac{\bar{z}-z}{\delta z}.
\end{equation}
This variable takes the value $-1$ when $z=z_{\rm max}$, the value $0$ when
$z=\bar{z}$, and the value $+1$ when $z=z_{\rm min}$. 
Thus, using quadratic interpolation, we can write
\begin{equation}
\theta(\mu,z)\simeq \Theta_-(\xi)\,\delta\theta_-(\mu) + \bar{\theta}(\mu)
+ \Theta_+(\xi)\,\delta\theta_+(\mu),
\end{equation}
where
\begin{eqnarray}
\bar{\theta}(\mu)  &=& \theta(\mu,\bar{z}),\\[0.5ex]
\delta\theta_-(\mu) &=& \theta(\mu,\bar{z}) - \theta(\mu,z_{\rm max}),\\[0.5ex]
\delta\theta_+(\mu) &=& \theta(\mu,z_{\rm min}) - \theta(\mu,\bar{z}),
\end{eqnarray}
and
\begin{eqnarray}\label{ve52}
\Theta_-(\xi) &=& - (1/2)\,\xi\,(\xi-1),\\[0.5ex]
\Theta_+(\xi) &=&+ (1/2)\,\xi\,(\xi+1).\label{ve53}
\end{eqnarray}
This scheme allows us to avoid having to tabulate a
two-dimensional function, whilst ensuring that the exact value of $\theta(\mu,z)$ is obtained
when $z=\bar{z}$, $z_{\rm min}$, or $z_{\rm max}$. The above interpolation
scheme is very similar to that adopted by Ptolemy in the Almagest.

Our procedure for determining the ecliptic longitude of a superior planet is described below. It is assumed that the ecliptic longitude, $\lambda_S$, and the
radial anomaly, $\zeta_S$, of the sun have already been calculated. The latter quantity is tabulated as a function of the solar mean anomaly
in Table~\ref{lt6}. In the following, $a$, $e$, $n$, $\tilde{n}$, $\bar{\lambda}_0$, and $M_0$ represent  elements of the orbit of the planet in question
about the sun, and $e_S$  represents the eccentricity of the sun's apparent orbit
about the earth. (In general, the subscript $S$ denotes the sun.) In particular, $a$ is the major radius of the planetary orbit in
units in which the major radius of the sun's apparent orbit about the
earth is {\em unity}. The requisite elements for all of the superior planets at the J2000 epoch ($t_0=2\,451\,545.0$ JD)
are listed in Table~\ref{lt4}. The ecliptic longitude of a superior
planet is specified by the following formulae:
\begin{eqnarray}
\bar{\lambda}&=&  \bar{\lambda}_0 + n\,(t-t_0) ,\\[0.5ex]
M &=& M_0  +\tilde{n}\,(t-t_0),\\[0.5ex]
q&=& 2\,e\,\sin \,M + (5/4)\,e^2\,\sin\,2M,\label{ve3.8}\\[0.5ex]
\zeta &=& e\,\cos M - e^2\,\sin^2 M,\\[0.5ex]
\mu&=& \lambda_S - \bar{\lambda}-q,\\[0.5ex]
\bar{\theta} &=& \theta(\mu,\bar{z})\equiv \tan^{-1} \left(\frac{\sin\mu}{a\,\bar{z}+\cos\mu}\right),\\[0.5ex]
\delta\theta_- &=& \theta(\mu,\bar{z}) - \theta(\mu,z_{\rm max}),\\[0.5ex]
\delta\theta_+&=& \theta(\mu,z_{\rm min}) - \theta(\mu,\bar{z}),\\[0.5ex]
z &=& \frac{1-\zeta}{1-\zeta_S},\\[0.5ex]
\xi &=& \frac{\bar{z}-z}{\delta z},\\[0.5ex]
\theta  &=&\Theta_-(\xi)\,\delta\theta_-+ \bar{\theta}
+ \Theta_+(\xi)\,\delta\theta_+,\\[0.5ex]
\lambda &=&\bar{\lambda} + q+ \theta.
\end{eqnarray}
Here, $\bar{z} = (1+e\,e_S)/(1-e_S^{\,2})$, $\delta z = (e+e_S)/(1-e_S^{\,2})$, $z_{\rm min} = \bar{z}-\delta z$,
and $z_{\rm max} = \bar{z}+\delta z$.  The constants $\bar{z}$, $\delta z$, $z_{\rm min}$, and $z_{\rm max}$ 
for each of the superior planets are listed in Table~\ref{vtx}.  Finally, the functions $\Theta_\pm$ are  tabulated in Table~\ref{vty}.

For the case of Mars, the above formulae are capable of matching NASA ephemeris data during the years 1995--2006 CE
with a mean error of $3'$ and a maximum error of $14'$. For the case of Jupiter, the mean error is
$1.6'$ and the maximum error $4'$. Finally, for the case of Saturn, the mean error is $0.5'$ and the
maximum error $1'$. 

\section{Mars}
The ecliptic longitude of Mars can be determined with the aid of Tables~\ref{vt8}--\ref{vt10}. Table~\ref{vt8} allows
the mean longitude, $\bar{\lambda}$, and the mean anomaly, $M$, of Mars to be calculated as functions of
time. Next, Table~\ref{vt9} permits the equation of center, $q$, and the radial anomaly, $\zeta$, to
be determined as functions of the mean anomaly. Finally, Table~\ref{vt10} allows the quantities
$\delta\theta_-$, $\bar{\theta}$, and $\delta\theta_+$ to be calculated as functions of the epicyclic
anomaly, $\mu$. 

The procedure for using the tables is as follows:
\begin{enumerate}
\item Determine the fractional Julian day number, $t$, corresponding to the date and time
at which the ecliptic longitude is to be calculated with the aid of Tables~\ref{kt1}--\ref{kt3}. Form $\Delta t = t-t_0$, where $t_0=2\,451\,545.0$ is the epoch. 
\item Calculate the ecliptic longitude, $\lambda_S$, and radial anomaly,
$\zeta_S$, of the sun using the procedure set out in Sect.~\ref{ssun}.
\item Enter Table~\ref{vt8} with the digit for each power of 10
in ${\Delta} t$ and take out the corresponding values of $\Delta\bar{\lambda}$ and $\Delta M$. If $\Delta t$ is negative then the corresponding
values are also negative.
The value of the mean longitude, $\bar{\lambda}$, is the
sum of all the $\Delta\bar{\lambda}$ values plus the value of $\bar{\lambda}$ at the epoch. Likewise, the value of the mean anomaly, $M$, is
the sum of all the $\Delta M$ values plus the value of $M$ at the epoch. 
Add as many multiples of $360^\circ$ to $\bar{\lambda}$ and $M$
as is required to make them both fall in the range $0^\circ$ to $360^\circ$. Round $M$ to the nearest degree. 
\item Enter Table~\ref{vt9} with the value of $M$ and take out the
corresponding value of the equation of center, $q$, and the radial anomaly, $\zeta$. It is necessary to interpolate if $M$ is odd.
\item Form the epicyclic anomaly, $\mu = \lambda_S-\bar{\lambda}-q$. Add as many multiples of $360^\circ$ to $\mu$ as is required to make it fall in the range $0^\circ$ to $360^\circ$. Round $\mu$ to the nearest degree.
\item Enter Table~\ref{vt10} with the value of $\mu$ and take
out the corresponding values of $\delta\theta_-$, $\bar{\theta}$, and
$\delta\theta_+$. If $\mu > 180^\circ$ then it is necessary to make use
of the identities $\delta\theta_\pm(360^\circ - \mu) =-\delta\theta_\pm(\mu)$
and $\bar{\theta}(360^\circ - \mu) =-\bar{\theta}(\mu)$.
\item Form $z = (1-\zeta)/(1-\zeta_S)$.
\item Obtain the values of $\bar{z}$ and $\delta z$ from Table~\ref{vtx}.
Form $\xi = (\bar{z}-z)/\delta z$.
\item Enter Table~\ref{vty} with the value of $\xi$ and take out
the corresponding values of $\Theta_-$ and $\Theta_+$. If $\xi<0$ then
it is necessary to use the identities $\Theta_+(\xi)=-\Theta_-(-\xi)$
and $\Theta_-(\xi)=-\Theta_+(-\xi)$. 
\item Form the equation of the epicycle, $\theta = \Theta_-\,\delta\theta_-+ \bar{\theta}
+ \Theta_+\,\delta\theta_+$.
\item The ecliptic longitude, $\lambda$, is the sum of the mean longitude, $\bar{\lambda}$,  the equation of center, $q$, and the equation
of the epicycle, $\theta$. If necessary convert $\lambda$
into  an angle in the range $0^\circ$ to $360^\circ$. The decimal fraction can
be converted into arc minutes
using Table~\ref{lt6a}. Round to the nearest arc minute. The final result
can be written in terms of the signs of the zodiac using the table in Sect.~\ref{szod}.
\end{enumerate}
Two examples of this procedure are given below.

~\\
\noindent {\em Example 1}: May 5, 2005 CE, 00:00 UT:\\
~\\
From Sect.~\ref{ssun}, $t-t_0=1\,950.5$ JD, $\lambda_S= 44.602^\circ$, $M_S\simeq 120^\circ$. Hence, it follows from Table~\ref{lt6} that
$\zeta_S(M_S)= -8.56\times 10^{-3}$. Making use of
Table~\ref{vt8}, we find:\\
\begin{tabular}{rrr}
&&\\
$t$(JD) & $ \bar{\lambda}(^\circ)$ & $M(^\circ)$\\[-2ex]
&&\\
+1000 & $164.071$ & $164.021$\\
+900 & $111.664$ & $111.619$\\
+50 & $26.204$ & $26.201$\\
+.5 & $0.262$ & $0.262$\\
Epoch & $355.460$ & $19.388$\\\cline{2-3}
&$657.661$ & $321.491$\\\cline{2-3}
Modulus & $297.661$ & $321.491$\\ 
&&\\
\end{tabular}\\
Given that $M\simeq 321^\circ$, Table~\ref{vt9} yields 
$$
q(321^\circ)= -7.345^\circ,\mbox{\hspace{0.5cm}}\zeta(321^\circ)=6.912\times 10^{-2}.
$$
Thus, 
$$
\mu=\lambda_S - \bar{\lambda}-q = 44.602-297.661 + 7.345= 114.286\simeq
114^\circ,
$$
where we have rounded the epicylic anomaly to the nearest degree. It follows from Table~\ref{vt10}
that 
$$
\delta\theta_-(114^\circ) = 3.853^\circ,\mbox{\hspace{0.5cm}}\bar{\theta}(114^\circ)=39.209^\circ, \mbox{\hspace{0.5cm}}\delta\theta_+(114^\circ) = 4.612^\circ.
$$
Now,
$$ 
z= (1-\zeta)/(1-\zeta_S) = (1-6.912\times 10^{-2})/(1+8.56\times 10^{-3}) =
0.9230.
$$
However, from Table~\ref{vtx}, $\bar{z}= 1.00184$ and $\delta z = 0.11014$,
so
$$
\xi = (\bar{z}-z)/\delta z = (1.00184-0.9230)/0.11014 \simeq 0.72.
$$
According to Table~\ref{vty}, 
$$
\Theta_-(0.72) = 0.101, \mbox{\hspace{0.5cm}}\Theta_+(0.72) = 0.619,
$$
so
$$
\theta  = \Theta_-\,\delta\theta_- + \bar{\theta}+\Theta_+\,\delta\theta_+ = 0.101\times 3.853+39.209+0.619\times 4.612 = 42.453^\circ.
$$ 
Finally,
$$
\lambda=\bar{\lambda} + q + \theta= 297.661-7.345+42.453=332.769 \simeq 332^\circ 46'.
$$
Thus,
the ecliptic longitude of Mars at 00:00 UT on May 5, 2005 CE was 2PI46.

~\\
\noindent {\em Example 2}: December 25, 1800 CE, 00:00 UT:\\
~\\
From Sect.~\ref{ssun}, $t-t_0=-72,690.5$ JD, $\lambda_S= 273.055^\circ$, $M_S\simeq 354^\circ$. Hence, it follows from Table~\ref{lt6} that
$\zeta_S(M_S)= 1.662\times 10^{-2}$. Making use of
Table~\ref{vt8}, we find:\\
\begin{tabular}{rrr}
&&\\
$t$(JD) & $\bar{\lambda}(^\circ)$ & $ M(^\circ)$\\[-2ex]
&&\\
-70,000 & $-324.983$ & $-321.453$\\
-2,000 & $-328.142$ & $-328.042$\\
-600 & $-314.443$ & $-314.412$\\
-90 & $-47.166$ & $-47.162$\\
-.5 & $-0.262$ & $-0.262$\\
Epoch & $355.460$ & $19.388$\\\cline{2-3}
&$-659.536$ & $-991.943$\\\cline{2-3}
Modulus & $60.464$ & $88.057$\\
&&\\
\end{tabular}\\
Given that $M\simeq 88^\circ$, Table~\ref{vt9} yields 
$$
q(88^\circ)= 10.739^\circ, \mbox{\hspace{0.5cm}}\zeta(88^\circ)=-5.45\times 10^{-3},
$$
so
$$
\mu=\lambda_S - \bar{\lambda}-q = 273.055-60.464-10.739= 201.852\simeq202^\circ.
$$
It follows from Table~\ref{vt10}
that 
$$
\delta\theta_-(202^\circ) = -5.980^\circ,\mbox{\hspace{0.5cm}}\bar{\theta}(202^\circ)=-32.007^\circ,\mbox{\hspace{0.5cm}}\delta\theta_+(202^\circ) = -8.955^\circ.
$$
Now, 
$$
z= (1-\zeta)/(1-\zeta_S) = (1+5.45\times 10^{-3})/(1-1.662\times 10^{-2}) =
1.02244,
$$
so
$$
\xi = (\bar{z}-z)/\delta z = (1.00184-1.02244)/0.11014 \simeq -0.19.
$$ 
According to Table~\ref{vty}, 
$$
\Theta_-(-0.19) = -0.113, \mbox{\hspace{0.5cm}}
\Theta_+(-0.19) = -0.077,
$$
so
$$
\theta  = \Theta_-\,\delta\theta_- + \bar{\theta}+\Theta_+\,\delta\theta_+ = -0.113\times 5.980-32.007-0.077\times 8.955 = -30.642^\circ.
$$
Finally,
$$
\lambda=\bar{\lambda} + q + \theta= 60.464+10.739-30.642 = 40.561 \simeq 40^\circ 34'.
$$ 
Thus,
the ecliptic longitude of Mars at 00:00 UT on December 25, 1800 CE was 10TA34.

\begin{figure}[h]
\epsfysize=3in
\centerline{\epsffile{epsfiles/epicycle.eps}}
\caption[\em The geocentric orbit of a superior planet.]{\em The geocentric orbit of a superior planet. Here, $G$, $G'$, $P$,  $\mu$,  $\theta$, $\bar{\lambda}$, $q$, and $\Upsilon$ represent
the earth, guide-point, planet,  epicyclic anomaly, equation of the epicycle, mean longitude, equation of center, and spring equinox, respectively. View is
from northern ecliptic pole. Both $G'$ and $P$ orbit counterclockwise.}\label{vf5x}
\end{figure}

\section{Determination of Conjunction, Opposition, and Station Dates}
Figure~\ref{vf5x} shows the geocentric orbit of a superior planet. Recall that the vector $G'P$ is always {\em parallel}\/ to the vector connecting the earth to the sun. It follows that
a so-called {\em conjunction}, at which the sun lies
directly between the planet and the earth, occurs whenever the epicyclic anomaly, $\mu$, takes the value $0^\circ$. At
a conjunction, the planet is furthest from the earth, and has the same ecliptic longitude as the
sun, and is, therefore, invisible. Conversely, a so-called {\em opposition}, at which the earth lies directly between the planet and the sun, occurs whenever $\mu=180^\circ$. At an opposition, the planet is closest to the earth, and also directly opposite the 
sun in the sky, and, therefore, at its brightest. Now, a superior planet rotates around the epicycle at a
faster angular velocity than its guide-point rotates around the deferent. Moreover, both the planet and guide-point rotate
in the same direction. It follows that the planet is traveling {\em backward}\/ in the sky (relative to the direction
of its mean motion) at opposition. This phenomenon is called {\em retrograde motion}. The period of
retrograde motion begins and ends at  {\em stations}---so-called because when the planet reaches them
it appears to stand still in the sky for a few days whilst it reverses direction.

Tables~\ref{vt8}--\ref{vt10} can be used to determine the dates of the conjunctions, oppositions, and
stations of Mars. Consider the first conjunction after the epoch (January 1, 2000 CE). We can estimate the
time at which this event occurs by approximating the epicyclic anomaly as the so-called
{\em mean epicyclic anomaly}: 
$$
\mu \simeq \bar{\mu} = \bar{\lambda}_S-\bar{\lambda} = \bar{\lambda}_{0\,S}-\bar{\lambda}_0 +(n_S-n)\,(t-t_0) = 284.998 + 0.46157617\,(t-t_0).
$$
 We obtain
$$
t\simeq t_0 + (360-284.998)/0.46157617\simeq t_0 + 162\, {\rm JD}.
$$ 
A calculation of the epicyclic anomaly at this time, using Tables~\ref{vt8}--\ref{vt10}, yields $\mu=-9.583^\circ$. Now, the
actual conjunction occurs when $\mu=0^\circ$. Hence, our next estimate is 
$$
t\simeq t_0+162+9.583/0.46157617\simeq t_0 + 183\,{\rm JD}.
$$
A
calculation of the epicyclic anomaly at this time gives $0.294^\circ$. Thus, our final estimate is
$$
t=t_0 +183-0.294/0.461557617=t_0+182.4\, {\rm JD},
$$
which corresponds to July 1, 2000 CE.

Consider the first opposition of Mars after the epoch. Our first estimate of the time at which this
event takes place is 
$$
t\simeq t_0+(540-284.998)/0.46157617\simeq t_0 + 552\, {\rm JD}.
$$
A calculation of
the epicyclic anomaly at this time yields $\mu=188.649^\circ$. Now, the actual
opposition occurs when $\mu=180^\circ$. 
Hence, our second estimate is
$$
t\simeq t_0 +552-8.649/0.46157617\simeq t_0+533\,{\rm JD}.
$$
A calculation of the epicyclic anomaly
at this time  gives $181.455^\circ$. Thus, our third estimate is 
$$
t\simeq t_0+533-1.455/0.46157617 \simeq t_0+ 530\,{\rm JD}.
$$ A calculation of the epicyclic anomaly at this time yields $180.244^\circ$.
Hence, our final estimate is 
$$
t=t_0+530-0.244/0.46157617=t_0+529.5\,{\rm JD},
$$ 
which corresponds to
June 13, 2001 CE. Incidentally, it is clear from the above analysis that the
{\em mean}\/ time period between successive conjunctions,  or  oppositions, of Mars is $360/0.46157617= 779.9$ JD, which is
equivalent to $2.14$ years.

Let us now consider the stations of Mars. We can approximate the ecliptic longitude of a superior
planet
as
\begin{equation}
\lambda \simeq \bar{\lambda} + \bar{\theta},
\end{equation}
where
\begin{equation}
\bar{\theta} = \tan^{-1}\left(\frac{\sin \bar{\mu}}{\bar{a}+\cos\bar{\mu}}\right),
\end{equation}
and $\bar{a} = a\,\bar{z}$. 
Note that $d\bar{\lambda}/dt= n$ and $d\bar{\mu}/dt = n_S-n$. It follows that
\begin{equation}
\frac{d\lambda}{dt} \simeq n + \left(\frac{\bar{a}\, \cos\bar{\mu}+1}{1+2\,\bar{a}\,\cos\bar{\mu}+\bar{a}^2}\right) (n_S-n).
\end{equation}
Now, a station corresponds to $d\lambda/dt = 0$ ({\em i.e.}, a local maximum or minimum of $\lambda$), which gives
\begin{equation}
\cos\bar{\mu} \simeq - \frac{(\bar{a}^2 + n_S/n)}{\bar{a}\,(1+n_S/n)}.
\end{equation}
For the case of Mars, we find that $\bar{\mu}=163.3^\circ$ or $196.7^\circ$. The first solution corresponds
to the so-called {\em retrograde station}, at which the planet switches from direct to retrograde motion.
The second solution corresponds to the so-called {\em direct station}, at which the planet switches
from retrograde to direct motion. The {\em mean}\/ time interval between a retrograde station and the following
opposition, or between an opposition and the following direct station, is $(180-163.3)/0.46157617\simeq 36$ JD. 
Unfortunately, the only option for accurately determining the  dates at which the stations occur is to calculate
the ecliptic longitude of Mars over a range of days centered 36 days  before and after its opposition.

Table~\ref{vtmars} shows the conjunctions, oppositions, and
stations of Mars for the years 2000--2020 CE, calculated using the
techniques described above.

\section{Jupiter}
The ecliptic longitude of Jupiter can be determined with the aid of Tables~\ref{vt11}--\ref{vt13}. Table~\ref{vt11} allows
the mean longitude, $\bar{\lambda}$, and the mean anomaly, $M$, of Jupiter to be calculated as functions of
time. Next, Table~\ref{vt12} permits the equation of center, $q$, and the radial anomaly, $\zeta$, to
be determined as functions of the mean anomaly. Finally, Table~\ref{vt13} allows the quantities
$\delta\theta_-$, $\bar{\theta}$, and $\delta\theta_+$ to be calculated as functions of the epicyclic
anomaly, $\mu$. 
The procedure for using the tables is analogous to the previously described procedure for
using the Mars tables.
One example of this procedure is given below.

~\\
\noindent {\em Example}: May 5,  2005 CE, 00:00 UT:\\
~\\
From before, $t-t_0=1\,950.5$ JD, $\lambda_S= 44.602^\circ$, $M_S\simeq 120^\circ$, and $\zeta_S= -8.56\times 10^{-3}$. Making use of
Table~\ref{vt11}, we find:\\
\begin{tabular}{rrr}
&&\\
$t$(JD) & $ \bar{\lambda}(^\circ)$ & $M(^\circ)$\\[-2ex]
&&\\
+1000 & $83.125$ & $83.081$\\
+900 & $74.813$ & $74.773$\\
+50 & $4.156$ & $4.154$\\
+.5 & $0.042$ & $0.042$\\
Epoch & $34.365$ & $19.348$\\\cline{2-3}
&$196.501$ & $181.398$\\\cline{2-3}
Modulus & $196.501$ & $181.398$\\ 
&&\\
\end{tabular}\\
Given that $M\simeq 181^\circ$, Table~\ref{vt12} yields 
$$
q(181^\circ)= -0.091^\circ,\mbox{\hspace{0.5cm}}\zeta(181^\circ)=-4.838\times 10^{-2}.
$$
Thus, 
$$
\mu=\lambda_S - \bar{\lambda}-q = 44.602-196.501 + 0.091= -151.808\simeq
208^\circ,
$$
where we have rounded the epicylic anomaly to the nearest degree. It follows from Table~\ref{vt13}
that 
$$
\delta\theta_-(208^\circ) = -0.447^\circ,\mbox{\hspace{0.5cm}}\bar{\theta}(208^\circ)=-6.194^\circ, \mbox{\hspace{0.5cm}}\delta\theta_+(208^\circ) = -0.522^\circ.
$$
Now,
$$ 
z= (1-\zeta)/(1-\zeta_S) = (1+4.838\times 10^{-2})/(1+8.56\times 10^{-3}) =
1.0395.
$$
However, from Table~\ref{vtx}, $\bar{z}= 1.00109$ and $\delta z = 0.06512$,
so
$$
\xi = (\bar{z}-z)/\delta z = (1.00109-1.0395)/0.06512 \simeq -0.59.
$$
According to Table~\ref{vty}, 
$$
\Theta_-(-0.59) = -0.469, \mbox{\hspace{0.5cm}}\Theta_+(-0.59) = -0.121,
$$
so
$$
\theta  = \Theta_-\,\delta\theta_- + \bar{\theta}+\Theta_+\,\delta\theta_+ = 0.469\times 0.447-6.194+0.121\times 0.522 =- 5.921^\circ.
$$ 
Finally,
$$
\lambda=\bar{\lambda} + q + \theta= 196.501-0.091-5.921=190.489 \simeq 190^\circ 29'.
$$
Thus,
the ecliptic longitude of Jupiter at 00:00 UT on May 5, 2005 CE was 10LI29.

The conjunctions, oppositions, and stations of Jupiter can be investigated
using analogous methods to those employed earlier to examine the
conjunctions, oppositions, and stations of Mars. We find that the mean
time period between successive oppositions or conjunctions of
Jupiter is 1.09 yr. Furthermore, on average, the retrograde and direct
stations of Jupiter occur when the epicyclic anomaly takes the
values $\mu=125.6^\circ$ and $234.4^\circ$, respectively. Finally,
the mean time period between a retrograde station and the following
opposition, or between the opposition and the following direct
station, is 60 JD. The conjunctions, oppositions, and stations of Jupiter
during the years 2000--2010 CE are shown in Table~\ref{vtjupiter}.

\section{Saturn}
The ecliptic longitude of Saturn can be determined with the aid of Tables~\ref{vt14}--\ref{vt16}. Table~\ref{vt14} allows
the mean longitude, $\bar{\lambda}$, and the mean anomaly, $M$, of Saturn to be calculated as functions of
time. Next, Table~\ref{vt15} permits the equation of center, $q$, and the radial anomaly, $\zeta$, to
be determined as functions of the mean anomaly. Finally, Table~\ref{vt16} allows the quantities
$\delta\theta_-$, $\bar{\theta}$, and $\delta\theta_+$ to be calculated as functions of the epicyclic
anomaly, $\mu$. 
The procedure for using the tables is analogous to the previously described procedure for
using the Mars tables.
One example of this procedure is given below.

~\\
\noindent {\em Example}: May 5,  2005 CE, 00:00 UT:\\
~\\
From before, $t-t_0=1\,950.5$ JD, $\lambda_S= 44.602^\circ$, $M_S\simeq 120^\circ$, and
$\zeta_S= -8.56\times 10^{-3}$. Making use of
Table~\ref{vt14}, we find:\\
\begin{tabular}{rrr}
&&\\
$t$(JD) & $ \bar{\lambda}(^\circ)$ & $M(^\circ)$\\[-2ex]
&&\\
+1000 & $33.508$ & $33.482$\\
+900 & $30.157$ & $30.133$\\
+50 & $1.675$ & $1.674$\\
+.5 & $0.017$ & $0.017$\\
Epoch & $50.059$ & $317.857$\\\cline{2-3}
&$115.416$ & $383.163$\\\cline{2-3}
Modulus & $115.416$ & $23.163$\\ 
&&\\
\end{tabular}\\
Given that $M\simeq 23^\circ$, Table~\ref{vt15} yields 
$$
q(23^\circ)= 2.561^\circ,\mbox{\hspace{0.5cm}}\zeta(23^\circ)=4.913\times 10^{-2}.
$$
Thus, 
$$
\mu=\lambda_S - \bar{\lambda}-q = 44.602-115.416 -2.561= -73.375\simeq
287^\circ,
$$
where we have rounded the epicylic anomaly to the nearest degree. It follows from Table~\ref{vt16}
that 
$$
\delta\theta_-(287^\circ) = -0.353^\circ,\mbox{\hspace{0.5cm}}\bar{\theta}(287^\circ)=-5.551^\circ, \mbox{\hspace{0.5cm}}\delta\theta_+(287^\circ) = -0.405^\circ.
$$
Now,
$$ 
z= (1-\zeta)/(1-\zeta_S) = (1-4.913\times 10^{-2})/(1+8.56\times 10^{-3}) =
0.9428.
$$
However, from Table~\ref{vtx}, $\bar{z}= 1.00118$ and $\delta z = 0.07059$,
so
$$
\xi = (\bar{z}-z)/\delta z = (1.00118-0.9428)/0.07059 \simeq 0.83.
$$
According to Table~\ref{vty}, 
$$
\Theta_-(0.83) = 0.071, \mbox{\hspace{0.5cm}}\Theta_+(0.83) = 0.759,
$$
so
$$
\theta  = \Theta_-\,\delta\theta_- + \bar{\theta}+\Theta_+\,\delta\theta_+ = -0.071\times 0.353-5.551-0.759\times 0.405 =- 5.883^\circ.
$$ 
Finally,
$$
\lambda=\bar{\lambda} + q + \theta= 115.416+2.561-5.883=112.094 \simeq 112^\circ 06'.
$$
Thus,
the ecliptic longitude of Saturn at 00:00 UT on May 5, 2005 CE was 22CN06.

The conjunctions, oppositions, and stations of Saturn can be investigated
using analogous methods to those employed earlier to examine the
conjunctions, oppositions, and stations of Mars. We find that the mean
time period between successive oppositions or conjunctions of
Saturn is 1.035 yr. Furthermore, on average, the retrograde and direct
stations of Saturn occur when the epicyclic anomaly takes the
values $\mu=114.5^\circ$ and $245.5^\circ$, respectively. Finally,
the mean time period between a retrograde station and the following
opposition, or between the opposition and the following direct
station, is 69 JD. The conjunctions, oppositions, and stations of Saturn
during the years 2000--2010 CE are shown in Table~\ref{vtsaturn}.

\clearpage
\begin{table}
\centering
\begin{tabular}{l|cccc}
Planet & $\bar{z}$ & $\delta z$ & $z_{\rm min}$ & $z_{\rm max}$\\\hline
&&&&\\[-2ex]
Mercury & 1.04774 & 0.23216 & 0.81558 & 1.27990\\
Venus & 1.00016 & 0.02349 & 0.97667 & 1.02365\\
Mars & 1.00184 & 0.11014 & 0.89170 & 1.11198\\
Jupiter & 1.00109 & 0.06512 & 0.93597 & 1.06602\\
Saturn & 1.00118 & 0.07059 & 0.93059 & 1.07177\\
\end{tabular}
\caption{\em Constants associated with the epicycles of the inferior and superior planets.}\label{vtx}
\end{table}

\begin{table}
\centering
\begin{tabular}{ccc|ccc|ccc|ccc}
$\xi$ & $\Theta_-$ & $\Theta_+$ & $\xi$ & $\Theta_-$ & $\Theta_+$ & $\xi$ & $\Theta_-$ & $\Theta_+$
&$\xi$ & $\Theta_-$ & $\Theta_+$\\\hline
&&&&&&&&&&&\\[-2ex]
  0.00 &   0.000 &   0.000 &   0.25 &   0.094 &   0.156 &   0.50 &   0.125 &   0.375 &   0.75 &   0.094 &   0.656\\
  0.01 &   0.005 &   0.005 &   0.26 &   0.096 &   0.164 &   0.51 &   0.125 &   0.385 &   0.76 &   0.091 &   0.669\\
  0.02 &   0.010 &   0.010 &   0.27 &   0.099 &   0.171 &   0.52 &   0.125 &   0.395 &   0.77 &   0.089 &   0.681\\
  0.03 &   0.015 &   0.015 &   0.28 &   0.101 &   0.179 &   0.53 &   0.125 &   0.405 &   0.78 &   0.086 &   0.694\\
  0.04 &   0.019 &   0.021 &   0.29 &   0.103 &   0.187 &   0.54 &   0.124 &   0.416 &   0.79 &   0.083 &   0.707\\
  0.05 &   0.024 &   0.026 &   0.30 &   0.105 &   0.195 &   0.55 &   0.124 &   0.426 &   0.80 &   0.080 &   0.720\\
  0.06 &   0.028 &   0.032 &   0.31 &   0.107 &   0.203 &   0.56 &   0.123 &   0.437 &   0.81 &   0.077 &   0.733\\
  0.07 &   0.033 &   0.037 &   0.32 &   0.109 &   0.211 &   0.57 &   0.123 &   0.447 &   0.82 &   0.074 &   0.746\\
  0.08 &   0.037 &   0.043 &   0.33 &   0.111 &   0.219 &   0.58 &   0.122 &   0.458 &   0.83 &   0.071 &   0.759\\
  0.09 &   0.041 &   0.049 &   0.34 &   0.112 &   0.228 &   0.59 &   0.121 &   0.469 &   0.84 &   0.067 &   0.773\\
  0.10 &   0.045 &   0.055 &   0.35 &   0.114 &   0.236 &   0.60 &   0.120 &   0.480 &   0.85 &   0.064 &   0.786\\
  0.11 &   0.049 &   0.061 &   0.36 &   0.115 &   0.245 &   0.61 &   0.119 &   0.491 &   0.86 &   0.060 &   0.800\\
  0.12 &   0.053 &   0.067 &   0.37 &   0.117 &   0.253 &   0.62 &   0.118 &   0.502 &   0.87 &   0.057 &   0.813\\
  0.13 &   0.057 &   0.073 &   0.38 &   0.118 &   0.262 &   0.63 &   0.117 &   0.513 &   0.88 &   0.053 &   0.827\\
  0.14 &   0.060 &   0.080 &   0.39 &   0.119 &   0.271 &   0.64 &   0.115 &   0.525 &   0.89 &   0.049 &   0.841\\
  0.15 &   0.064 &   0.086 &   0.40 &   0.120 &   0.280 &   0.65 &   0.114 &   0.536 &   0.90 &   0.045 &   0.855\\
  0.16 &   0.067 &   0.093 &   0.41 &   0.121 &   0.289 &   0.66 &   0.112 &   0.548 &   0.91 &   0.041 &   0.869\\
  0.17 &   0.071 &   0.099 &   0.42 &   0.122 &   0.298 &   0.67 &   0.111 &   0.559 &   0.92 &   0.037 &   0.883\\
  0.18 &   0.074 &   0.106 &   0.43 &   0.123 &   0.307 &   0.68 &   0.109 &   0.571 &   0.93 &   0.033 &   0.897\\
  0.19 &   0.077 &   0.113 &   0.44 &   0.123 &   0.317 &   0.69 &   0.107 &   0.583 &   0.94 &   0.028 &   0.912\\
  0.20 &   0.080 &   0.120 &   0.45 &   0.124 &   0.326 &   0.70 &   0.105 &   0.595 &   0.95 &   0.024 &   0.926\\
  0.21 &   0.083 &   0.127 &   0.46 &   0.124 &   0.336 &   0.71 &   0.103 &   0.607 &   0.96 &   0.019 &   0.941\\
  0.22 &   0.086 &   0.134 &   0.47 &   0.125 &   0.345 &   0.72 &   0.101 &   0.619 &   0.97 &   0.015 &   0.955\\
  0.23 &   0.089 &   0.141 &   0.48 &   0.125 &   0.355 &   0.73 &   0.099 &   0.631 &   0.98 &   0.010 &   0.970\\
  0.24 &   0.091 &   0.149 &   0.49 &   0.125 &   0.365 &   0.74 &   0.096 &   0.644 &   0.99 &   0.005 &   0.985\\
  0.25 &   0.094 &   0.156 &   0.50 &   0.125 &   0.375 &   0.75 &   0.094 &   0.656 &   1.00 &  0.000 &   1.000\\
\end{tabular}
\caption[\em  Epicyclic interpolation coefficients.]{\em Epicyclic interpolation coefficients. Note that $\Theta_\pm (\xi) = - \Theta_\mp (-\xi)$.}\label{vty}
\end{table}

\newpage
\begin{table}
\centering
\begin{tabular}{rrrr|rrrr}
$\Delta t$(JD)& $\Delta\bar{\lambda}(^\circ)$ &  $\Delta M(^\circ)$ & $\Delta \bar{F}(^\circ)$& $\Delta t$(JD) & $\Delta\bar{\lambda}(^\circ)$ & $\Delta M(^\circ)$ 
&$\Delta \bar{F}(^\circ)$\\ \hline
&&&&&&&\\[-1.75ex]
10,000 & 200.712 & 200.208 & 200.409 & 1,000 & 164.071 & 164.021 & 164.041\\
20,000 &  41.424 &  40.415 &  40.819 & 2,000 & 328.142 & 328.042 & 328.082\\
30,000 & 242.135 & 240.623 & 241.228 & 3,000 & 132.214 & 132.062 & 132.123\\
40,000 &  82.847 &  80.830 &  81.638 & 4,000 & 296.285 & 296.083 & 296.164\\
50,000 & 283.559 & 281.038 & 282.047 & 5,000 & 100.356 & 100.104 & 100.205\\
60,000 & 124.271 & 121.246 & 122.456 & 6,000 & 264.427 & 264.125 & 264.246\\
70,000 & 324.983 & 321.453 & 322.866 & 7,000 &  68.498 &  68.145 &  68.287\\
80,000 & 165.694 & 161.661 & 163.275 & 8,000 & 232.569 & 232.166 & 232.328\\
90,000 &   6.406 &   1.868 &   3.685 & 9,000 &  36.641 &  36.187 &  36.368\\
&&&&&&&\\
100 &  52.407 &  52.402 &  52.404 & 10 &   5.241 &   5.240 &   5.240\\
200 & 104.814 & 104.804 & 104.808 & 20 &  10.481 &  10.480 &  10.481\\
300 & 157.221 & 157.206 & 157.212 & 30 &  15.722 &  15.721 &  15.721\\
400 & 209.628 & 209.608 & 209.616 & 40 &  20.963 &  20.961 &  20.962\\
500 & 262.036 & 262.010 & 262.020 & 50 &  26.204 &  26.201 &  26.202\\
600 & 314.443 & 314.412 & 314.425 & 60 &  31.444 &  31.441 &  31.442\\
700 &   6.850 &   6.815 &   6.829 & 70 &  36.685 &  36.681 &  36.683\\
800 &  59.257 &  59.217 &  59.233 & 80 &  41.926 &  41.922 &  41.923\\
900 & 111.664 & 111.619 & 111.637 & 90 &  47.166 &  47.162 &  47.164\\
&&&&&&&\\
1 &   0.524 &   0.524 &   0.524 & 0.1 &   0.052 &   0.052 &   0.052\\
2 &   1.048 &   1.048 &   1.048 & 0.2 &   0.105 &   0.105 &   0.105\\
3 &   1.572 &   1.572 &   1.572 & 0.3 &   0.157 &   0.157 &   0.157\\
4 &   2.096 &   2.096 &   2.096 & 0.4 &   0.210 &   0.210 &   0.210\\
5 &   2.620 &   2.620 &   2.620 & 0.5 &   0.262 &   0.262 &   0.262\\
6 &   3.144 &   3.144 &   3.144 & 0.6 &   0.314 &   0.314 &   0.314\\
7 &   3.668 &   3.668 &   3.668 & 0.7 &   0.367 &   0.367 &   0.367\\
8 &   4.193 &   4.192 &   4.192 & 0.8 &   0.419 &   0.419 &   0.419\\
9 &   4.717 &   4.716 &   4.716 & 0.9 &   0.472 &   0.472 &   0.472\\
\end{tabular}
\caption[\em Mean motion of Mars.]{\em Mean motion of Mars.  Here, $\Delta t = t-t_0$, $\Delta\bar{\lambda} = \bar{\lambda}-\bar{\lambda}_0$, $\Delta M = M - M_0$,
and $\Delta\bar{F} = \bar{F}-\bar{F}_0$.  At epoch  ($t_0 = 2\,451\,545.0$ JD), $\bar{\lambda}_0 = 355.460^\circ$, $M_0 = 19.388^\circ$, and
$\bar{F}_0 =305.796^\circ$. }\label{vt8}
\end{table}

\newpage
\begin{table}\centering
\small{ \begin{tabular}{rrr|rrr|rrr|rrr}
$M(^\circ)$ & $q(^\circ)$  & $100\,\zeta$ & $M(^\circ)$ & $q(^\circ)$  & $100\,\zeta$ & $M(^\circ)$ & $q(^\circ)$  & $100\,\zeta$& $M(^\circ)$ & $q(^\circ)$  & $100\,\zeta$\\\hline
&&&&&&&&&&&\\[-1.75ex]
 0 &   0.000 &  9.339 &  90 &  10.702 & -0.872 & 180 &   0.000 & -9.339 & 270 & -10.702 & -0.872\\
  2 &   0.417 &  9.333 &  92 &  10.652 & -1.197 & 182 &  -0.330 & -9.335 & 272 & -10.739 & -0.545\\
  4 &   0.833 &  9.312 &  94 &  10.589 & -1.519 & 184 &  -0.660 & -9.321 & 274 & -10.763 & -0.217\\
  6 &   1.249 &  9.279 &  96 &  10.514 & -1.839 & 186 &  -0.989 & -9.298 & 276 & -10.773 &  0.114\\
  8 &   1.662 &  9.232 &  98 &  10.426 & -2.155 & 188 &  -1.317 & -9.265 & 278 & -10.770 &  0.444\\
 10 &   2.072 &  9.171 & 100 &  10.326 & -2.468 & 190 &  -1.645 & -9.224 & 280 & -10.753 &  0.776\\
 12 &   2.479 &  9.098 & 102 &  10.214 & -2.776 & 192 &  -1.971 & -9.173 & 282 & -10.722 &  1.107\\
 14 &   2.882 &  9.011 & 104 &  10.091 & -3.081 & 194 &  -2.296 & -9.113 & 284 & -10.678 &  1.438\\
 16 &   3.281 &  8.911 & 106 &   9.957 & -3.380 & 196 &  -2.619 & -9.044 & 286 & -10.619 &  1.768\\
 18 &   3.674 &  8.799 & 108 &   9.811 & -3.675 & 198 &  -2.940 & -8.966 & 288 & -10.546 &  2.097\\
 20 &   4.062 &  8.674 & 110 &   9.655 & -3.964 & 200 &  -3.259 & -8.878 & 290 & -10.458 &  2.424\\
 22 &   4.443 &  8.537 & 112 &   9.489 & -4.248 & 202 &  -3.575 & -8.782 & 292 & -10.357 &  2.749\\
 24 &   4.817 &  8.388 & 114 &   9.313 & -4.527 & 204 &  -3.889 & -8.676 & 294 & -10.241 &  3.071\\
 26 &   5.184 &  8.227 & 116 &   9.127 & -4.799 & 206 &  -4.199 & -8.562 & 296 & -10.111 &  3.389\\
 28 &   5.542 &  8.054 & 118 &   8.932 & -5.065 & 208 &  -4.506 & -8.438 & 298 &  -9.967 &  3.705\\
 30 &   5.892 &  7.870 & 120 &   8.727 & -5.324 & 210 &  -4.810 & -8.306 & 300 &  -9.809 &  4.016\\
 32 &   6.233 &  7.675 & 122 &   8.514 & -5.576 & 212 &  -5.110 & -8.165 & 302 &  -9.637 &  4.322\\
 34 &   6.564 &  7.470 & 124 &   8.293 & -5.822 & 214 &  -5.405 & -8.015 & 304 &  -9.452 &  4.623\\
 36 &   6.885 &  7.254 & 126 &   8.064 & -6.060 & 216 &  -5.696 & -7.857 & 306 &  -9.252 &  4.919\\
 38 &   7.195 &  7.029 & 128 &   7.827 & -6.292 & 218 &  -5.983 & -7.690 & 308 &  -9.040 &  5.208\\
 40 &   7.494 &  6.794 & 130 &   7.583 & -6.515 & 220 &  -6.264 & -7.515 & 310 &  -8.814 &  5.491\\
 42 &   7.782 &  6.550 & 132 &   7.332 & -6.731 & 222 &  -6.540 & -7.331 & 312 &  -8.575 &  5.768\\
 44 &   8.059 &  6.297 & 134 &   7.074 & -6.939 & 224 &  -6.810 & -7.139 & 314 &  -8.323 &  6.036\\
 46 &   8.323 &  6.036 & 136 &   6.810 & -7.139 & 226 &  -7.074 & -6.939 & 316 &  -8.059 &  6.297\\
 48 &   8.575 &  5.768 & 138 &   6.540 & -7.331 & 228 &  -7.332 & -6.731 & 318 &  -7.782 &  6.550\\
 50 &   8.814 &  5.491 & 140 &   6.264 & -7.515 & 230 &  -7.583 & -6.515 & 320 &  -7.494 &  6.794\\
 52 &   9.040 &  5.208 & 142 &   5.983 & -7.690 & 232 &  -7.827 & -6.292 & 322 &  -7.195 &  7.029\\
 54 &   9.252 &  4.919 & 144 &   5.696 & -7.857 & 234 &  -8.064 & -6.060 & 324 &  -6.885 &  7.254\\
 56 &   9.452 &  4.623 & 146 &   5.405 & -8.015 & 236 &  -8.293 & -5.822 & 326 &  -6.564 &  7.470\\
 58 &   9.637 &  4.322 & 148 &   5.110 & -8.165 & 238 &  -8.514 & -5.576 & 328 &  -6.233 &  7.675\\
 60 &   9.809 &  4.016 & 150 &   4.810 & -8.306 & 240 &  -8.727 & -5.324 & 330 &  -5.892 &  7.870\\
 62 &   9.967 &  3.705 & 152 &   4.506 & -8.438 & 242 &  -8.932 & -5.065 & 332 &  -5.542 &  8.054\\
 64 &  10.111 &  3.389 & 154 &   4.199 & -8.562 & 244 &  -9.127 & -4.799 & 334 &  -5.184 &  8.227\\
 66 &  10.241 &  3.071 & 156 &   3.889 & -8.676 & 246 &  -9.313 & -4.527 & 336 &  -4.817 &  8.388\\
 68 &  10.357 &  2.749 & 158 &   3.575 & -8.782 & 248 &  -9.489 & -4.248 & 338 &  -4.443 &  8.537\\
 70 &  10.458 &  2.424 & 160 &   3.259 & -8.878 & 250 &  -9.655 & -3.964 & 340 &  -4.062 &  8.674\\
 72 &  10.546 &  2.097 & 162 &   2.940 & -8.966 & 252 &  -9.811 & -3.675 & 342 &  -3.674 &  8.799\\
 74 &  10.619 &  1.768 & 164 &   2.619 & -9.044 & 254 &  -9.957 & -3.380 & 344 &  -3.281 &  8.911\\
 76 &  10.678 &  1.438 & 166 &   2.296 & -9.113 & 256 & -10.091 & -3.081 & 346 &  -2.882 &  9.011\\
 78 &  10.722 &  1.107 & 168 &   1.971 & -9.173 & 258 & -10.214 & -2.776 & 348 &  -2.479 &  9.098\\
 80 &  10.753 &  0.776 & 170 &   1.645 & -9.224 & 260 & -10.326 & -2.468 & 350 &  -2.072 &  9.171\\
 82 &  10.770 &  0.444 & 172 &   1.317 & -9.265 & 262 & -10.426 & -2.155 & 352 &  -1.662 &  9.232\\
 84 &  10.773 &  0.114 & 174 &   0.989 & -9.298 & 264 & -10.514 & -1.839 & 354 &  -1.249 &  9.279\\
 86 &  10.763 & -0.217 & 176 &   0.660 & -9.321 & 266 & -10.589 & -1.519 & 356 &  -0.833 &  9.312\\
 88 &  10.739 & -0.545 & 178 &   0.330 & -9.335 & 268 & -10.652 & -1.197 & 358 &  -0.417 &  9.333\\
 90 &  10.702 & -0.872 & 180 &   0.000 & -9.339 & 270 & -10.702 & -0.872 & 360 &  -0.000 &  9.339\\ 
\end{tabular}}
\caption{\em Deferential anomalies of Mars.}\label{vt9}
\end{table}

\newpage
\begin{table}\centering
\small{ \begin{tabular}{rrrr|rrrr|rrrr|rrrr}
$\mu$ & $\delta\theta_-$  & $\bar{\theta}~~~~$ & $\delta\theta_+$ &
$\mu$ & $\delta\theta_-$  & $\bar{\theta}~~~~$ & $\delta\theta_+$ &
$\mu$ & $\delta\theta_-$  & $\bar{\theta}~~~~$ & $\delta\theta_+$ &
$\mu$ & $\delta\theta_-$  & $\bar{\theta}~~~~$ & $\delta\theta_+$ \\\hline
&&&&&&&&&&&&&&&\\[-1.75ex]
    0 & \tiny{  0.000} &   0.000 & \tiny{  0.000} &  45 & \tiny{  1.159} &  17.566 & \tiny{  1.329} &  90 & \tiny{  2.679} &  33.228 & \tiny{  3.125} & 135 & \tiny{  5.180} &  40.793 & \tiny{  6.547}\\
  1 & \tiny{  0.025} &   0.396 & \tiny{  0.028} &  46 & \tiny{  1.187} &  17.945 & \tiny{  1.362} &  91 & \tiny{  2.721} &  33.527 & \tiny{  3.176} & 136 & \tiny{  5.246} &  40.716 & \tiny{  6.658}\\
  2 & \tiny{  0.049} &   0.792 & \tiny{  0.056} &  47 & \tiny{  1.216} &  18.322 & \tiny{  1.394} &  92 & \tiny{  2.764} &  33.822 & \tiny{  3.228} & 137 & \tiny{  5.312} &  40.619 & \tiny{  6.771}\\
  3 & \tiny{  0.074} &   1.187 & \tiny{  0.084} &  48 & \tiny{  1.244} &  18.699 & \tiny{  1.427} &  93 & \tiny{  2.807} &  34.114 & \tiny{  3.281} & 138 & \tiny{  5.378} &  40.503 & \tiny{  6.885}\\
  4 & \tiny{  0.099} &   1.583 & \tiny{  0.113} &  49 & \tiny{  1.273} &  19.075 & \tiny{  1.461} &  94 & \tiny{  2.851} &  34.403 & \tiny{  3.335} & 139 & \tiny{  5.442} &  40.366 & \tiny{  7.001}\\
  5 & \tiny{  0.123} &   1.979 & \tiny{  0.141} &  50 & \tiny{  1.302} &  19.450 & \tiny{  1.494} &  95 & \tiny{  2.895} &  34.688 & \tiny{  3.390} & 140 & \tiny{  5.506} &  40.206 & \tiny{  7.118}\\
  6 & \tiny{  0.148} &   2.374 & \tiny{  0.169} &  51 & \tiny{  1.331} &  19.824 & \tiny{  1.528} &  96 & \tiny{  2.940} &  34.969 & \tiny{  3.445} & 141 & \tiny{  5.568} &  40.024 & \tiny{  7.235}\\
  7 & \tiny{  0.173} &   2.770 & \tiny{  0.197} &  52 & \tiny{  1.360} &  20.196 & \tiny{  1.562} &  97 & \tiny{  2.985} &  35.246 & \tiny{  3.501} & 142 & \tiny{  5.628} &  39.817 & \tiny{  7.354}\\
  8 & \tiny{  0.197} &   3.165 & \tiny{  0.226} &  53 & \tiny{  1.390} &  20.568 & \tiny{  1.596} &  98 & \tiny{  3.031} &  35.519 & \tiny{  3.558} & 143 & \tiny{  5.687} &  39.584 & \tiny{  7.474}\\
  9 & \tiny{  0.222} &   3.560 & \tiny{  0.254} &  54 & \tiny{  1.419} &  20.939 & \tiny{  1.630} &  99 & \tiny{  3.078} &  35.788 & \tiny{  3.616} & 144 & \tiny{  5.744} &  39.325 & \tiny{  7.594}\\
 10 & \tiny{  0.247} &   3.955 & \tiny{  0.282} &  55 & \tiny{  1.449} &  21.309 & \tiny{  1.665} & 100 & \tiny{  3.125} &  36.053 & \tiny{  3.675} & 145 & \tiny{  5.797} &  39.038 & \tiny{  7.714}\\
 11 & \tiny{  0.272} &   4.350 & \tiny{  0.311} &  56 & \tiny{  1.479} &  21.677 & \tiny{  1.700} & 101 & \tiny{  3.173} &  36.313 & \tiny{  3.735} & 146 & \tiny{  5.848} &  38.721 & \tiny{  7.833}\\
 12 & \tiny{  0.297} &   4.745 & \tiny{  0.339} &  57 & \tiny{  1.510} &  22.045 & \tiny{  1.735} & 102 & \tiny{  3.221} &  36.568 & \tiny{  3.796} & 147 & \tiny{  5.895} &  38.373 & \tiny{  7.952}\\
 13 & \tiny{  0.322} &   5.140 & \tiny{  0.368} &  58 & \tiny{  1.540} &  22.411 & \tiny{  1.771} & 103 & \tiny{  3.270} &  36.819 & \tiny{  3.857} & 148 & \tiny{  5.938} &  37.992 & \tiny{  8.069}\\
 14 & \tiny{  0.347} &   5.534 & \tiny{  0.396} &  59 & \tiny{  1.571} &  22.776 & \tiny{  1.807} & 104 & \tiny{  3.320} &  37.065 & \tiny{  3.920} & 149 & \tiny{  5.976} &  37.577 & \tiny{  8.184}\\
 15 & \tiny{  0.372} &   5.928 & \tiny{  0.425} &  60 & \tiny{  1.602} &  23.139 & \tiny{  1.843} & 105 & \tiny{  3.370} &  37.306 & \tiny{  3.984} & 150 & \tiny{  6.009} &  37.126 & \tiny{  8.297}\\
 16 & \tiny{  0.397} &   6.322 & \tiny{  0.453} &  61 & \tiny{  1.633} &  23.502 & \tiny{  1.879} & 106 & \tiny{  3.421} &  37.541 & \tiny{  4.049} & 151 & \tiny{  6.036} &  36.638 & \tiny{  8.405}\\
 17 & \tiny{  0.422} &   6.716 & \tiny{  0.482} &  62 & \tiny{  1.665} &  23.863 & \tiny{  1.916} & 107 & \tiny{  3.472} &  37.771 & \tiny{  4.115} & 152 & \tiny{  6.056} &  36.110 & \tiny{  8.509}\\
 18 & \tiny{  0.447} &   7.110 & \tiny{  0.511} &  63 & \tiny{  1.696} &  24.222 & \tiny{  1.953} & 108 & \tiny{  3.525} &  37.996 & \tiny{  4.182} & 153 & \tiny{  6.069} &  35.541 & \tiny{  8.607}\\
 19 & \tiny{  0.472} &   7.503 & \tiny{  0.540} &  64 & \tiny{  1.728} &  24.581 & \tiny{  1.991} & 109 & \tiny{  3.578} &  38.214 & \tiny{  4.251} & 154 & \tiny{  6.072} &  34.929 & \tiny{  8.698}\\
 20 & \tiny{  0.497} &   7.896 & \tiny{  0.568} &  65 & \tiny{  1.761} &  24.938 & \tiny{  2.029} & 110 & \tiny{  3.631} &  38.426 & \tiny{  4.321} & 155 & \tiny{  6.066} &  34.271 & \tiny{  8.780}\\
 21 & \tiny{  0.523} &   8.288 & \tiny{  0.597} &  66 & \tiny{  1.793} &  25.293 & \tiny{  2.067} & 111 & \tiny{  3.686} &  38.632 & \tiny{  4.391} & 156 & \tiny{  6.050} &  33.567 & \tiny{  8.852}\\
 22 & \tiny{  0.548} &   8.680 & \tiny{  0.626} &  67 & \tiny{  1.826} &  25.647 & \tiny{  2.106} & 112 & \tiny{  3.741} &  38.831 & \tiny{  4.464} & 157 & \tiny{  6.022} &  32.813 & \tiny{  8.911}\\
 23 & \tiny{  0.573} &   9.072 & \tiny{  0.656} &  68 & \tiny{  1.859} &  25.999 & \tiny{  2.145} & 113 & \tiny{  3.796} &  39.023 & \tiny{  4.537} & 158 & \tiny{  5.980} &  32.007 & \tiny{  8.955}\\
 24 & \tiny{  0.599} &   9.464 & \tiny{  0.685} &  69 & \tiny{  1.893} &  26.349 & \tiny{  2.184} & 114 & \tiny{  3.853} &  39.209 & \tiny{  4.612} & 159 & \tiny{  5.925} &  31.149 & \tiny{  8.982}\\
 25 & \tiny{  0.625} &   9.855 & \tiny{  0.714} &  70 & \tiny{  1.927} &  26.698 & \tiny{  2.224} & 115 & \tiny{  3.910} &  39.386 & \tiny{  4.688} & 160 & \tiny{  5.854} &  30.235 & \tiny{  8.988}\\
 26 & \tiny{  0.650} &  10.246 & \tiny{  0.744} &  71 & \tiny{  1.961} &  27.045 & \tiny{  2.264} & 116 & \tiny{  3.968} &  39.556 & \tiny{  4.765} & 161 & \tiny{  5.766} &  29.265 & \tiny{  8.972}\\
 27 & \tiny{  0.676} &  10.636 & \tiny{  0.773} &  72 & \tiny{  1.995} &  27.390 & \tiny{  2.305} & 117 & \tiny{  4.026} &  39.718 & \tiny{  4.844} & 162 & \tiny{  5.660} &  28.236 & \tiny{  8.929}\\
 28 & \tiny{  0.702} &  11.026 & \tiny{  0.803} &  73 & \tiny{  2.030} &  27.734 & \tiny{  2.346} & 118 & \tiny{  4.086} &  39.872 & \tiny{  4.925} & 163 & \tiny{  5.535} &  27.146 & \tiny{  8.855}\\
 29 & \tiny{  0.728} &  11.415 & \tiny{  0.833} &  74 & \tiny{  2.065} &  28.075 & \tiny{  2.387} & 119 & \tiny{  4.146} &  40.017 & \tiny{  5.007} & 164 & \tiny{  5.389} &  25.996 & \tiny{  8.747}\\
 30 & \tiny{  0.754} &  11.804 & \tiny{  0.863} &  75 & \tiny{  2.100} &  28.415 & \tiny{  2.429} & 120 & \tiny{  4.206} &  40.153 & \tiny{  5.091} & 165 & \tiny{  5.221} &  24.783 & \tiny{  8.601}\\
 31 & \tiny{  0.780} &  12.192 & \tiny{  0.893} &  76 & \tiny{  2.136} &  28.753 & \tiny{  2.472} & 121 & \tiny{  4.268} &  40.279 & \tiny{  5.176} & 166 & \tiny{  5.030} &  23.506 & \tiny{  8.411}\\
 32 & \tiny{  0.806} &  12.580 & \tiny{  0.923} &  77 & \tiny{  2.172} &  29.088 & \tiny{  2.515} & 122 & \tiny{  4.330} &  40.396 & \tiny{  5.262} & 167 & \tiny{  4.815} &  22.167 & \tiny{  8.174}\\
 33 & \tiny{  0.833} &  12.968 & \tiny{  0.953} &  78 & \tiny{  2.209} &  29.421 & \tiny{  2.558} & 123 & \tiny{  4.393} &  40.502 & \tiny{  5.351} & 168 & \tiny{  4.576} &  20.764 & \tiny{  7.886}\\
 34 & \tiny{  0.859} &  13.354 & \tiny{  0.984} &  79 & \tiny{  2.246} &  29.752 & \tiny{  2.602} & 124 & \tiny{  4.456} &  40.598 & \tiny{  5.441} & 169 & \tiny{  4.311} &  19.299 & \tiny{  7.541}\\
 35 & \tiny{  0.886} &  13.741 & \tiny{  1.014} &  80 & \tiny{  2.283} &  30.081 & \tiny{  2.647} & 125 & \tiny{  4.520} &  40.683 & \tiny{  5.533} & 170 & \tiny{  4.021} &  17.774 & \tiny{  7.138}\\
 36 & \tiny{  0.913} &  14.126 & \tiny{  1.045} &  81 & \tiny{  2.321} &  30.408 & \tiny{  2.692} & 126 & \tiny{  4.584} &  40.756 & \tiny{  5.626} & 171 & \tiny{  3.707} &  16.189 & \tiny{  6.673}\\
 37 & \tiny{  0.939} &  14.511 & \tiny{  1.076} &  82 & \tiny{  2.359} &  30.732 & \tiny{  2.737} & 127 & \tiny{  4.649} &  40.816 & \tiny{  5.721} & 172 & \tiny{  3.368} &  14.549 & \tiny{  6.145}\\
 38 & \tiny{  0.966} &  14.896 & \tiny{  1.107} &  83 & \tiny{  2.397} &  31.054 & \tiny{  2.784} & 128 & \tiny{  4.715} &  40.864 & \tiny{  5.818} & 173 & \tiny{  3.005} &  12.857 & \tiny{  5.553}\\
 39 & \tiny{  0.994} &  15.279 & \tiny{  1.138} &  84 & \tiny{  2.436} &  31.373 & \tiny{  2.830} & 129 & \tiny{  4.780} &  40.899 & \tiny{  5.917} & 174 & \tiny{  2.621} &  11.116 & \tiny{  4.899}\\
 40 & \tiny{  1.021} &  15.662 & \tiny{  1.169} &  85 & \tiny{  2.475} &  31.689 & \tiny{  2.878} & 130 & \tiny{  4.847} &  40.920 & \tiny{  6.018} & 175 & \tiny{  2.217} &   9.333 & \tiny{  4.186}\\
 41 & \tiny{  1.048} &  16.045 & \tiny{  1.201} &  86 & \tiny{  2.515} &  32.003 & \tiny{  2.926} & 131 & \tiny{  4.913} &  40.926 & \tiny{  6.120} & 176 & \tiny{  1.796} &   7.513 & \tiny{  3.420}\\
 42 & \tiny{  1.076} &  16.426 & \tiny{  1.233} &  87 & \tiny{  2.555} &  32.314 & \tiny{  2.975} & 132 & \tiny{  4.980} &  40.918 & \tiny{  6.224} & 177 & \tiny{  1.360} &   5.662 & \tiny{  2.608}\\
 43 & \tiny{  1.103} &  16.807 & \tiny{  1.265} &  88 & \tiny{  2.596} &  32.622 & \tiny{  3.024} & 133 & \tiny{  5.047} &  40.893 & \tiny{  6.330} & 178 & \tiny{  0.913} &   3.788 & \tiny{  1.760}\\
 44 & \tiny{  1.131} &  17.187 & \tiny{  1.297} &  89 & \tiny{  2.637} &  32.927 & \tiny{  3.074} & 134 & \tiny{  5.113} &  40.851 & \tiny{  6.438} & 179 & \tiny{  0.458} &   1.898 & \tiny{  0.886}\\
 45 & \tiny{  1.159} &  17.566 & \tiny{  1.329} &  90 & \tiny{  2.679} &  33.228 & \tiny{  3.125} & 135 & \tiny{  5.180} &  40.793 & \tiny{  6.547} & 180 & \tiny{  0.000} &   0.000 & \tiny{  0.000}\\ 
\end{tabular}}
\caption[\em Epicyclic anomalies of Mars.]{\em Epicyclic anomalies of Mars. All quantities are in degrees. Note that $\bar{\theta}(360^\circ-\mu) = -\bar{\theta}(\mu)$, and $\delta\theta_{\pm}(360^\circ-\mu) = -\delta\theta_{\pm}(\mu)$. }\label{vt10}
\end{table}

\newpage
\begin{table}\centering
\begin{tabular}{lcl}
Event & Date & $\lambda$ \\\hline
&&\\[-1.75ex]
Conjunction & 01/07/2000 & 10CN13\\
Station (R) & 12/05/2001 & 29SG00\\
Opposition & 13/06/2001 & 22SG44\\
Station (D) & 19/07/2001 & 15SG02\\
Conjunction & 10/08/2002 & 18LE05\\
Station (R) & 29/07/2003 & 10PI20\\
Opposition & 28/08/2003 & 05PI03\\
Station (D) & 27/09/2003 & 29AQ55\\
Conjunction & 15/09/2004 & 23VI06\\
Station (R) & 02/10/2005 & 23TA31\\
Opposition & 07/11/2005 & 15TA06\\
Station (D) & 09/12/2005 & 08TA24\\
Conjunction & 23/10/2006 & 29LI44\\
Station (R) & 15/11/2007 & 12CN36\\
Opposition & 24/12/2007 & 02CN45\\
Station (D) & 31/01/2008 & 24GE15\\
Conjunction & 05/12/2008 & 14SG08\\
Station (R) & 20/12/2009 & 19LE35\\
Opposition & 29/01/2010 & 09LE45\\
Station (D) & 10/03/2010 & 00LE20\\
Conjunction & 04/02/2011 & 15AQ42\\
Station (R) & 24/01/2012 & 23VI01\\
Opposition & 03/03/2012 & 13VI42\\
Station (D) & 14/04/2012 & 03VI51\\
Conjunction & 17/04/2013 & 28AR06\\
Station (R) & 01/03/2014 & 27LI31\\
Opposition & 08/04/2014 & 19LI00\\
Station (D) & 20/05/2014 & 09LI04\\
Conjunction & 14/06/2015 & 23GE28\\
Station (R) & 17/04/2016 & 08SG45\\
Opposition & 22/05/2016 & 01SG43\\
Station (D) & 29/06/2016 & 22SC55\\
Conjunction & 27/07/2017 & 04LE11\\
Station (R) & 27/06/2018 & 09AQ35\\
Opposition & 27/07/2018 & 04AQ22\\
Station (D) & 27/08/2018 & 28CP43\\
Conjunction & 02/09/2019 & 09VI43\\
Station (R) & 10/09/2020 & 28AR08\\
Opposition & 13/10/2020 & 20AR59\\
Station (D) & 13/11/2020 & 15AR05\\
\end{tabular}
\caption[\em The conjunctions, oppositions, and stations of Mars
during the years 2000--2020 CE.]{\em The conjunctions, oppositions, and stations of Mars
during the years 2000--2020 CE. (R) indicates a retrograde station, and (D)
a direct station.}\label{vtmars}
\end{table}

\newpage
\begin{table}
\centering
\begin{tabular}{rrrr|rrrr}
$\Delta t$(JD)& $\Delta\bar{\lambda}(^\circ)$ &  $\Delta M(^\circ)$ & $\Delta \bar{F}(^\circ)$& $\Delta t$(JD) & $\Delta\bar{\lambda}(^\circ)$ & $\Delta M(^\circ)$ 
&$\Delta \bar{F}(^\circ)$\\ \hline
&&&&&&&\\[-1.75ex]
10,000 & 111.251 & 110.810 & 110.812 & 1,000 &  83.125 &  83.081 &  83.081\\
20,000 & 222.501 & 221.620 & 221.624 & 2,000 & 166.250 & 166.162 & 166.162\\
30,000 & 333.752 & 332.430 & 332.437 & 3,000 & 249.375 & 249.243 & 249.244\\
40,000 &  85.003 &  83.240 &  83.249 & 4,000 & 332.500 & 332.324 & 332.325\\
50,000 & 196.253 & 194.050 & 194.061 & 5,000 &  55.625 &  55.405 &  55.406\\
60,000 & 307.504 & 304.860 & 304.873 & 6,000 & 138.750 & 138.486 & 138.487\\
70,000 &  58.755 &  55.670 &  55.685 & 7,000 & 221.875 & 221.567 & 221.569\\
80,000 & 170.006 & 166.480 & 166.498 & 8,000 & 305.001 & 304.648 & 304.650\\
90,000 & 281.256 & 277.290 & 277.310 & 9,000 &  28.126 &  27.729 &  27.731\\
&&&&&&&\\
100 &   8.313 &   8.308 &   8.308 & 10 &   0.831 &   0.831 &   0.831\\
200 &  16.625 &  16.616 &  16.616 & 20 &   1.663 &   1.662 &   1.662\\
300 &  24.938 &  24.924 &  24.924 & 30 &   2.494 &   2.492 &   2.492\\
400 &  33.250 &  33.232 &  33.232 & 40 &   3.325 &   3.323 &   3.323\\
500 &  41.563 &  41.541 &  41.541 & 50 &   4.156 &   4.154 &   4.154\\
600 &  49.875 &  49.849 &  49.849 & 60 &   4.988 &   4.985 &   4.985\\
700 &  58.188 &  58.157 &  58.157 & 70 &   5.819 &   5.816 &   5.816\\
800 &  66.500 &  66.465 &  66.465 & 80 &   6.650 &   6.646 &   6.646\\
900 &  74.813 &  74.773 &  74.773 & 90 &   7.481 &   7.477 &   7.477\\
&&&&&&&\\
1 &   0.083 &   0.083 &   0.083 & 0.1 &   0.008 &   0.008 &   0.008\\
2 &   0.166 &   0.166 &   0.166 & 0.2 &   0.017 &   0.017 &   0.017\\
3 &   0.249 &   0.249 &   0.249 & 0.3 &   0.025 &   0.025 &   0.025\\
4 &   0.333 &   0.332 &   0.332 & 0.4 &   0.033 &   0.033 &   0.033\\
5 &   0.416 &   0.415 &   0.415 & 0.5 &   0.042 &   0.042 &   0.042\\
6 &   0.499 &   0.498 &   0.498 & 0.6 &   0.050 &   0.050 &   0.050\\
7 &   0.582 &   0.582 &   0.582 & 0.7 &   0.058 &   0.058 &   0.058\\
8 &   0.665 &   0.665 &   0.665 & 0.8 &   0.067 &   0.066 &   0.066\\
9 &   0.748 &   0.748 &   0.748 & 0.9 &   0.075 &   0.075 &   0.075\\
\end{tabular}
\caption[\em Mean motion of Jupiter.]{\em Mean motion of Jupiter.  Here, $\Delta t = t-t_0$, $\Delta\bar{\lambda} = \bar{\lambda}-\bar{\lambda}_0$, $\Delta M = M - M_0$, and $\Delta\bar{F} = \bar{F} - \bar{F}_0$. At epoch  ($t_0 = 2\,451\,545.0$ JD), $\bar{\lambda}_0 = 34.365^\circ$,  $M_0 = 19.348^\circ$, and $\bar{F}_0=293.660^\circ$. }\label{vt11}
\end{table}

\newpage
\begin{table}\centering
\small{ \begin{tabular}{rrr|rrr|rrr|rrr}
$M(^\circ)$ & $q(^\circ)$  & $100\,\zeta$ & $M(^\circ)$ & $q(^\circ)$  & $100\,\zeta$ & $M(^\circ)$ & $q(^\circ)$  & $100\,\zeta$& $M(^\circ)$ & $q(^\circ)$  & $100\,\zeta$\\\hline
&&&&&&&&&&&\\[-1.75ex]
  0 &   0.000 &  4.839 &  90 &   5.545 & -0.234 & 180 &   0.000 & -4.839 & 270 &  -5.545 & -0.234\\
  2 &   0.205 &  4.835 &  92 &   5.530 & -0.403 & 182 &  -0.182 & -4.836 & 272 &  -5.553 & -0.065\\
  4 &   0.410 &  4.826 &  94 &   5.508 & -0.571 & 184 &  -0.363 & -4.828 & 274 &  -5.554 &  0.105\\
  6 &   0.614 &  4.810 &  96 &   5.479 & -0.737 & 186 &  -0.545 & -4.815 & 276 &  -5.549 &  0.274\\
  8 &   0.818 &  4.787 &  98 &   5.444 & -0.903 & 188 &  -0.725 & -4.796 & 278 &  -5.537 &  0.444\\
 10 &   1.020 &  4.758 & 100 &   5.403 & -1.067 & 190 &  -0.905 & -4.772 & 280 &  -5.518 &  0.613\\
 12 &   1.221 &  4.723 & 102 &   5.355 & -1.230 & 192 &  -1.085 & -4.743 & 282 &  -5.492 &  0.782\\
 14 &   1.420 &  4.681 & 104 &   5.301 & -1.391 & 194 &  -1.263 & -4.709 & 284 &  -5.459 &  0.950\\
 16 &   1.617 &  4.633 & 106 &   5.241 & -1.550 & 196 &  -1.439 & -4.669 & 286 &  -5.419 &  1.117\\
 18 &   1.812 &  4.579 & 108 &   5.175 & -1.707 & 198 &  -1.615 & -4.624 & 288 &  -5.372 &  1.283\\
 20 &   2.004 &  4.519 & 110 &   5.102 & -1.862 & 200 &  -1.789 & -4.574 & 290 &  -5.318 &  1.448\\
 22 &   2.194 &  4.453 & 112 &   5.024 & -2.014 & 202 &  -1.961 & -4.519 & 292 &  -5.257 &  1.611\\
 24 &   2.380 &  4.382 & 114 &   4.941 & -2.163 & 204 &  -2.131 & -4.459 & 294 &  -5.190 &  1.773\\
 26 &   2.563 &  4.304 & 116 &   4.851 & -2.310 & 206 &  -2.298 & -4.394 & 296 &  -5.116 &  1.932\\
 28 &   2.742 &  4.221 & 118 &   4.757 & -2.454 & 208 &  -2.464 & -4.324 & 298 &  -5.035 &  2.089\\
 30 &   2.918 &  4.132 & 120 &   4.657 & -2.595 & 210 &  -2.627 & -4.249 & 300 &  -4.947 &  2.244\\
 32 &   3.089 &  4.038 & 122 &   4.551 & -2.732 & 212 &  -2.787 & -4.169 & 302 &  -4.853 &  2.396\\
 34 &   3.256 &  3.938 & 124 &   4.441 & -2.867 & 214 &  -2.945 & -4.085 & 304 &  -4.752 &  2.545\\
 36 &   3.419 &  3.834 & 126 &   4.326 & -2.997 & 216 &  -3.100 & -3.995 & 306 &  -4.645 &  2.691\\
 38 &   3.576 &  3.724 & 128 &   4.207 & -3.124 & 218 &  -3.251 & -3.902 & 308 &  -4.532 &  2.834\\
 40 &   3.729 &  3.610 & 130 &   4.082 & -3.248 & 220 &  -3.399 & -3.803 & 310 &  -4.413 &  2.973\\
 42 &   3.877 &  3.491 & 132 &   3.954 & -3.367 & 222 &  -3.543 & -3.701 & 312 &  -4.287 &  3.108\\
 44 &   4.019 &  3.368 & 134 &   3.821 & -3.482 & 224 &  -3.684 & -3.594 & 314 &  -4.156 &  3.240\\
 46 &   4.156 &  3.240 & 136 &   3.684 & -3.594 & 226 &  -3.821 & -3.482 & 316 &  -4.019 &  3.368\\
 48 &   4.287 &  3.108 & 138 &   3.543 & -3.701 & 228 &  -3.954 & -3.367 & 318 &  -3.877 &  3.491\\
 50 &   4.413 &  2.973 & 140 &   3.399 & -3.803 & 230 &  -4.082 & -3.248 & 320 &  -3.729 &  3.610\\
 52 &   4.532 &  2.834 & 142 &   3.251 & -3.902 & 232 &  -4.207 & -3.124 & 322 &  -3.576 &  3.724\\
 54 &   4.645 &  2.691 & 144 &   3.100 & -3.995 & 234 &  -4.326 & -2.997 & 324 &  -3.419 &  3.834\\
 56 &   4.752 &  2.545 & 146 &   2.945 & -4.085 & 236 &  -4.441 & -2.867 & 326 &  -3.256 &  3.938\\
 58 &   4.853 &  2.396 & 148 &   2.787 & -4.169 & 238 &  -4.551 & -2.732 & 328 &  -3.089 &  4.038\\
 60 &   4.947 &  2.244 & 150 &   2.627 & -4.249 & 240 &  -4.657 & -2.595 & 330 &  -2.918 &  4.132\\
 62 &   5.035 &  2.089 & 152 &   2.464 & -4.324 & 242 &  -4.757 & -2.454 & 332 &  -2.742 &  4.221\\
 64 &   5.116 &  1.932 & 154 &   2.298 & -4.394 & 244 &  -4.851 & -2.310 & 334 &  -2.563 &  4.304\\
 66 &   5.190 &  1.773 & 156 &   2.131 & -4.459 & 246 &  -4.941 & -2.163 & 336 &  -2.380 &  4.382\\
 68 &   5.257 &  1.611 & 158 &   1.961 & -4.519 & 248 &  -5.024 & -2.014 & 338 &  -2.194 &  4.453\\
 70 &   5.318 &  1.448 & 160 &   1.789 & -4.574 & 250 &  -5.102 & -1.862 & 340 &  -2.004 &  4.519\\
 72 &   5.372 &  1.283 & 162 &   1.615 & -4.624 & 252 &  -5.175 & -1.707 & 342 &  -1.812 &  4.579\\
 74 &   5.419 &  1.117 & 164 &   1.439 & -4.669 & 254 &  -5.241 & -1.550 & 344 &  -1.617 &  4.633\\
 76 &   5.459 &  0.950 & 166 &   1.263 & -4.709 & 256 &  -5.301 & -1.391 & 346 &  -1.420 &  4.681\\
 78 &   5.492 &  0.782 & 168 &   1.085 & -4.743 & 258 &  -5.355 & -1.230 & 348 &  -1.221 &  4.723\\
 80 &   5.518 &  0.613 & 170 &   0.905 & -4.772 & 260 &  -5.403 & -1.067 & 350 &  -1.020 &  4.758\\
 82 &   5.537 &  0.444 & 172 &   0.725 & -4.796 & 262 &  -5.444 & -0.903 & 352 &  -0.818 &  4.787\\
 84 &   5.549 &  0.274 & 174 &   0.545 & -4.815 & 264 &  -5.479 & -0.737 & 354 &  -0.614 &  4.810\\
 86 &   5.554 &  0.105 & 176 &   0.363 & -4.828 & 266 &  -5.508 & -0.571 & 356 &  -0.410 &  4.826\\
 88 &   5.553 & -0.065 & 178 &   0.182 & -4.836 & 268 &  -5.530 & -0.403 & 358 &  -0.205 &  4.835\\
 90 &   5.545 & -0.234 & 180 &   0.000 & -4.839 & 270 &  -5.545 & -0.234 & 360 &  -0.000 &  4.839\\
\end{tabular}}
\caption{\em Deferential anomalies of Jupiter.}\label{vt12}
\end{table}

\newpage
\begin{table}\centering
\small{ \begin{tabular}{rrrr|rrrr|rrrr|rrrr}
$\mu$ & $\delta\theta_-$  & $\bar{\theta}~~~~$ & $\delta\theta_+$ &
$\mu$ & $\delta\theta_-$  & $\bar{\theta}~~~~$ & $\delta\theta_+$ &
$\mu$ & $\delta\theta_-$  & $\bar{\theta}~~~~$ & $\delta\theta_+$ &
$\mu$ & $\delta\theta_-$  & $\bar{\theta}~~~~$ & $\delta\theta_+$ \\\hline
&&&&&&&&&&&&&&&\\[-1.75ex]
  0 & \tiny{  0.000} &   0.000 & \tiny{  0.000} &  45 & \tiny{  0.366} &   6.816 & \tiny{  0.410} &  90 & \tiny{  0.649} &  10.868 & \tiny{  0.736} & 135 & \tiny{  0.616} &   8.927 & \tiny{  0.713}\\
  1 & \tiny{  0.008} &   0.161 & \tiny{  0.009} &  46 & \tiny{  0.374} &   6.948 & \tiny{  0.418} &  91 & \tiny{  0.653} &  10.902 & \tiny{  0.741} & 136 & \tiny{  0.609} &   8.796 & \tiny{  0.705}\\
  2 & \tiny{  0.017} &   0.322 & \tiny{  0.019} &  47 & \tiny{  0.381} &   7.077 & \tiny{  0.427} &  92 & \tiny{  0.657} &  10.933 & \tiny{  0.746} & 137 & \tiny{  0.601} &   8.661 & \tiny{  0.697}\\
  3 & \tiny{  0.025} &   0.483 & \tiny{  0.028} &  48 & \tiny{  0.389} &   7.206 & \tiny{  0.436} &  93 & \tiny{  0.661} &  10.961 & \tiny{  0.750} & 138 & \tiny{  0.593} &   8.522 & \tiny{  0.688}\\
  4 & \tiny{  0.033} &   0.644 & \tiny{  0.037} &  49 & \tiny{  0.396} &   7.333 & \tiny{  0.444} &  94 & \tiny{  0.664} &  10.986 & \tiny{  0.755} & 139 & \tiny{  0.585} &   8.380 & \tiny{  0.679}\\
  5 & \tiny{  0.042} &   0.805 & \tiny{  0.046} &  50 & \tiny{  0.404} &   7.459 & \tiny{  0.453} &  95 & \tiny{  0.668} &  11.008 & \tiny{  0.759} & 140 & \tiny{  0.576} &   8.233 & \tiny{  0.669}\\
  6 & \tiny{  0.050} &   0.965 & \tiny{  0.056} &  51 & \tiny{  0.411} &   7.583 & \tiny{  0.461} &  96 & \tiny{  0.671} &  11.026 & \tiny{  0.763} & 141 & \tiny{  0.567} &   8.083 & \tiny{  0.659}\\
  7 & \tiny{  0.058} &   1.126 & \tiny{  0.065} &  52 & \tiny{  0.419} &   7.705 & \tiny{  0.470} &  97 & \tiny{  0.674} &  11.041 & \tiny{  0.766} & 142 & \tiny{  0.558} &   7.929 & \tiny{  0.649}\\
  8 & \tiny{  0.067} &   1.286 & \tiny{  0.074} &  53 & \tiny{  0.426} &   7.826 & \tiny{  0.478} &  98 & \tiny{  0.677} &  11.053 & \tiny{  0.770} & 143 & \tiny{  0.548} &   7.771 & \tiny{  0.638}\\
  9 & \tiny{  0.075} &   1.446 & \tiny{  0.084} &  54 & \tiny{  0.434} &   7.946 & \tiny{  0.486} &  99 & \tiny{  0.680} &  11.062 & \tiny{  0.773} & 144 & \tiny{  0.538} &   7.610 & \tiny{  0.627}\\
 10 & \tiny{  0.083} &   1.606 & \tiny{  0.093} &  55 & \tiny{  0.441} &   8.063 & \tiny{  0.495} & 100 & \tiny{  0.682} &  11.067 & \tiny{  0.777} & 145 & \tiny{  0.528} &   7.445 & \tiny{  0.615}\\
 11 & \tiny{  0.092} &   1.766 & \tiny{  0.102} &  56 & \tiny{  0.448} &   8.180 & \tiny{  0.503} & 101 & \tiny{  0.684} &  11.069 & \tiny{  0.780} & 146 & \tiny{  0.517} &   7.276 & \tiny{  0.603}\\
 12 & \tiny{  0.100} &   1.925 & \tiny{  0.111} &  57 & \tiny{  0.455} &   8.294 & \tiny{  0.511} & 102 & \tiny{  0.686} &  11.068 & \tiny{  0.782} & 147 & \tiny{  0.506} &   7.104 & \tiny{  0.590}\\
 13 & \tiny{  0.108} &   2.084 & \tiny{  0.121} &  58 & \tiny{  0.462} &   8.407 & \tiny{  0.519} & 103 & \tiny{  0.688} &  11.063 & \tiny{  0.785} & 148 & \tiny{  0.495} &   6.929 & \tiny{  0.577}\\
 14 & \tiny{  0.116} &   2.242 & \tiny{  0.130} &  59 & \tiny{  0.470} &   8.517 & \tiny{  0.527} & 104 & \tiny{  0.690} &  11.054 & \tiny{  0.787} & 149 & \tiny{  0.484} &   6.750 & \tiny{  0.564}\\
 15 & \tiny{  0.125} &   2.400 & \tiny{  0.139} &  60 & \tiny{  0.477} &   8.626 & \tiny{  0.535} & 105 & \tiny{  0.692} &  11.042 & \tiny{  0.789} & 150 & \tiny{  0.472} &   6.568 & \tiny{  0.550}\\
 16 & \tiny{  0.133} &   2.558 & \tiny{  0.148} &  61 & \tiny{  0.484} &   8.734 & \tiny{  0.543} & 106 & \tiny{  0.693} &  11.027 & \tiny{  0.791} & 151 & \tiny{  0.459} &   6.383 & \tiny{  0.536}\\
 17 & \tiny{  0.141} &   2.715 & \tiny{  0.158} &  62 & \tiny{  0.490} &   8.839 & \tiny{  0.551} & 107 & \tiny{  0.694} &  11.008 & \tiny{  0.793} & 152 & \tiny{  0.447} &   6.194 & \tiny{  0.522}\\
 18 & \tiny{  0.149} &   2.872 & \tiny{  0.167} &  63 & \tiny{  0.497} &   8.942 & \tiny{  0.559} & 108 & \tiny{  0.695} &  10.985 & \tiny{  0.794} & 153 & \tiny{  0.434} &   6.003 & \tiny{  0.507}\\
 19 & \tiny{  0.158} &   3.028 & \tiny{  0.176} &  64 & \tiny{  0.504} &   9.044 & \tiny{  0.567} & 109 & \tiny{  0.695} &  10.959 & \tiny{  0.795} & 154 & \tiny{  0.421} &   5.808 & \tiny{  0.492}\\
 20 & \tiny{  0.166} &   3.184 & \tiny{  0.185} &  65 & \tiny{  0.511} &   9.143 & \tiny{  0.574} & 110 & \tiny{  0.696} &  10.929 & \tiny{  0.796} & 155 & \tiny{  0.407} &   5.610 & \tiny{  0.476}\\
 21 & \tiny{  0.174} &   3.339 & \tiny{  0.194} &  66 & \tiny{  0.517} &   9.240 & \tiny{  0.582} & 111 & \tiny{  0.696} &  10.895 & \tiny{  0.796} & 156 & \tiny{  0.393} &   5.410 & \tiny{  0.460}\\
 22 & \tiny{  0.182} &   3.494 & \tiny{  0.204} &  67 & \tiny{  0.524} &   9.336 & \tiny{  0.590} & 112 & \tiny{  0.696} &  10.858 & \tiny{  0.797} & 157 & \tiny{  0.379} &   5.206 & \tiny{  0.444}\\
 23 & \tiny{  0.191} &   3.648 & \tiny{  0.213} &  68 & \tiny{  0.531} &   9.429 & \tiny{  0.597} & 113 & \tiny{  0.695} &  10.817 & \tiny{  0.797} & 158 & \tiny{  0.365} &   5.000 & \tiny{  0.427}\\
 24 & \tiny{  0.199} &   3.801 & \tiny{  0.222} &  69 & \tiny{  0.537} &   9.520 & \tiny{  0.605} & 114 & \tiny{  0.695} &  10.772 & \tiny{  0.796} & 159 & \tiny{  0.350} &   4.792 & \tiny{  0.410}\\
 25 & \tiny{  0.207} &   3.954 & \tiny{  0.231} &  70 & \tiny{  0.543} &   9.609 & \tiny{  0.612} & 115 & \tiny{  0.694} &  10.723 & \tiny{  0.796} & 160 & \tiny{  0.336} &   4.581 & \tiny{  0.393}\\
 26 & \tiny{  0.215} &   4.106 & \tiny{  0.240} &  71 & \tiny{  0.550} &   9.696 & \tiny{  0.619} & 116 & \tiny{  0.693} &  10.671 & \tiny{  0.795} & 161 & \tiny{  0.320} &   4.367 & \tiny{  0.375}\\
 27 & \tiny{  0.223} &   4.257 & \tiny{  0.249} &  72 & \tiny{  0.556} &   9.780 & \tiny{  0.626} & 117 & \tiny{  0.691} &  10.614 & \tiny{  0.794} & 162 & \tiny{  0.305} &   4.151 & \tiny{  0.357}\\
 28 & \tiny{  0.231} &   4.407 & \tiny{  0.258} &  73 & \tiny{  0.562} &   9.862 & \tiny{  0.633} & 118 & \tiny{  0.690} &  10.554 & \tiny{  0.792} & 163 & \tiny{  0.289} &   3.933 & \tiny{  0.339}\\
 29 & \tiny{  0.239} &   4.557 & \tiny{  0.268} &  74 & \tiny{  0.568} &   9.942 & \tiny{  0.640} & 119 & \tiny{  0.688} &  10.490 & \tiny{  0.790} & 164 & \tiny{  0.274} &   3.713 & \tiny{  0.321}\\
 30 & \tiny{  0.248} &   4.705 & \tiny{  0.277} &  75 & \tiny{  0.574} &  10.019 & \tiny{  0.647} & 120 & \tiny{  0.686} &  10.422 & \tiny{  0.788} & 165 & \tiny{  0.258} &   3.491 & \tiny{  0.302}\\
 31 & \tiny{  0.256} &   4.853 & \tiny{  0.286} &  76 & \tiny{  0.580} &  10.094 & \tiny{  0.654} & 121 & \tiny{  0.683} &  10.350 & \tiny{  0.786} & 166 & \tiny{  0.241} &   3.267 & \tiny{  0.283}\\
 32 & \tiny{  0.264} &   5.000 & \tiny{  0.295} &  77 & \tiny{  0.585} &  10.167 & \tiny{  0.661} & 122 & \tiny{  0.680} &  10.274 & \tiny{  0.783} & 167 & \tiny{  0.225} &   3.041 & \tiny{  0.264}\\
 33 & \tiny{  0.272} &   5.146 & \tiny{  0.304} &  78 & \tiny{  0.591} &  10.237 & \tiny{  0.667} & 123 & \tiny{  0.677} &  10.194 & \tiny{  0.780} & 168 & \tiny{  0.208} &   2.814 & \tiny{  0.244}\\
 34 & \tiny{  0.280} &   5.292 & \tiny{  0.313} &  79 & \tiny{  0.596} &  10.304 & \tiny{  0.674} & 124 & \tiny{  0.674} &  10.110 & \tiny{  0.776} & 169 & \tiny{  0.192} &   2.585 & \tiny{  0.225}\\
 35 & \tiny{  0.288} &   5.436 & \tiny{  0.322} &  80 & \tiny{  0.602} &  10.369 & \tiny{  0.680} & 125 & \tiny{  0.670} &  10.023 & \tiny{  0.772} & 170 & \tiny{  0.175} &   2.354 & \tiny{  0.205}\\
 36 & \tiny{  0.296} &   5.579 & \tiny{  0.331} &  81 & \tiny{  0.607} &  10.431 & \tiny{  0.686} & 126 & \tiny{  0.666} &   9.931 & \tiny{  0.768} & 171 & \tiny{  0.158} &   2.123 & \tiny{  0.185}\\
 37 & \tiny{  0.304} &   5.721 & \tiny{  0.339} &  82 & \tiny{  0.612} &  10.491 & \tiny{  0.692} & 127 & \tiny{  0.662} &   9.835 & \tiny{  0.764} & 172 & \tiny{  0.140} &   1.890 & \tiny{  0.165}\\
 38 & \tiny{  0.311} &   5.862 & \tiny{  0.348} &  83 & \tiny{  0.617} &  10.548 & \tiny{  0.698} & 128 & \tiny{  0.657} &   9.736 & \tiny{  0.759} & 173 & \tiny{  0.123} &   1.656 & \tiny{  0.145}\\
 39 & \tiny{  0.319} &   6.002 & \tiny{  0.357} &  84 & \tiny{  0.622} &  10.602 & \tiny{  0.704} & 129 & \tiny{  0.652} &   9.632 & \tiny{  0.753} & 174 & \tiny{  0.106} &   1.421 & \tiny{  0.124}\\
 40 & \tiny{  0.327} &   6.141 & \tiny{  0.366} &  85 & \tiny{  0.627} &  10.654 & \tiny{  0.710} & 130 & \tiny{  0.647} &   9.524 & \tiny{  0.748} & 175 & \tiny{  0.088} &   1.185 & \tiny{  0.104}\\
 41 & \tiny{  0.335} &   6.278 & \tiny{  0.375} &  86 & \tiny{  0.632} &  10.702 & \tiny{  0.715} & 131 & \tiny{  0.641} &   9.413 & \tiny{  0.742} & 176 & \tiny{  0.071} &   0.949 & \tiny{  0.083}\\
 42 & \tiny{  0.343} &   6.415 & \tiny{  0.384} &  87 & \tiny{  0.636} &  10.748 & \tiny{  0.721} & 132 & \tiny{  0.636} &   9.297 & \tiny{  0.735} & 177 & \tiny{  0.053} &   0.712 & \tiny{  0.062}\\
 43 & \tiny{  0.351} &   6.550 & \tiny{  0.392} &  88 & \tiny{  0.641} &  10.791 & \tiny{  0.726} & 133 & \tiny{  0.629} &   9.178 & \tiny{  0.728} & 178 & \tiny{  0.035} &   0.475 & \tiny{  0.042}\\
 44 & \tiny{  0.358} &   6.684 & \tiny{  0.401} &  89 & \tiny{  0.645} &  10.831 & \tiny{  0.731} & 134 & \tiny{  0.623} &   9.055 & \tiny{  0.721} & 179 & \tiny{  0.018} &   0.238 & \tiny{  0.021}\\
 45 & \tiny{  0.366} &   6.816 & \tiny{  0.410} &  90 & \tiny{  0.649} &  10.868 & \tiny{  0.736} & 135 & \tiny{  0.616} &   8.927 & \tiny{  0.713} & 180 & \tiny{  0.000} &   0.000 & \tiny{  0.000}\\
\end{tabular}}
\caption[\em  Epicyclic anomalies of Jupiter. ]{\em Epicyclic anomalies of Jupiter. All quantities are in degrees. Note that $\bar{\theta}(360^\circ-\mu) = -\bar{\theta}(\mu)$, and $\delta\theta_{\pm}(360^\circ-\mu) = -\delta\theta_{\pm}(\mu)$. }\label{vt13}
\end{table}

\newpage
\begin{table}\centering
{\small\begin{tabular}{lcl}
Event & Date & $\lambda$ \\\hline
&&\\[-1.75ex]
Conjunction & 08/05/2000 & 17TA53\\
Station (R) & 29/09/2000 & 11GE13\\
Opposition & 28/11/2000 & 06GE08\\
Station (D) & 25/01/2001 & 01GE10\\
Conjunction & 14/06/2001 & 23GE30\\
Station (R) & 02/11/2001 & 15CN41\\
Opposition & 01/01/2002 & 10CN37\\
Station (D) & 01/03/2002 & 05CN37\\
Conjunction & 20/07/2002 & 27CN11\\
Station (R) & 04/12/2002 & 18LE06\\
Opposition & 02/02/2003 & 13LE06\\
Station (D) & 04/04/2003 & 08LE03\\
Conjunction & 22/08/2003 & 28LE55\\
Station (R) & 04/01/2004 & 18VI54\\
Opposition & 04/03/2004 & 13VI58\\
Station (D) & 05/05/2004 & 08VI55\\
Conjunction & 22/09/2004 & 29VI21\\
Station (R) & 02/02/2005 & 18LI53\\
Opposition & 03/04/2005 & 14LI00\\
Station (D) & 05/06/2005 & 08LI58\\
Conjunction & 22/10/2005 & 29LI16\\
Station (R) & 04/03/2006 & 18SC54\\
Opposition & 04/05/2006 & 14SC03\\
Station (D) & 06/07/2006 & 09SC02\\
Conjunction & 22/11/2006 & 29SC34\\
Station (R) & 06/04/2007 & 19SG49\\
Opposition & 06/06/2007 & 14SG57\\
Station (D) & 07/08/2007 & 09SG58\\
Conjunction & 23/12/2007 & 01CP03\\
Station (R) & 09/05/2008 & 22CP23\\
Opposition & 09/07/2008 & 17CP30\\
Station (D) & 08/09/2008 & 12CP33\\
Conjunction & 24/01/2009 & 04AQ23\\
Station (R) & 15/06/2009 & 27AQ01\\
Opposition & 14/08/2009 & 22AQ04\\
Station (D) & 13/10/2009 & 17AQ10\\
Conjunction & 28/02/2010 & 09PI43\\
Station (R) & 23/07/2010 & 03AR20\\
Opposition & 21/09/2010 & 28PI19\\
Station (D) & 18/11/2010 & 23PI26\\
\end{tabular}}
\caption[\em The conjunctions, oppositions, and stations of Jupiter
during the years 2000--2010 CE.]{\em The conjunctions, oppositions, and stations of Jupiter
during the years 2000--2010 CE. (R) indicates a retrograde station, and (D)
a direct station.}\label{vtjupiter}
\end{table}

\newpage
\begin{table}
\centering
\begin{tabular}{rrrr|rrrr}
$\Delta t$(JD)& $\Delta\bar{\lambda}(^\circ)$ &  $\Delta M(^\circ)$ & $\Delta \bar{F}(^\circ)$& $\Delta t$(JD) & $\Delta\bar{\lambda}(^\circ)$ & $\Delta M(^\circ)$ 
&$\Delta \bar{F}(^\circ)$\\ \hline
&&&&&&&\\[-1.75ex]
10,000 & 335.083 & 334.815 & 334.779 & 1,000 &  33.508 &  33.482 &  33.478\\
20,000 & 310.166 & 309.630 & 309.559 & 2,000 &  67.017 &  66.963 &  66.956\\
30,000 & 285.249 & 284.446 & 284.338 & 3,000 & 100.525 & 100.445 & 100.434\\
40,000 & 260.332 & 259.261 & 259.118 & 4,000 & 134.033 & 133.926 & 133.912\\
50,000 & 235.415 & 234.076 & 233.897 & 5,000 & 167.541 & 167.408 & 167.390\\
60,000 & 210.498 & 208.891 & 208.677 & 6,000 & 201.050 & 200.889 & 200.868\\
70,000 & 185.581 & 183.706 & 183.456 & 7,000 & 234.558 & 234.371 & 234.346\\
80,000 & 160.664 & 158.522 & 158.236 & 8,000 & 268.066 & 267.852 & 267.824\\
90,000 & 135.747 & 133.337 & 133.015 & 9,000 & 301.575 & 301.334 & 301.302\\
&&&&&&&\\
100 &   3.351 &   3.348 &   3.348 & 10 &   0.335 &   0.335 &   0.335\\
200 &   6.702 &   6.696 &   6.696 & 20 &   0.670 &   0.670 &   0.670\\
300 &  10.052 &  10.044 &  10.043 & 30 &   1.005 &   1.004 &   1.004\\
400 &  13.403 &  13.393 &  13.391 & 40 &   1.340 &   1.339 &   1.339\\
500 &  16.754 &  16.741 &  16.739 & 50 &   1.675 &   1.674 &   1.674\\
600 &  20.105 &  20.089 &  20.087 & 60 &   2.010 &   2.009 &   2.009\\
700 &  23.456 &  23.437 &  23.435 & 70 &   2.346 &   2.344 &   2.343\\
800 &  26.807 &  26.785 &  26.782 & 80 &   2.681 &   2.679 &   2.678\\
900 &  30.157 &  30.133 &  30.130 & 90 &   3.016 &   3.013 &   3.013\\
&&&&&&&\\
1 &   0.034 &   0.033 &   0.033 & 0.1 &   0.003 &   0.003 &   0.003\\
2 &   0.067 &   0.067 &   0.067 & 0.2 &   0.007 &   0.007 &   0.007\\
3 &   0.101 &   0.100 &   0.100 & 0.3 &   0.010 &   0.010 &   0.010\\
4 &   0.134 &   0.134 &   0.134 & 0.4 &   0.013 &   0.013 &   0.013\\
5 &   0.168 &   0.167 &   0.167 & 0.5 &   0.017 &   0.017 &   0.017\\
6 &   0.201 &   0.201 &   0.201 & 0.6 &   0.020 &   0.020 &   0.020\\
7 &   0.235 &   0.234 &   0.234 & 0.7 &   0.023 &   0.023 &   0.023\\
8 &   0.268 &   0.268 &   0.268 & 0.8 &   0.027 &   0.027 &   0.027\\
9 &   0.302 &   0.301 &   0.301 & 0.9 &   0.030 &   0.030 &   0.030\\
\end{tabular}
\caption[\em Mean motion of Saturn.]{\em Mean motion of Saturn.  Here, $\Delta t = t-t_0$, $\Delta\bar{\lambda} = \bar{\lambda}-\bar{\lambda}_0$,  $\Delta M = M - M_0$, and $\Delta\bar{F} = \bar{F} - \bar{F}_0$.  At epoch  ($t_0 = 2\,451\,545.0$ JD), $\bar{\lambda}_0 = 50.059^\circ$,  $M_0 = 317.857^\circ$, and
$\bar{F}_0 =296.482^\circ$. }\label{vt14}
\end{table}

\newpage
\begin{table}\centering
\small{ \begin{tabular}{rrr|rrr|rrr|rrr}
$M(^\circ)$ & $q(^\circ)$  & $100\,\zeta$ & $M(^\circ)$ & $q(^\circ)$  & $100\,\zeta$ & $M(^\circ)$ & $q(^\circ)$  & $100\,\zeta$& $M(^\circ)$ & $q(^\circ)$  & $100\,\zeta$\\\hline
&&&&&&&&&&&\\[-1.75ex]
  0 &   0.000 &  5.386 &  90 &   6.172 & -0.290 & 180 &   0.000 & -5.386 & 270 &  -6.172 & -0.290\\
  2 &   0.230 &  5.383 &  92 &   6.154 & -0.478 & 182 &  -0.201 & -5.383 & 272 &  -6.183 & -0.102\\
  4 &   0.459 &  5.372 &  94 &   6.128 & -0.664 & 184 &  -0.402 & -5.374 & 274 &  -6.186 &  0.087\\
  6 &   0.688 &  5.354 &  96 &   6.095 & -0.850 & 186 &  -0.602 & -5.360 & 276 &  -6.182 &  0.276\\
  8 &   0.916 &  5.328 &  98 &   6.055 & -1.034 & 188 &  -0.802 & -5.339 & 278 &  -6.169 &  0.465\\
 10 &   1.143 &  5.296 & 100 &   6.007 & -1.217 & 190 &  -1.001 & -5.313 & 280 &  -6.149 &  0.654\\
 12 &   1.368 &  5.256 & 102 &   5.953 & -1.397 & 192 &  -1.199 & -5.281 & 282 &  -6.122 &  0.842\\
 14 &   1.591 &  5.209 & 104 &   5.891 & -1.576 & 194 &  -1.396 & -5.243 & 284 &  -6.086 &  1.030\\
 16 &   1.811 &  5.156 & 106 &   5.823 & -1.753 & 196 &  -1.591 & -5.200 & 286 &  -6.043 &  1.217\\
 18 &   2.029 &  5.095 & 108 &   5.748 & -1.927 & 198 &  -1.785 & -5.150 & 288 &  -5.992 &  1.402\\
 20 &   2.245 &  5.027 & 110 &   5.666 & -2.098 & 200 &  -1.977 & -5.095 & 290 &  -5.933 &  1.586\\
 22 &   2.456 &  4.953 & 112 &   5.578 & -2.267 & 202 &  -2.168 & -5.035 & 292 &  -5.867 &  1.768\\
 24 &   2.665 &  4.873 & 114 &   5.484 & -2.433 & 204 &  -2.356 & -4.969 & 294 &  -5.793 &  1.949\\
 26 &   2.869 &  4.785 & 116 &   5.384 & -2.596 & 206 &  -2.542 & -4.897 & 296 &  -5.711 &  2.127\\
 28 &   3.070 &  4.692 & 118 &   5.277 & -2.755 & 208 &  -2.725 & -4.820 & 298 &  -5.622 &  2.302\\
 30 &   3.266 &  4.592 & 120 &   5.165 & -2.911 & 210 &  -2.906 & -4.737 & 300 &  -5.525 &  2.476\\
 32 &   3.457 &  4.486 & 122 &   5.048 & -3.063 & 212 &  -3.084 & -4.649 & 302 &  -5.421 &  2.646\\
 34 &   3.644 &  4.375 & 124 &   4.924 & -3.211 & 214 &  -3.259 & -4.556 & 304 &  -5.310 &  2.813\\
 36 &   3.825 &  4.257 & 126 &   4.796 & -3.356 & 216 &  -3.430 & -4.458 & 306 &  -5.191 &  2.976\\
 38 &   4.002 &  4.134 & 128 &   4.662 & -3.496 & 218 &  -3.598 & -4.354 & 308 &  -5.065 &  3.136\\
 40 &   4.172 &  4.006 & 130 &   4.524 & -3.632 & 220 &  -3.763 & -4.246 & 310 &  -4.933 &  3.292\\
 42 &   4.337 &  3.873 & 132 &   4.380 & -3.764 & 222 &  -3.923 & -4.133 & 312 &  -4.793 &  3.444\\
 44 &   4.495 &  3.735 & 134 &   4.232 & -3.892 & 224 &  -4.080 & -4.015 & 314 &  -4.648 &  3.591\\
 46 &   4.648 &  3.591 & 136 &   4.080 & -4.015 & 226 &  -4.232 & -3.892 & 316 &  -4.495 &  3.735\\
 48 &   4.793 &  3.444 & 138 &   3.923 & -4.133 & 228 &  -4.380 & -3.764 & 318 &  -4.337 &  3.873\\
 50 &   4.933 &  3.292 & 140 &   3.763 & -4.246 & 230 &  -4.524 & -3.632 & 320 &  -4.172 &  4.006\\
 52 &   5.065 &  3.136 & 142 &   3.598 & -4.354 & 232 &  -4.662 & -3.496 & 322 &  -4.002 &  4.134\\
 54 &   5.191 &  2.976 & 144 &   3.430 & -4.458 & 234 &  -4.796 & -3.356 & 324 &  -3.825 &  4.257\\
 56 &   5.310 &  2.813 & 146 &   3.259 & -4.556 & 236 &  -4.924 & -3.211 & 326 &  -3.644 &  4.375\\
 58 &   5.421 &  2.646 & 148 &   3.084 & -4.649 & 238 &  -5.048 & -3.063 & 328 &  -3.457 &  4.486\\
 60 &   5.525 &  2.476 & 150 &   2.906 & -4.737 & 240 &  -5.165 & -2.911 & 330 &  -3.266 &  4.592\\
 62 &   5.622 &  2.302 & 152 &   2.725 & -4.820 & 242 &  -5.277 & -2.755 & 332 &  -3.070 &  4.692\\
 64 &   5.711 &  2.127 & 154 &   2.542 & -4.897 & 244 &  -5.384 & -2.596 & 334 &  -2.869 &  4.785\\
 66 &   5.793 &  1.949 & 156 &   2.356 & -4.969 & 246 &  -5.484 & -2.433 & 336 &  -2.665 &  4.873\\
 68 &   5.867 &  1.768 & 158 &   2.168 & -5.035 & 248 &  -5.578 & -2.267 & 338 &  -2.456 &  4.953\\
 70 &   5.933 &  1.586 & 160 &   1.977 & -5.095 & 250 &  -5.666 & -2.098 & 340 &  -2.245 &  5.027\\
 72 &   5.992 &  1.402 & 162 &   1.785 & -5.150 & 252 &  -5.748 & -1.927 & 342 &  -2.029 &  5.095\\
 74 &   6.043 &  1.217 & 164 &   1.591 & -5.200 & 254 &  -5.823 & -1.753 & 344 &  -1.811 &  5.156\\
 76 &   6.086 &  1.030 & 166 &   1.396 & -5.243 & 256 &  -5.891 & -1.576 & 346 &  -1.591 &  5.209\\
 78 &   6.122 &  0.842 & 168 &   1.199 & -5.281 & 258 &  -5.953 & -1.397 & 348 &  -1.368 &  5.256\\
 80 &   6.149 &  0.654 & 170 &   1.001 & -5.313 & 260 &  -6.007 & -1.217 & 350 &  -1.143 &  5.296\\
 82 &   6.169 &  0.465 & 172 &   0.802 & -5.339 & 262 &  -6.055 & -1.034 & 352 &  -0.916 &  5.328\\
 84 &   6.182 &  0.276 & 174 &   0.602 & -5.360 & 264 &  -6.095 & -0.850 & 354 &  -0.688 &  5.354\\
 86 &   6.186 &  0.087 & 176 &   0.402 & -5.374 & 266 &  -6.128 & -0.664 & 356 &  -0.459 &  5.372\\
 88 &   6.183 & -0.102 & 178 &   0.201 & -5.383 & 268 &  -6.154 & -0.478 & 358 &  -0.230 &  5.383\\
 90 &   6.172 & -0.290 & 180 &   0.000 & -5.386 & 270 &  -6.172 & -0.290 & 360 &  -0.000 &  5.386\\
\end{tabular}}
\caption{\em Deferential anomalies of Saturn.}\label{vt15}
\end{table}

\newpage
\begin{table}\centering
\small{ \begin{tabular}{rrrr|rrrr|rrrr|rrrr}
$\mu$ & $\delta\theta_-$  & $\bar{\theta}~~~~$ & $\delta\theta_+$ &
$\mu$ & $\delta\theta_-$  & $\bar{\theta}~~~~$ & $\delta\theta_+$ &
$\mu$ & $\delta\theta_-$  & $\bar{\theta}~~~~$ & $\delta\theta_+$ &
$\mu$ & $\delta\theta_-$  & $\bar{\theta}~~~~$ & $\delta\theta_+$ \\\hline
&&&&&&&&&&&&&&&\\[-1.75ex]
  0 & \tiny{  0.000} &   0.000 & \tiny{  0.000} &  45 & \tiny{  0.242} &   3.944 & \tiny{  0.276} &  90 & \tiny{  0.391} &   5.979 & \tiny{  0.450} & 135 & \tiny{  0.322} &   4.573 & \tiny{  0.375}\\
  1 & \tiny{  0.006} &   0.095 & \tiny{  0.006} &  46 & \tiny{  0.247} &   4.017 & \tiny{  0.282} &  91 & \tiny{  0.393} &   5.989 & \tiny{  0.452} & 136 & \tiny{  0.318} &   4.499 & \tiny{  0.370}\\
  2 & \tiny{  0.011} &   0.190 & \tiny{  0.013} &  47 & \tiny{  0.252} &   4.089 & \tiny{  0.287} &  92 & \tiny{  0.394} &   5.997 & \tiny{  0.453} & 137 & \tiny{  0.313} &   4.423 & \tiny{  0.364}\\
  3 & \tiny{  0.017} &   0.284 & \tiny{  0.019} &  48 & \tiny{  0.256} &   4.160 & \tiny{  0.292} &  93 & \tiny{  0.395} &   6.004 & \tiny{  0.454} & 138 & \tiny{  0.308} &   4.346 & \tiny{  0.358}\\
  4 & \tiny{  0.023} &   0.379 & \tiny{  0.026} &  49 & \tiny{  0.261} &   4.230 & \tiny{  0.298} &  94 & \tiny{  0.396} &   6.008 & \tiny{  0.456} & 139 & \tiny{  0.302} &   4.267 & \tiny{  0.352}\\
  5 & \tiny{  0.028} &   0.474 & \tiny{  0.032} &  50 & \tiny{  0.265} &   4.299 & \tiny{  0.303} &  95 & \tiny{  0.397} &   6.011 & \tiny{  0.457} & 140 & \tiny{  0.297} &   4.186 & \tiny{  0.346}\\
  6 & \tiny{  0.034} &   0.568 & \tiny{  0.039} &  51 & \tiny{  0.270} &   4.367 & \tiny{  0.308} &  96 & \tiny{  0.397} &   6.012 & \tiny{  0.458} & 141 & \tiny{  0.292} &   4.104 & \tiny{  0.340}\\
  7 & \tiny{  0.040} &   0.662 & \tiny{  0.045} &  52 & \tiny{  0.274} &   4.433 & \tiny{  0.313} &  97 & \tiny{  0.398} &   6.011 & \tiny{  0.459} & 142 & \tiny{  0.286} &   4.020 & \tiny{  0.333}\\
  8 & \tiny{  0.045} &   0.757 & \tiny{  0.052} &  53 & \tiny{  0.279} &   4.499 & \tiny{  0.318} &  98 & \tiny{  0.399} &   6.008 & \tiny{  0.459} & 143 & \tiny{  0.280} &   3.935 & \tiny{  0.327}\\
  9 & \tiny{  0.051} &   0.851 & \tiny{  0.058} &  54 & \tiny{  0.283} &   4.564 & \tiny{  0.323} &  99 & \tiny{  0.399} &   6.004 & \tiny{  0.460} & 144 & \tiny{  0.274} &   3.848 & \tiny{  0.320}\\
 10 & \tiny{  0.057} &   0.945 & \tiny{  0.064} &  55 & \tiny{  0.287} &   4.627 & \tiny{  0.328} & 100 & \tiny{  0.399} &   5.997 & \tiny{  0.460} & 145 & \tiny{  0.269} &   3.760 & \tiny{  0.313}\\
 11 & \tiny{  0.062} &   1.038 & \tiny{  0.071} &  56 & \tiny{  0.292} &   4.689 & \tiny{  0.333} & 101 & \tiny{  0.399} &   5.989 & \tiny{  0.461} & 146 & \tiny{  0.262} &   3.670 & \tiny{  0.306}\\
 12 & \tiny{  0.068} &   1.132 & \tiny{  0.077} &  57 & \tiny{  0.296} &   4.750 & \tiny{  0.338} & 102 & \tiny{  0.399} &   5.979 & \tiny{  0.461} & 147 & \tiny{  0.256} &   3.578 & \tiny{  0.299}\\
 13 & \tiny{  0.074} &   1.225 & \tiny{  0.084} &  58 & \tiny{  0.300} &   4.810 & \tiny{  0.343} & 103 & \tiny{  0.399} &   5.966 & \tiny{  0.461} & 148 & \tiny{  0.250} &   3.486 & \tiny{  0.292}\\
 14 & \tiny{  0.079} &   1.318 & \tiny{  0.090} &  59 & \tiny{  0.304} &   4.869 & \tiny{  0.347} & 104 & \tiny{  0.399} &   5.952 & \tiny{  0.460} & 149 & \tiny{  0.243} &   3.392 & \tiny{  0.284}\\
 15 & \tiny{  0.085} &   1.410 & \tiny{  0.096} &  60 & \tiny{  0.308} &   4.926 & \tiny{  0.352} & 105 & \tiny{  0.399} &   5.937 & \tiny{  0.460} & 150 & \tiny{  0.237} &   3.296 & \tiny{  0.276}\\
 16 & \tiny{  0.090} &   1.502 & \tiny{  0.103} &  61 & \tiny{  0.312} &   4.982 & \tiny{  0.356} & 106 & \tiny{  0.398} &   5.919 & \tiny{  0.460} & 151 & \tiny{  0.230} &   3.199 & \tiny{  0.269}\\
 17 & \tiny{  0.096} &   1.594 & \tiny{  0.109} &  62 & \tiny{  0.316} &   5.037 & \tiny{  0.361} & 107 & \tiny{  0.397} &   5.899 & \tiny{  0.459} & 152 & \tiny{  0.223} &   3.101 & \tiny{  0.261}\\
 18 & \tiny{  0.102} &   1.686 & \tiny{  0.115} &  63 & \tiny{  0.320} &   5.091 & \tiny{  0.365} & 108 & \tiny{  0.397} &   5.877 & \tiny{  0.458} & 153 & \tiny{  0.216} &   3.002 & \tiny{  0.253}\\
 19 & \tiny{  0.107} &   1.777 & \tiny{  0.122} &  64 & \tiny{  0.323} &   5.143 & \tiny{  0.370} & 109 & \tiny{  0.396} &   5.854 & \tiny{  0.457} & 154 & \tiny{  0.209} &   2.901 & \tiny{  0.244}\\
 20 & \tiny{  0.113} &   1.868 & \tiny{  0.128} &  65 & \tiny{  0.327} &   5.194 & \tiny{  0.374} & 110 & \tiny{  0.395} &   5.828 & \tiny{  0.456} & 155 & \tiny{  0.202} &   2.800 & \tiny{  0.236}\\
 21 & \tiny{  0.118} &   1.958 & \tiny{  0.134} &  66 & \tiny{  0.330} &   5.243 & \tiny{  0.378} & 111 & \tiny{  0.394} &   5.801 & \tiny{  0.455} & 156 & \tiny{  0.195} &   2.697 & \tiny{  0.228}\\
 22 & \tiny{  0.124} &   2.048 & \tiny{  0.141} &  67 & \tiny{  0.334} &   5.292 & \tiny{  0.382} & 112 & \tiny{  0.392} &   5.772 & \tiny{  0.454} & 157 & \tiny{  0.187} &   2.593 & \tiny{  0.219}\\
 23 & \tiny{  0.129} &   2.138 & \tiny{  0.147} &  68 & \tiny{  0.337} &   5.338 & \tiny{  0.386} & 113 & \tiny{  0.391} &   5.740 & \tiny{  0.452} & 158 & \tiny{  0.180} &   2.488 & \tiny{  0.210}\\
 24 & \tiny{  0.134} &   2.227 & \tiny{  0.153} &  69 & \tiny{  0.341} &   5.384 & \tiny{  0.390} & 114 & \tiny{  0.389} &   5.707 & \tiny{  0.450} & 159 & \tiny{  0.172} &   2.382 & \tiny{  0.202}\\
 25 & \tiny{  0.140} &   2.315 & \tiny{  0.159} &  70 & \tiny{  0.344} &   5.428 & \tiny{  0.394} & 115 & \tiny{  0.387} &   5.672 & \tiny{  0.448} & 160 & \tiny{  0.165} &   2.275 & \tiny{  0.193}\\
 26 & \tiny{  0.145} &   2.403 & \tiny{  0.165} &  71 & \tiny{  0.347} &   5.470 & \tiny{  0.398} & 116 & \tiny{  0.386} &   5.635 & \tiny{  0.446} & 161 & \tiny{  0.157} &   2.167 & \tiny{  0.184}\\
 27 & \tiny{  0.151} &   2.490 & \tiny{  0.171} &  72 & \tiny{  0.350} &   5.511 & \tiny{  0.401} & 117 & \tiny{  0.384} &   5.596 & \tiny{  0.444} & 162 & \tiny{  0.149} &   2.059 & \tiny{  0.175}\\
 28 & \tiny{  0.156} &   2.577 & \tiny{  0.178} &  73 & \tiny{  0.353} &   5.551 & \tiny{  0.405} & 118 & \tiny{  0.381} &   5.555 & \tiny{  0.442} & 163 & \tiny{  0.142} &   1.949 & \tiny{  0.166}\\
 29 & \tiny{  0.161} &   2.663 & \tiny{  0.184} &  74 & \tiny{  0.356} &   5.589 & \tiny{  0.408} & 119 & \tiny{  0.379} &   5.512 & \tiny{  0.439} & 164 & \tiny{  0.134} &   1.839 & \tiny{  0.156}\\
 30 & \tiny{  0.167} &   2.749 & \tiny{  0.190} &  75 & \tiny{  0.359} &   5.625 & \tiny{  0.412} & 120 & \tiny{  0.377} &   5.467 & \tiny{  0.437} & 165 & \tiny{  0.126} &   1.727 & \tiny{  0.147}\\
 31 & \tiny{  0.172} &   2.834 & \tiny{  0.196} &  76 & \tiny{  0.362} &   5.660 & \tiny{  0.415} & 121 & \tiny{  0.374} &   5.421 & \tiny{  0.434} & 166 & \tiny{  0.118} &   1.616 & \tiny{  0.138}\\
 32 & \tiny{  0.177} &   2.918 & \tiny{  0.202} &  77 & \tiny{  0.365} &   5.694 & \tiny{  0.418} & 122 & \tiny{  0.371} &   5.372 & \tiny{  0.431} & 167 & \tiny{  0.109} &   1.503 & \tiny{  0.128}\\
 33 & \tiny{  0.182} &   3.002 & \tiny{  0.208} &  78 & \tiny{  0.367} &   5.726 & \tiny{  0.421} & 123 & \tiny{  0.368} &   5.322 & \tiny{  0.427} & 168 & \tiny{  0.101} &   1.390 & \tiny{  0.118}\\
 34 & \tiny{  0.188} &   3.085 & \tiny{  0.214} &  79 & \tiny{  0.370} &   5.756 & \tiny{  0.424} & 124 & \tiny{  0.365} &   5.270 & \tiny{  0.424} & 169 & \tiny{  0.093} &   1.276 & \tiny{  0.109}\\
 35 & \tiny{  0.193} &   3.167 & \tiny{  0.219} &  80 & \tiny{  0.372} &   5.784 & \tiny{  0.427} & 125 & \tiny{  0.362} &   5.215 & \tiny{  0.420} & 170 & \tiny{  0.085} &   1.162 & \tiny{  0.099}\\
 36 & \tiny{  0.198} &   3.248 & \tiny{  0.225} &  81 & \tiny{  0.375} &   5.811 & \tiny{  0.430} & 126 & \tiny{  0.359} &   5.159 & \tiny{  0.417} & 171 & \tiny{  0.076} &   1.047 & \tiny{  0.089}\\
 37 & \tiny{  0.203} &   3.329 & \tiny{  0.231} &  82 & \tiny{  0.377} &   5.837 & \tiny{  0.433} & 127 & \tiny{  0.355} &   5.101 & \tiny{  0.413} & 172 & \tiny{  0.068} &   0.932 & \tiny{  0.080}\\
 38 & \tiny{  0.208} &   3.409 & \tiny{  0.237} &  83 & \tiny{  0.379} &   5.861 & \tiny{  0.435} & 128 & \tiny{  0.352} &   5.042 & \tiny{  0.409} & 173 & \tiny{  0.060} &   0.816 & \tiny{  0.070}\\
 39 & \tiny{  0.213} &   3.488 & \tiny{  0.243} &  84 & \tiny{  0.381} &   5.883 & \tiny{  0.438} & 129 & \tiny{  0.348} &   4.980 & \tiny{  0.404} & 174 & \tiny{  0.051} &   0.700 & \tiny{  0.060}\\
 40 & \tiny{  0.218} &   3.566 & \tiny{  0.248} &  85 & \tiny{  0.383} &   5.903 & \tiny{  0.440} & 130 & \tiny{  0.344} &   4.917 & \tiny{  0.400} & 175 & \tiny{  0.043} &   0.584 & \tiny{  0.050}\\
 41 & \tiny{  0.223} &   3.644 & \tiny{  0.254} &  86 & \tiny{  0.385} &   5.922 & \tiny{  0.442} & 131 & \tiny{  0.340} &   4.851 & \tiny{  0.395} & 176 & \tiny{  0.034} &   0.467 & \tiny{  0.040}\\
 42 & \tiny{  0.228} &   3.720 & \tiny{  0.260} &  87 & \tiny{  0.387} &   5.939 & \tiny{  0.444} & 132 & \tiny{  0.336} &   4.784 & \tiny{  0.390} & 177 & \tiny{  0.026} &   0.351 & \tiny{  0.030}\\
 43 & \tiny{  0.233} &   3.796 & \tiny{  0.265} &  88 & \tiny{  0.388} &   5.954 & \tiny{  0.446} & 133 & \tiny{  0.331} &   4.716 & \tiny{  0.386} & 178 & \tiny{  0.017} &   0.234 & \tiny{  0.020}\\
 44 & \tiny{  0.238} &   3.871 & \tiny{  0.271} &  89 & \tiny{  0.390} &   5.967 & \tiny{  0.448} & 134 & \tiny{  0.327} &   4.645 & \tiny{  0.380} & 179 & \tiny{  0.009} &   0.117 & \tiny{  0.010}\\
 45 & \tiny{  0.242} &   3.944 & \tiny{  0.276} &  90 & \tiny{  0.391} &   5.979 & \tiny{  0.450} & 135 & \tiny{  0.322} &   4.573 & \tiny{  0.375} & 180 & \tiny{  0.000} &   0.000 & \tiny{  0.000}\\
\end{tabular}}
\caption[\em Epicyclic anomalies of Saturn.]{\em Epicyclic anomalies of Saturn. All quantities are in degrees. Note that $\bar{\theta}(360^\circ-\mu) = -\bar{\theta}(\mu)$, and $\delta\theta_{\pm}(360^\circ-\mu) = -\delta\theta_{\pm}(\mu)$. }\label{vt16}
\end{table}

\newpage
\begin{table}\centering
{\small\begin{tabular}{lcl}
Event & Date &$\lambda$ \\\hline
&&\\[-1.75ex]
Conjunction & 10/05/2000 & 20TA26\\
Station (R) & 12/09/2000 & 00GE59\\
Opposition & 19/11/2000 & 27TA29\\
Station (D) & 24/01/2001 & 24TA03\\
Conjunction & 25/05/2001 & 04GE22\\
Station (R) & 26/09/2001 & 14GE59\\
Opposition & 03/12/2001 & 11GE29\\
Station (D) & 08/02/2002 & 08GE02\\
Conjunction & 09/06/2002 & 18GE28\\
Station (R) & 11/10/2002 & 29GE06\\
Opposition & 17/12/2002 & 25GE36\\
Station (D) & 22/02/2003 & 22GE08\\
Conjunction & 24/06/2003 & 02CN39\\
Station (R) & 25/10/2003 & 13CN15\\
Opposition & 31/12/2003 & 09CN46\\
Station (D) & 07/03/2004 & 06CN17\\
Conjunction & 08/07/2004 & 16CN50\\
Station (R) & 08/11/2004 & 27CN21\\
Opposition & 13/01/2005 & 23CN52\\
Station (D) & 22/03/2005 & 20CN23\\
Conjunction & 23/07/2005 & 00LE56\\
Station (R) & 22/11/2005 & 11LE19\\
Opposition & 27/01/2006 & 07LE51\\
Station (D) & 05/04/2006 & 04LE22\\
Conjunction & 07/08/2006 & 14LE51\\
Station (R) & 06/12/2006 & 25LE04\\
Opposition & 10/02/2007 & 21LE38\\
Station (D) & 19/04/2007 & 18LE09\\
Conjunction & 22/08/2007 & 28LE32\\
Station (R) & 19/12/2007 & 08VI34\\
Opposition & 24/02/2008 & 05VI10\\
Station (D) & 03/05/2008 & 01VI41\\
Conjunction & 04/09/2008 & 11VI56\\
Station (R) & 31/12/2008 & 21VI46\\
Opposition & 08/03/2009 & 18VI23\\
Station (D) & 17/05/2009 & 14VI56\\
Conjunction & 17/09/2009 & 25VI01\\
Station (R) & 13/01/2010 & 04LI40\\
Opposition & 22/03/2010 & 01LI18\\
Station (D) & 30/05/2010 & 27VI51\\
Conjunction & 01/10/2010 & 07LI46\\
\end{tabular}}
\caption[\em  The conjunctions, oppositions, and stations of Saturn
during the years 2000--2010 CE. ]{\em The conjunctions, oppositions, and stations of Saturn
during the years 2000--2010 CE. (R) indicates a retrograde station, and (D)
a direct station.}\label{vtsaturn}
\end{table}
